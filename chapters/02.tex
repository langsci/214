\documentclass[output=paper]{langscibook} 
\ChapterDOI{10.5281/zenodo.2654351}
\author{James McElvenny\affiliation{University of Edinburgh}}
\title{Alternating sounds and the formal franchise in phonology}
\label{chap:mcelvenny} 

\abstract{A matter of some controversy in the intersecting worlds of late nineteenth-century linguistics and anthropology was the nature of ``alternating sounds''. This phenomenon is the apparent tendency, long assumed to be characteristic of ``primitive'' languages, to freely vary the pronunciation of words, without any discernible system. Franz Boas (1858--1942), rebutting received opinion in the American anthropological establishment, denied the existence of this phenomenon, arguing that it was an artefact of observation. Georg von der Gabelentz (1840--1893), on the other hand, embraced the phenomenon and fashioned it into a critique of the comparative method as it was practised in Germany.

Both Boas and Gabelentz -- and indeed also their opponents -- were well versed in the Humboldtian tradition of language scholarship, in particular as developed and transmitted by H. Steinthal (1823--1899). Although the late nineteenth-century debates surrounding alternating sounds were informed by a number of sources, this chapter argues that Steinthal's writings served as a key point of reference and offered several motifs that were taken up by his scholarly successors. In addition, and most crucially, the chapter demonstrates that the positions at which the participants in these debates arrived were determined not so much by any simple technical disagreements but by underlying philosophical differences and sociological factors. This episode in the joint history of linguistics and anthropology is telling for what it reveals about the dominant mindset and temperament of these disciplines in relation to the formal analysis of the world's languages.\largerpage[-1]
}

\begin{document}
\maketitle

\section{Introduction}
\label{sec:mcelvenny:intro}

Phonology is in many ways the promised land of formal conceptions of language. The apparent orderly transmutation of sounds over time stimulated the mechanical minds of historical-comparative linguists, ultimately inspiring the Neogrammarians to their postulation of exceptionless sound laws. The vanguard of linguistic formalism in subsequent generations continued to look to sound patterns -- although now chiefly in their synchronic aspect -- as the pristine embodiment of the self-contained systems they sought. In this way, the classical American \isi{structuralist} grammar sets out from the firm ground of phonology and ascends to increasingly less regular linguistic levels.

But a question that remained controversial into the last decades of the nineteenth century was just how far the formal franchise in phonology should be extended. Do the sound systems of all languages of the world meet the standards of arbitrariness and regularity identified in the Indo-European languages? An apparent phenomenon prevalent in the ``primitive'' languages of the Americas, Africa and the South Seas suggested limits to law-governed language. European scholars and adventurers who tried to learn and transcribe the words of these languages were frequently frustrated by the way in which native informants would seemingly change the pronunciation of the same word from utterance to utterance. From the perspective of present-day phonological theory, this phenomenon would be considered variously a manifestation of free variation, allophonic variation and difficulty perceiving articulations markedly foreign to the recorder's own phonological system. Nineteenth-century scholars, by contrast, conceptualized this phenomenon in a number of different, competing ways. These differences in conceptualization led to terminological instability, but a common cover term, also adopted here, was ``alternating sounds''.

This chapter explores some responses from prominent language scholars in the mid- to late nineteenth century to the phenomenon of alternating sounds, and looks at what these responses reveal about the underlying philosophical commitments and sociological structure of the intersecting fields of anthropology and linguistics in this era. The investigation spans the intellectual worlds of America and Germany which, although closely intertwined, were organized around different disciplinary structures. The figures featured here who were active in America described themselves as anthropologists, for whom linguistic research was one of the ``four fields'' of American anthropology.\footnote{The essays contained in \citet{Kuklick2008} provide an excellent comparative overview of the history of anthropology in America and Europe, including their disciplinary structures.} The corresponding German discussion, on the other hand, took place largely within the discipline of linguistics, in which the study of ``exotic'' languages was a niche pursuit. The exception is the work of H. Steinthal (1823--1899), who is put forward in this chapter as an inspiration to -- and therefore link between -- both the German and American worlds. His \emph{Völkerpsychologie}, developed with his collaborator M. Lazarus (1824--1903), strove to offer an all-encompassing scientific account of human culture, history and society.\footnote{For a detailed account of \emph{Völkerpsychologie} from its beginnings with Steinthal and Lazarus to its later developments and ultimate fate, see \citet{Klautke2013}.}

The starting point for this chapter, in \sectref{sec:mcelvenny:america}, is the \citeyear{Boas1889} paper ``On alternating sounds'' by Franz Boas (1858--1942), a milestone marking the way to modern explanations of alternating sounds and modern views on the equality of all languages. Here Boas rebutted the received position of the American anthropological establishment, represented in particular by such luminaries as Daniel Garrison Brinton (1837--1899) and John Wesley Powell (1834--1902), which held that the alternating sounds observed in American languages were a manifestation of their alleged primitiveness. Boas argued, by contrast, that the alternating sounds were an illusion caused by the conflicts of incommensurable phonological systems in informant and ethnographer.

From a present-day perspective, this episode may seem like a simple case of science triumphing over naivety and prejudice. But arguments presented on both sides of the American debate could claim some degree of theoretical sophistication. Indeed, Brinton and Boas shared a key source of theoretical inspiration in the work of Steinthal, whose views were in turn anchored in the linguistic writings of Wilhelm von Humboldt (1767--1835). While phonological issues occupy at most a peripheral place in Steinthal's work, aspects of his linguistic and psychological theory would seem to have informed the later debate. \sectref{sec:mcelvenny:humboldtian} offers an account of the nuanced views advanced by Steinthal and their possible links to later arguments.

Despite its now canonical status, the American debate was not the only reconsideration of principles of phonological regularity around the turn of the nineteenth to the twentieth century. In Germany, Georg von der Gabelentz (1840--1893), also drawing on the Humboldtian tradition as transmitted by Steinthal, affirmed the existence of alternating sounds, in a turn that could be seen as prefiguring key features of later phonemic theory. Like Boas, Gabelentz fashioned his treatment of alternating sounds into a critique of the linguistic establishment. But unlike Boas, Gabelentz' goal was not to extend the formal franchise to all languages, but rather to redefine it and thereby challenge the \isi{comparative method} as it was practised at the time. \sectref{sec:mcelvenny:gabelentz} looks at Gabelentz' proposals for alternative methods in historical-comparative linguistics and their rather unfavourable reception. 

Finally, \sectref{sec:mcelvenny:conc} brings the American and German debates together to discuss what they reveal about the dominant mindset and temperament in the intersecting fields of linguistics and anthropology in relation to questions of the nature and correct treatment of linguistic form.

\section{Alternating sounds in America}
\label{sec:mcelvenny:america}

Boas' (\citeyear{Boas1889}) ``On alternating sounds'' occupies a prominent place in the standard disciplinary narrative of linguistic anthropology as a text that helped to establish the scientific foundations of the field. According to this story, Boas overcame contemporary evolutionary prejudice by demonstrating that an alleged characteristic of ``primitive'' languages was in fact nothing more than an artefact introduced by insufficiently trained observers.\footnote{``Evolutionary prejudice'' was the term later used by Boas' student Edward Sapir (1884--1939) to describe the assumption that the world's languages can be categorized according to their putative level of grammatical development \citep[see][130--132]{Sapir1921}.} Alternating sounds, in various guises, were a recurring motif in the description of exotic languages throughout the nineteenth century, but the two key figures against whom Boas developed his position were Brinton and Powell, the leading anthropologists of the previous generation.\footnote{On the relationship between Boas, Brinton and Powell in the context of late nineteenth-century American anthropology, see \citet{Darnell1988} and \citet{Darnell1998}. See also Laplantine's (\citeyear{Laplantine2018}) preface to her translation of Boas' (\citeyear{Boas1911}) \emph{Handbook of American Indian Languages} for a succinct summary of his life and work in context.}

In the year before Boas' seminal article appeared, Brinton reaffirmed several tropes about ``primitive'' languages in an 1888 address to the American Philosophical Society, ``The Language of Palæolithic Man'', which in an 1890 volume of his collected papers became ``The earliest form of human speech, as revealed by American tongues'' \citep{Brinton18901888}. As the titles suggest, Brinton sought insights into the nature of the earliest stages of human \isi{language evolution} through an examination of the supposedly characteristic features of American languages. While much of Brinton's paper focuses on the lexical and grammatical properties of these languages, it begins with a discussion of their phonological features.

Primitive speech, in Brinton's assessment, has not yet attained the levels of arbitrariness and fixedness that characterize the more developed languages: in European languages individual sounds carry no sense, words have fixed sound forms, and the articulated word alone is enough to convey its meaning. American languages, by contrast, frequently attach meaning to individual phonetic segments \citep[394]{Brinton18901888}, word meaning is often modified by such devices as ``[t]one, accent, stress, vocal inflection, quantity and pause'' \citep[399]{Brinton18901888} that are not reducible to graphic writing, and sounds in words can vary freely: ``In spite of the significance attached to the phonetic elements, they are, in many American languages, singularly vague and fluctuating'' \citep[397]{Brinton18901888}. His concluding observation is that ``[t]he laws of the conversion of sounds of the one organ into those of another have not yet been discovered; but the above examples, which are by no means isolated ones, serve to admonish us that the phonetic elements of primitive speech probably had no fixedness'' \citep[398--399]{Brinton18901888}.

Under the name of ``synthetic sounds'', this same phenomenon of apparent fluctuating phonology in American languages found a place in Powell's (\citeyear{Powell18801877}) \emph{Introduction to the Study of Indian Languages}. Given Powell's influential position as director of the Bureau of American Ethnology, which was founded on his initiative in 1879, the \emph{Introduction} achieved widespread use in the recording of American languages, not only in projects officially sponsored by the Bureau, but also in the efforts of other researchers and amateurs, including Boas and his students \citep[see][50--51]{Darnell1998}.

Powell was very conscious of the difficulties associated with capturing the phonology of American languages\rephrase{, and h}{. H}is commitment to scientific rigour led him to commission the noted Sanskrit scholar and general linguist William Dwight Whitney (1827--1894) to devise a standardized alphabet for recording American languages. Despite Powell's efforts to encourage its use, the alphabet was generally considered inadequate and impractical by many of those who worked for the Bureau. Whitney himself felt no great attachment to the alphabet, regarding its design and implementation not as a theoretical task but merely a matter of expedience \citep[see][50--51]{Darnell1998}. For Powell, however, the alphabet was a foundational element of language description: his \emph{Introduction} opens with a sophisticated discussion of articulatory phonetics and the principles of accurate transcription, which observes a number of phonological peculiarities of American languages still recognized today, such as ejective consonants (``interrupted sounds'') \citep[1--16]{Powell18801877}.

``Synthetic sounds'' appear in this discussion as another characteristic of American phonologies. \citet[12]{Powell18801877} speaks of the ``indefinite character of some of the sounds of a[n American Indian] language'', although this is not due to the chaotic variation imagined by Brinton but rather because the sounds are ``made by the organs of speech in positions and with movement comprehending in part at least the positions and movement used in making the several sounds to which they seem to be allied''. That is, Powell believes these ``synthetic'' sounds are insufficiently ``differentiated'' -- they are produced by articulating several simple sounds at once. Through historical sound change, such synthetic sounds have been simplified and disappeared from the European languages, but this is a process yet to take place in the American languages. In their present undifferentiated state, these sounds ``will be heard by the student now as one, now as another sound, even from the same speaker.'' There is, however, a trace of humility in Powell's approach to the American languages, an admission that science may not yet have fully grasped the principles underlying this phenomenon: ``When the phonology of our Indian tongues is thoroughly understood, much light will be thrown upon the whole science of phonology […]'' \citep[13]{Powell18801877}.

In response to views of the kind put forward by Brinton and Powell, Boas argued that such sounds are not a peculiarity of primitive languages at all, but rather the result of perceptual error on the part of the language researcher. All languages, European and American alike, make use of a fixed and finite repertoire of the total range of sounds that can be produced by the human articulatory organs. When an observer encounters a sound in a foreign language that is not present in their native repertoire, they will ``apperceive'' it as a similar sound that is in their repertoire. A term with a long history and a diverse range of uses, ``apperceive'' became in the early nineteenth century part of the technical apparatus of Johann Friedrich Herbart's (1776--1841) associational psychology, from where it was taken up into the \emph{Völkerpsychologie} of Steinthal and Lazarus, and later into the \emph{Bewusstseinspsychologie} of Wilhelm Wundt (1832--1920).\footnote{For a recent survey of approaches to what can retrospectively be called ``psycholinguistics'' in this period, including the work of Lazarus, Steinthal and Wundt, see \citet{Levelt2013}.} Boas' invocation of ``apperception'' is too fleeting and off-hand to align him with any specific school of psychology at the time, but his usage attests to a familiarity with contemporary psychological jargon and a desire to dress his own work in the latest technical garb.

According to Boas, the mapping from foreign to native sound that results through the process of apperception may vary from occasion to occasion, creating the illusion of alternating sounds. The presence of this perceptual filter on the part of the observer is demonstrated by the fact that ``the nationality even of well-trained observers may be readily recognized'' in the transcriptions they make of foreign sounds \citep[51]{Boas1889}. Boas sums up his argument with the following words:

\begin{quotation}
I think, from this evidence, it is clear that all such misspellings are due to a wrong apperception, which is due to the phonetic system of our native language. For this reason I maintain that there is no such phenomenon as synthetic or alternating sounds, and that their occurrence is in no way a sign of primitiveness of the speech in which they are said to occur; that alternating sounds are in reality alternating apperceptions of one and the same sound. A thorough study of all alleged alternating sounds or synthetic sounds will show that their existence may be explained by alternating apperceptions. \citep[52]{Boas1889}
\end{quotation}

Boas was no doubt correct to impugn the perception of his colleagues in many cases where they accused American languages of phonetic fluctuation. But it must be acknowledged that the potential for cross-linguistic phonological interference was already well recognized in the literature of the time. \citet[2]{Powell18801877} noted this difficulty in his own guide to transcription:

\begin{quotation}
[T]here are probably sounds in each [Indian language of North America] which do not appear in the English or any other civilized tongue; […] and further, […] there are perhaps sounds in each of such a character, or made with such uncertainty that the ear primarily trained to distinguish English speech is unable to clearly determine what these sounds are, even after many years of effort. \citep[2]{Powell18801877}
\end{quotation}

As is shown in the following sections, this awareness of cross-linguistic interference is clear in many other contemporary and antecedent sources, where it co-existed with a range of different attitudes towards alternating sounds. A scholar's stance in relation to these questions was therefore shaped to a very large degree by beliefs and commitments beyond the immediate language data.

A key motivation for Boas was of course to subvert the then current discourse of primitive languages and \isi{language evolution}. But this was not his only aim, and indeed this subversion was at least in part beholden to other goals. Although he enjoyed mostly respectful and collegial relations with both Brinton and Powell, Boas was always engaged in a project to proclaim his superior scientific expertise and secure institutional support for his coterie of students and adherents. The chief and most valid source of data in Boasian anthropology were the descriptions made and texts recorded by the scientifically trained observer in a fieldwork situation. By contrast, Brinton, the doyen of the previous generation, relied mainly on the critical philological analysis of written documents that had been collected and compiled by others \citep[see][21--24]{Darnell1988}. By diminishing existing written documentation, Boas' critique undermined the legitimacy of the mode of research employed by Brinton and boosted his own fieldwork-oriented approach.

Even among confirmed fieldworkers, Boas' critique helped to assert the exclusive expertise of his own school. In later years, Boas developed a reputation for his domineering role in the world of Americanist anthropology,\label{q:mcelvenny:domineering} freely blocking the work of researchers who did not meet his frequently quite arbitrary standards \citep[see][]{Darnell1998}. Pointing out the technical inadequacies of his predecessors, as in the case of alternating sounds, served this end well. In his \citeyear{Boas1911} \emph{Handbook of American Indian Languages}, which was explicitly intended to supersede Powell's (\citeyear{Powell18801877}) \emph{Introduction}, Boas' doctrine of the conditioned apperception of foreign sounds is incorporated as part of the propaedeutic guide to the correct recording of American languages, as a simple and uncontroversial methodological principle \citep[see][16--18]{Boas1911}.

That assertions of expertise are a decisive factor in Boas' campaign is demonstrated by his enduring commitment to the possibility of objective observation in language documentation. While previous transcribers of American languages may have been afflicted with a phonological filter, the goal of the Boasian anthropologist must be to eliminate this interference altogether. Even after the importation and elaboration of phonemic theory in America, Boas maintained a preference for fine-grained phonetic transcription. It was not enough for the observer to simply enter the foreign phonological system; they had to step outside phonology and record the given phonetic datum as accurately as possible.\footnote{Another perspective from which Boas' position should perhaps be explored is that of contemporary debates on the ``personal equation'' in recording data, which were prominent across the natural sciences \citep[see][]{Schaffer1988} and also played a role in attitudes to fieldwork in anthropology \citep[see][]{Kuklick2011}. I thank Judith Kaplan for drawing my attention to these debates.} Boas' zeal extended to correcting written texts from one of his native speaker informants, which were essentially phonemic in nature, to include as much phonetic detail as possible \citep[see][204--208]{Anderson1985}. Even the phonemic testimony of the native speaker did not pass Boasian muster.\footnote{A further piece of circumstantial evidence is perhaps Boas' work on a revised standard alphabet for American languages. After Powell's death in 1902, Boas was asked by William John McGee (1853--1912), Powell's successor at the Bureau of American Ethnology, to form a committee to update the Bureau's alphabet. The resulting system, published 1916, clearly contains many compromises between various conflicting constraints, but the overall Boasian impulse towards greater phonetic detail and specialist exclusivity is quite apparent \citep[see][195--197]{Darnell1998}.}

\section{Steinthal and the Humboldtian tradition}
\label{sec:mcelvenny:humboldtian}

The American debate on alternating sounds was shaped by a number of influences: the three figures mentioned in the previous section -- Brinton, Powell and Boas -- all had broad backgrounds spanning the natural sciences and humanities that informed their attitudes and approaches \citep[see][]{Darnell1998}. But a central point of reference -- in particular for Brinton and Boas -- was the Humboldtian tradition of linguistic scholarship as it was interpreted and propagated by Steinthal. Boas had met Steinthal personally in Berlin and freely acknowledged Steinthal's influence on his linguistic research. Brinton was the leading Humboldt scholar in America and frequently cited Steinthal (\citealt[see][63--69]{Bunzl1996}; \citealt[289--292]{TrautmannWaller2006}). Although the questions of phonology and language documentation that lay at the heart of the American debate on alternating sounds are addressed only at the periphery of Steinthal's work, we see in his texts several threads unpicked and woven into the later accounts.

Steinthal's great achievement in linguistics was to construct a monolithic theoretical edifice dealing with issues ranging from the mental processes underlying individual language use to \isi{language evolution} and typology, and to attempt an empirical demonstration of these principles through detailed investigations into the languages of the world. Through his collaboration with Lazarus from the 1850s onwards, Steinthal's linguistics became a central component of the broader project of \emph{Völkerpsychologie}.\footnote{\citet{TrautmannWaller2006} is a comprehensive intellectual biography of Steinthal, which examines his linguistic work and \emph{Völkerpsychologie} in depth. For studies of Steinthal's linguistics, see \citet{Bumann1965} and \citet{Ringmacher1996}.} 

Following Humboldt, Steinthal imagined an ``idea of language'' (\emph{Sprachidee}), an ideal form towards which linguistic expression strives. The evolution of language passes through three stages on the way to the full realization of this ideal; these stages are recapitulated in child language acquisition and can be discerned in the contours of ``primitive'' languages \citep[cf.][81--93]{Bumann1965}. The first stage consists in self-awareness, the psychological attainment that distinguishes humans from animals. Unlike animals, humans can represent, share and understand their ``intuitions'' (\emph{Anschauungen}), which they ``apperceive'' (\emph{appercieren}) in their consciousness. Here Steinthal invokes the core notion of ``apperception'' from Herbartian associational psychology; this is the same term that Boas would later use in generic form (see \sectref{sec:mcelvenny:america} above). 

At the first stage language is made up of nothing more than ``reflex sounds'' (\emph{Reflexlaute}), which merely represent and communicate intuitions in an unanalysed way. These sounds are brought forth through unreflected action and are solely mimetic in character. The further development of language occurs as speakers become increasingly aware of the thoughts they entertain in consciousness and begin to analyse them. At the second stage, language progresses beyond reflex sounds to a proper conscious analysis of thoughts. It is at this point that sentence structure develops, with a distinction between subject and predicate and individual words that can be abstracted from the sentence as a whole:

\begin{quotation}
It is therefore already at the point where language first appears in its true quality, where it achieves its full intellectual character, that it breaks through onomatopoeia. And \emph{words in their true conception develop only with the development of the sentence form; that is, simultaneously with the opposition of subject and predicate}, which soon establishes itself as the difference in the naming of things and expressions for circumstances and changes. The logical character of words seems to be decisively hostile to their onomatopoeic origin. \citep[424--425]{Steinthal1881}\footnote{Original: ``Also gerade schon da, wo die Sprache zuerst in ihrer wahren Eigentümlichkeit auftritt, wo sie ihren vollen intellektuellen Charakter gewinnt, durchbricht sie die Onomatopoie; und \emph{das Wort in seinem wahren Begriff entsteht erst mit der Satzform, also zugleich mit dem Gegensatze von Subjekt und Prädikat}, der sich bald zu dem Unterschiede der Benennung von Dingen und der Ausdrücke für Zustände und Veränderungen festsetzt. Der logische Charakter des Wortes scheint dem onomatopoetischen Ursprunge desselben entschieden feindlich zu sein.'' Italics in this quotation renders \emph{Sperrung} in the original.}
\end{quotation}

At the third stage of evolution, language continues its ascent from its mimetic origins: the etymological bond between words and their meanings fades from consciousness and the connection between them becomes truly arbitrary.

For Steinthal, the crucial moment in \isi{language evolution} is the second stage, \rephrase{since}{as } this is the point at which ``\isi{inner linguistic form}'' (\emph{innere Sprachform}) emerges \citep[425--426]{Steinthal1881}. ``Inner linguistic form'' is a term that first appears in Humboldt's (\citeyear{Humboldt19981836}) introduction to his work on the Kawi language of Java. The term is therefore generally associated with Humboldt, even though, as \citet{Borsche1989} definitively demonstrated, its elaboration into a theoretical construct is the later work of Steinthal. In Steinthal's hands, \isi{inner form} became a wide-ranging concept covering all aspects of the immanent structure of languages. Like ``apperception'', ``\isi{inner form}'' grew in the second half of the nineteenth century into a favourite but rather indefinite term in linguistic and philosophical scholarship. Despite the explosion of senses attached to the term in this period, its ultimate origin in Humboldt's essay and its deep association with Steinthal's work remained foremost in the minds of those who employed it. Both Brinton and Boas keenly spoke this idiom and acknowledged the tradition with which it was aligned: Brinton constantly advocated for attention to the \isi{inner form} of languages and \citet[81]{Boas1911} set capturing the unique \isi{inner form} of each language as the goal of the language sketches in his \emph{Handbook} \citep[cf.][98--105]{Darnell1988}. 

Steinthal's typological efforts were aimed at assessing how far towards the ``idea of language'' the \isi{inner form} had progressed in different languages and at identifying the grammatical means -- such as morphological or syntactic structures -- in which it manifests itself. His \citeyear{Steinthal1860} \emph{Charakteristik der hauptsächlichsten Typen des Sprachbaues} provided a survey and classification of the world's languages, in which the primary division is between those language with properly developed \isi{inner form} (\emph{Formsprachen}) and those without (\emph{formlose Sprachen}). This work was followed by his \citeyear{Steinthal1867} \emph{Mande-Neger-Sprachen}, which subjected several Mande languages of Africa -- Mandingo, Bambara, Soso and Vai -- to a detailed examination that revealed alleged developmental deficiencies in all aspects of their inner forms. This examination is based on a philological analysis of existing written sources, similar to the preferred research practice of Brinton. The analysis proceeds from both a ``phonetic'' (\emph{phonetisch}) perspective, which looks at the grammatical apparatus of the languages, and a ``psychological'' (\emph{psychologisch}) perspective, which investigates how expressions are formed.\footnote{For a discussion of the historical background to this dual-perspective approach to language description, see \citet[2--6]{McElvenny2017}.}

In his ``phonetic'' examination of the Mande languages, Steinthal found no way to distinguish individual words from the sentences in which they appear: there are allegedly no phonological processes observed to operate only at the word level distinct from the sentence as a whole. In their grammars, the languages supposedly rely on mechanisms that are not truly arbitrary, such as the ``interjectional'' process of reduplication, used for a variety of purposes in the languages. The grammatical affixes and particles that can be identified in the languages all seem to have transparent etymologies that link them to ``material'' words, which keep them bound to their mimetic origins. 

From the ``psychological'' perspective, the Mande languages did not fare any better. Steinthal's assessment of how various meanings are rendered using the lexical and grammatical means available in the languages reveals that the Herbartian processes of ``isolation'' (\emph{Isolirung}) and ``condensation'' (\emph{Verdichtung}) of ``representations'' (\emph{Vorstellungen}) in the minds of speakers are not carried out properly. The inevitable conclusion for \citet[255]{Steinthal1867} is that the speakers of Mande languages have not completely raised their ``intuitions'' to the level of ``representations'': ``in the consciousness of the Mande negro the concrete intuition with its material relations is still dominant, and its conversion into representations is not carried out completely''.\footnote{Original: ``im Bewußtsein des Mande-Negers ist die concrete Anschauung mit ihren materiellen Verhältnissen noch vorwiegend, und ihre Umsetzung in Vorstellungen ist unvollständig vollzogen.''}

Up to this point, Brinton's account of the ``primitive'' phonological features of American languages accords well with Steinthal's story of \isi{language evolution}. The alleged lack of arbitrariness and fixedness Brinton identified in the sounds of American languages are features that could be expected of languages at Steinthal's first stage of evolution. Steinthal in fact considered the possibility that a lack of arbitrariness in the earliest languages could lead to greater variability, since the sounds produced by reflex are bound to the mental moment and subject to all of its modifications:

\begin{quotation}
We may think that language, as long as it is still the immediate creation of the excited soul, shares in the fluctuations and inequalities of these excitations. So just as the representation, even though its content is the same, is not always the same in its psychological behaviour -- e.g. not always as lively and energetic to the same degree, vivid, strongly concentrated -- the word, as the reflex of this representation, is not always the same. The energy of thinking expresses itself most immediately in intonation, then also in the sharpness of articulation, i.e. the clearness and definiteness of the sound. And both together most certainly influence the quality or even the content of the sound, the way in which it is articulated. \citep[3--4]{Steinthal1867}\footnote{Original: ``Wir dürfen uns denken, daß die Sprache, so lange sie noch die unmittelbare Schöpfung der erregten Seele ist, auch an den Schwankungen und Ungleichheiten dieser Erregungen Theil hat. Wie also die, obschon ihrem Inhalte nach gleiche und selbe, Vorstellung doch in ihrem psychologischen Verhalten nicht immer gleich ist, z.\,B.\ nicht immer gleich lebendig und energisch, gleich anschaulich, gleich kräftig concentrirt: so lautet auch das Wort, als der Reflex dieser Vorstellung, nicht immer gleich. Die Energie des Denkens drückt sich am unmittelbarsten in der Weise der Betonung aus, dann auch in der Schärfe der Articulation, d.~h. der Klarheit und Bestimmtheit des Lautes; und beides zusammen beeinflußt sicherlich die Qualität oder den Inhalt selbst des Lautes, die Weise seiner Articulation.''}
\end{quotation}

But such questions remained hypothetical for Steinthal. According to \citet[3--4]{Steinthal1867}, the ``uncivilized peoples'' (\emph{culturlose Völker}) living today are not the \emph{Natur-Völker} of the earlier stages of human evolution. He accepted a greater degree of variation in the sounds of the languages of ``uncivilized peoples'' because these languages lacked the stabilizing and standardizing influence of an orthography, but even before the invention of writing, human language will ``have established itself in the consciousness'' and ``its word forms [will] have crystalized in definite shape'' (\emph{[…] hat sich die Sprache im Bewußtsein gefestigt, sind ihre Wortformen in bestimmter \isi{Gestalt} krystallisirt}; \citealt[4--5]{Steinthal1867}). 

In Steinthal's estimation, the Mande languages find themselves in this situation: they stand uneasily on the threshold to the second stage of evolution, but their apparent phonetic inconstancy in comparison with European languages is not due to enduring mimetic reflexes but simply anarchy arising from the absence of a regulating instance. \citet[257--266]{Steinthal1867} discounts the fact that the Vai do indeed possess a native writing system, since it is an imitation of European scripts fashioned without proper understanding of those scripts' underlying principles. The result is a massive syllabary -- of over 200 characters -- lacking system and internal order, which is chiefly used by distinguished members of the community to write books containing ``tales from the life of their authors, sayings, observations and fables -- without any unity'' \citep[260]{Steinthal1867}.\footnote{Original: ``Der Inhalt dieser Bücher besteht in der Erzählung von Ereignissen aus dem Leben ihrer Verfasser, in Sittensprüchen, Betrachtungen und Fabeln -- ohne alle Einheit.''} While the Vai may have a script, they do not have an orthography: they simply transcribe whatever pronunciation occurs to them as they write \citep[264--266]{Steinthal1867}, and this can vary even within the same text.\footnote{\citet{Steinthal1852} presented an account of the development of writing from ideographic systems to alphabets. Like his language typology, this represented an evolutionary scheme in which language users became progressively more aware of the structure of their languages. The Vai syllabary has reached the upper echelons of a pure phonetic script -- i.e. a script without ideographic elements -- but has not reached the highest point of a full alphabetic script \citep[262--264]{Steinthal1867}. The place of the Vai script in this hierarchy does not bear directly on the question of its consistency.}

The perceptual problems to which \citet{Boas1889} attributed alternating sounds in American languages were acknowledged by Steinthal in the case of the Mande languages. \citet{Steinthal1867} critiqued the transcriptions found in all of his sources, commenting, among other observations, that the influence of the transcriber's native phonology and writing habits had to be taken into consideration. On his English sources, he remarked:

\begin{quotation}
Since we frequently have to rely on English works, the influence of the English ear and English orthography must be taken into account. However, although this influence may be responsible for some things, it is hardly responsible for everything. The same sources offer at times, both consciously and unconsciously, double forms, e.g. \emph{bombong} and \emph{bambang}, ``hard'' […]. The most frequent alternation is perhaps that between \emph{i} and \emph{e}. \citep[9]{Steinthal1867}\footnote{Original: ``Da wir mehrfach auf englische Arbeiten angewiesen sind, so darf hierbei der Einfluß des englischen Ohrs und der englischen Orthographie nicht unberücksichtigt bleiben. Indessen, er mag manches verschulden, schwerlich alles. Dieselben Quellen stellen zuweilen unbewußt und bewußt doppelte Formen auf; z.\,B.\ \emph{bombong} und \emph{bambang}, hart […]. Am meisten vielleicht wechseln \emph{i} und \emph{e} mit einander.''}
\end{quotation}

While phonology as such was never among Steinthal's core concerns, in his empirically oriented researches he was inevitably confronted with the practical difficulties that arise in reducing to writing the sounds of exotic languages with no native orthographic tradition. The dangers he identified in written materials produced by European observers were precisely those that Boas would later turn into the fatal failures of perception on the part of his predecessors. On the other hand, the corroboration Brinton provided for existing accounts of phonetic fluctuation in American languages could in principle be licensed by Steinthal's scheme of \isi{language evolution}, although Steinthal explicitly denied that any language spoken today would still find itself at this most elementary stage. Steinthal accepted greater degrees of variation in the languages of ``uncivilized peoples'', but only because they lacked a standard imposed by authority.

\section{Phonetic latitude and sound laws}
\label{sec:mcelvenny:gabelentz}

Around the same time that Boas launched his attack against alternating sounds \mbox{-- but} independently of the American debate -- Georg von der Gabelentz marshalled related phonetic phenomena to mount a critique of the linguistic establishment in Germany. His opponents were the Neogrammarians, whose work was built upon an insistence on the exceptionless nature of sound change, and Gabelentz embraced the prospect of relative regularity in languages as a means to undermining this fundamental Neogrammarian tenet. As in the American context, a key theoretical reference in Germany -- in particular for Gabelentz -- was the work of Steinthal.

In his magnum opus, \emph{Die Sprachwissenschaft}, \citet[341--384]{Gabelentz20161891} undertakes an extensive investigation into contemporary linguistic typology that is essentially organized around the principles espoused by Steinthal.\footnote{\citet{Gabelentz1889} had already presented key parts of this section of his book in an address to the Saxon Academy of Sciences. An English translation can be found in \citet{McElvenny2019}.} Gabelentz rejected the strong distinction between ``formal'' and ``material'' elements in language hypothesized by Steinthal and used by him to demonstrate the alleged inferior mental development of speakers of the Mande and other languages. Instead, argued \citet[380--384]{Gabelentz20161891}, linguistic form is the product of an aesthetic drive to achieve subjective self-expression.\footnote{Jean-Michel Fortis, in \hyperref[chap:fortis]{Chapter 3} of this volume, examines similar aesthetic ideas in the work of Edward Sapir, and their possible connection to Gabelentz' work.} In his view, all shaping of linguistic expression, regardless of how transparently its origin shows through, is formal in nature \citep[see][]{McElvenny2016}. \citet[406--408]{Gabelentz20161891} accepted, however, that in the most primitive stages linguistic forms would have been created spontaneously and freely, and only over time become constrained and fixed through force of collective habit.

Given the dominance of historical-comparative grammar in the disciplinary linguistics of his day, Gabelentz dedicates an entire ``book'', or primary section, of his \emph{Sprachwissenschaft} to this approach to language study. He finds that the principle of gradual fixing of the linguistic system applies also on the phonetic plane, and uses this principle both to critique the supposed exceptionless nature of sound change as promulgated by the Neogrammarians and as a means to explain how sound change can occur at all. ``Fluctuating articulations'' (\emph{schwankende Articulationen}), according to \citet[196]{Gabelentz20161891}, are a very real part of languages, and indeed they are the force driving sound change in the first place. If, as the Neogrammarians argued, sound change proceeded according to inviolable rules then everyone would always speak the same way. For sound change to occur, one speaker has to innovate a new pronunciation and then it has to spread to the rest of the speaker community. \citet[196--197]{Gabelentz20161891} is very clear that he means not only variation in pronunciation between speakers, but also variation in the same speaker over the course of their lives and even from utterance to utterance.

The way in which Gabelentz describes the range of variation that each language allows in fact seems to evince an inchoate concept of the phoneme as an ideal sound which may have multiple realizations:

\begin{quotation}
But languages, even the smallest dialects, distinguish only a certain number of sounds, which are related to individual phonetic phenomena like species to individuals, like circles to points; a language draws the boundaries more broadly or narrowly, but it always tolerates a certain degree of latitude. \citep[35]{Gabelentz20161891}.\footnote{Original: ``Die Sprache aber, und wäre es die kleinste Mundart, unterscheidet nur eine bestimmte Anzahl von Lauten, die sich zu den lautlichen Einzelerscheinungen verhalten wie Arten zu Individuen, wie Kreise zu Punkten; sie zieht die Grenzen weiter oder enger, immer aber duldet sie einen gewissen Spielraum.''}
\end{quotation}
 
The ``degree of latitude'' allowed may vary from language to language, according to \citet[197--198]{Gabelentz20161891}, and this greater or lesser latitude provides the theoretical basis for countenancing the possibility of alternating sounds of a more extreme kind outside the familiar European languages.

A similar recognition of variation within limits is also a feature of Boas' (\citeyear{Boas1911}) account of alternating sounds in the officially codified version of the \emph{Handbook}. Here Boas admits variations in the realization of sounds in languages, but crucially he denies that the range or latitude of variation can vary from language to language: the American languages admit neither more nor less variation in their sounds than any other languages, and certainly no more than European languages. Taking the example of a sound in Pawnee, \citet[17]{Boas1911} insists:

\begin{quotation}
Thus the Pawnee language contains a sound which may be heard more or less distinctly sometimes as an \emph{l}, sometimes an \emph{r}, sometimes as \emph{n}, and again as \emph{d}, which, however, without any doubt, is throughout the same sound, although modified to a certain extent by its position in the word and by surrounding sounds. […] This peculiar sound is, of course, entirely foreign to our phonetic system; but its variations are not greater than those of the English \emph{r} in various combinations, as in \emph{broth}, \emph{mother}, \emph{where}. \citep[17]{Boas1911}
\end{quotation}

Gabelentz' theoretically grounded belief in varying degrees of latitude in pronunciation leads him, in contrast to Boas, to accept and repeat several well-known cases of alternating sounds from the corners of the world: \citet[202--204]{Gabelentz20161891} offers examples from Samoan, Malay languages, Australian languages and of course various American languages. Gabelentz is willing to trust the data on alternating sounds delivered by scholars in the field, insisting that they are fully qualified observers who through extended immersion in the foreign language have had the opportunity to overcome the interference of their native phonology. Indeed, it is because they have become so accustomed to the phonological systems of the languages they record that they have developed the feeling for the languages that allows them to perceive the subtle alternating articulations:

\begin{quotation}
We could raise the following objection: most of our informants were not schooled in the scientific observation of sounds; they judge the foreign sounds according to their native language, and intermediate grades between these sounds seem at one moment to tend to one side and in another moment to another side. We may retort that at least some of these men have lived long enough among the aborigines that their ear has become as accustomed to the foreign language as it was previously to their native language. It is to this, or rather to their multilingual schooling, that they owe precisely this fine ability to hear that allows them to perceive those uncertain, fluctuating articulations. \citep[204--205]{Gabelentz20161891}\footnote{Original: ``Folgenden Einwand könnte man erheben: Die meisten Gewährsmänner waren nicht zu wissenschaftlicher Lautbeobachtung geschult; sie beurtheilten die fremden Laute nach denen ihrer Muttersprache, und Zwischenstufen zwischen diesen schienen ihnen bald nach der einen, bald nach der anderen Seite zu neigen. Darauf ist zu entgegnen, dass mindestens ein Theil jener Männer lange genug unter den Eingeborenen gelebt, um ihr Ohr an die fremde Sprache so zu gewöhnen, wie es vordem an die Muttersprache gewöhnt gewesen. Dieser, oder richtiger ihrer mehrsprachigen Schulung, verdankten sie eben das feinere Gehör, das sie jene unsicheren, schwankenden Articulationen empfinden liess.''}
\end{quotation}

In an inversion of the assignment of expertise effected by Boas, Gabelentz endorses the data and uses them to undermine the theoretical edifice of the Neogrammarians. Rather than refining sound laws to explain away exceptions, \citet[198]{Gabelentz20161891} advocated statistical surveys that would embrace all variants observed, the deviants as well as the well-behaved regular forms. Gabelentz' model for this endeavour was perhaps the statistical analyses undertaken by William Dwight Whitney (1827--1894) of variant forms throughout the history of Sanskrit and in modern English dialects (\citealt{Whitney1874}; \citealt{Whitney189618751878}; \citealt[cf.][vix-xx, xxii-xxiii]{Silverstein1971}).\footnote{\citet{Gabelentz1894t} also later proposed using a statistical approach for the typological study of languages. \citet{McElvenny2018typ} offers an English translation of this text.} Wilhelm Wundt similarly suggested that a statistical approach to the study of sound change may prove more fruitful than the absolutism of the Neogrammarians \citep[see][]{Formigari2018}.

Gabelentz' first steps towards applying a statistical method were taken in an \citeyear{Gabelentz1893} address to the Berlin Academy of Sciences in which he tried to prove a genealogical relationship between the Basque and Berber languages.\footnote{For a comprehensive account of this episode, including Gabelentz' initial address, the subsequent book-length publication \citep{Gabelentz1894bb}, and the reaction of Gabelentz' colleagues, see \citet{HurchPurgay2019}.} As \citet[593--594]{Gabelentz1893} himself noted, the hypothesis that the Basques of southern Europe, whose language could not be aligned with any known family, were in some way related to the ``Hamites'' of North Africa was not a novel idea. That no linguistic proof of this relationship had yet been given, he contended, was due to the inflexibility of the \isi{comparative method} as it was practised at the time. The \isi{comparative method} needed to be ramified to accommodate the radical mutability of linguistic form that had been discovered in regions beyond the familiar Indo-European context, as in Indo-Chinese and Melanesian sources:

\begin{quotation}
The belief in the constancy of the outer and \isi{inner linguistic form} is among the achievements to which our science clings most tenaciously, and the facts that could shake this belief are for their part newly acquired and poorly known, since they are in the territory of Indo-Chinese and Melanesian. \citep[594]{Gabelentz1893}\footnote{Original: ``Der Glaube an die Beständigkeit der äusseren und inneren Sprachform gehört zu den Errungschaften, an denen unsere Wissenschaft am zähesten festhält, und die Thatsachen, die ihn erschüttern könnten, sind ihrerseits neuer Erwerb und wenig bekannt, da sie auf indo-chinesischem und melanesischem Gebiete liegen.''}
\end{quotation}

Looking across Basque dialects, \citet{Gabelentz1893} postulated extremely irregular sound correspondences between apparently cognate words, leading him to the conclusion that ``they offer a picture of phonetic wildness which, as far as I know, must be one of a kind in the world of languages'' (\emph{sie geben ein Bild lautlicher Verwilderung, das meines Wissens in der Sprachenwelt kaum Seinesgleichen hat}; \citealt[596]{Gabelentz1893}). He found a similar situation in the Berber languages. On this basis, \citet[604]{Gabelentz1893} assumed the existence of a ``prehistoric period of the most uncertain articulation'' (\emph{vorgeschichtlichen Periode der unsichersten Articulation}) in these languages, ``where the phonetic images appeared before the soul only in vague outlines, as if they were drawn with a mop or paint-roller'' (\emph{wo die Lautbilder der Seele nur in vagen Umrissen vorgeschwebt haben, als wären sie mit dem Wischer gezeichnet oder mit dem Vertreiberpinsel gemalt}). To bring order into this chaos, Gabelentz employed his statistical method, tabulating the frequencies of putative correspondences across the Basque dialects, the Berber languages and between these two groups.

The extraordinarily large latitude in pronunciation of the kind attributed to Basque and Berber is, \citet[606]{Gabelentz1893} argued, characteristic of languages ``at a lower level of culture'' (\emph{auf niederer Culturstufe}). At this cultural level, articulated forms are only rejected when they cannot be understood. This lack of constraint on variation leaves linguistic forms subject to the temperamental and corporeal contingencies of the moment, as in Steinthal's conception of the first stage of \isi{language evolution}. Distant analogues of such cases can even be observed in Indo-European languages, claimed Gabelentz, offering the example of an uneducated Saxon from Germany (the same example with a more moderate moral occurs also in \citealt[398]{Gabelentz20161891}):

\begin{quotation}
In this way a strange thing can happen, that a very indefinite sound image appears before the soul, and yet the mouth produces a very clearly articulated sound, although not always the same one, but rather at one moment this one and then at another that one, depending on chance and mood. […] I can offer an example of an at least distant analogue of this from our own languages. The Saxon, who does not distinguish between \emph{d} and \emph{t}, between \emph{i} and \emph{ü}, \emph{e} and \emph{ö}, \emph{ei} and \emph{eu}, \emph{äu}, can in the heat of the moment pronounce the \emph{d} as \emph{t} and -- when he is talking about deep, dark, terrible things -- turn all \emph{i}, \emph{e} and \emph{ei} into \emph{ü}, \emph{ö}, \emph{eu} in a kind of onomatopoeia. \citep[606--607]{Gabelentz1893}\footnote{Original: ``So kann das Seltsame geschehen, dass der Seele ein sehr unbestimmtes Lautbild vorschwebt, und doch der Mund ein sehr scharfes hervorbringt, aber nicht immer dasselbe, sondern bald dieses bald jenes, je nach Zufall und Stimmung. […] Aus unserem Sprachkreise wüsste ich wenigstens entfernt Analoges anzuführen. Dem Obersachsen, der zwischen \emph{d} und \emph{t}, zwischen \emph{i} und \emph{ü}, \emph{e} und \emph{ö}, \emph{ei} und \emph{eu}, \emph{äu} nicht unterscheidet, kann es geschehen, dass er im Affecte jedes \emph{d} wie \emph{t} ausspricht, und dass er, wo es sich um tiefe, dunkele, grausige Dinge handelt, alle \emph{i}, \emph{e} und \emph{ei} lautmalend in \emph{ü}, \emph{ö}, \emph{eu} verwandelt.''}
\end{quotation}

Needless to say, Gabelentz' attempted reform of the \isi{comparative method} did not gain a foothold. The exceptionless dismissal of Gabelentz' approach may not, however, have been so much due to his underlying premises as to his cavalier treatment of his sources. Even among those who could be expected to sympathize with Gabelentz' proposal, the criticism was widespread that he had not properly curated or analysed the Basque and Berber data, which led him to obvious errors in presentation and interpretation \citep[cf.][]{HurchPurgay2019}. \citet{Brinton1894}, for one, in his brief review of the \citeyear{Gabelentz1894bb} expanded book version of Gabelentz' Basque and Berber studies, did not criticize Gabelentz' underlying views on variation, but did note that he had not properly distinguished between cognates and loan words in his analyses.

Hugo Schuchardt (1842--1927) -- one of the most prominent contemporary opponents of the Neogrammarians, as he acknowledged himself \citep[see, e.g., ][]{Schuchardt1928} -- was similarly unimpressed by Gabelentz' methodological laxness, despite being sympathetic to the motivating idea of the radical mutability of linguistic forms. In a review of \citet{Gabelentz1894bb}, he questioned the wisdom of taking such an adventurous course in comparing these languages when the more conventional and uncontroversial methods had yet to be tried properly:

\largerpage[1]\begin{quotation}
The Kabyle and Tuareg words that the author [Gabelentz] compares to the Basque words differ from these greatly for the most part. He does indeed attempt to explain this on the basis of muddled and washed out phonetic confusion. However, even if I do not dispute this possibility in general, it still seems to me that we should for the time being -- that is, as long as further and more careful examinations of Basque phonetic history are not available -- not seek refuge in this ``last resort''. \citep[334]{Schuchardt1893}\footnote{Original: ``Die kabylischen und tuaregischen Wörter, die der Verf. zu baskischen Wörtern stellt, weichen von diesen zum grossen Theil sehr stark ab. Zwar sucht er das aus einer verworrenen und verwaschenen Lautirung zu erklären: aber wenn ich auch im Allgemeinen die Möglichkeit einer solchen nicht bestreite, so dünkt mich doch, wir sollten vorderhand, d.~h. so lange nicht mehr und sorgfältigere Untersuchungen über die baskische Lautgeschichte vorhanden sind, hier nicht zu dieser `ultima ratio' unsere Zuflucht nehmen.''}
\end{quotation}

Gabelentz' freewheeling approach, commented \citet[334]{Schuchardt1893}, offers no credible way to navigate language history. It could just as easily be used to link Basque to the languages of the Caucasus or the Ural as to those of North Africa. Although there were linguists dissatisfied with the rigid system-building of the Neogrammarians and prepared to face the messiness of the raw data, Gabelentz' scheme did not present a viable alternative for them.

\section{Conclusion}
\label{sec:mcelvenny:conc}

In the last decades of the nineteenth century, the phenomenon of alternating sounds was instrumentalized in different ways by scholars hoping to advance their various academic and disciplinary agendas. In America, Boas denied the reality of the phenomenon as part of a project to assert the scientific superiority of the anthropological school he was busily building up. In Germany, Gabelentz moved in the opposite direction, embracing the phenomenon as a means to undermine the hegemony of Neogrammarian linguistics. The positions of both Boas and Gabelentz -- and indeed also their rivals -- were informed in no small way by the mid-nineteenth-century writings of Steinthal, who developed a unified theory of the psychological basis and evolution of language with a strongly empirical accent.

Although both Boas and Gabelentz indulge in exaggeration and caricature in their critiques, and exhibit obvious faults in elaborating their own positions, their views have had very different fates in the received histories of linguistics and anthropology. External factors no doubt play a role here: Boas achieved institutional dominance and is feted as the founding father of modern American anthropology, while Gabelentz died early and disappeared into relative historical obscurity. 

The different fates of their views on alternating sounds are perhaps also indicative of the temperament of linguistics and anthropology as disciplines. Despite his apparent hostility to later conceptions of the phoneme, Boas' attack on the notion of alternating sounds is celebrated for expanding the formal franchise, making all languages equal subjects under the laws of linguistics. Gabelentz' efforts to problematize the \isi{comparative method}, by contrast, could find no supporters: his dismembering of current historical linguistics offered no practical alternative. Boas is more welcome than Gabelentz in fields that place a premium on technical progress, conceived positivistically as the ability to capture and catalogue phenomena within a universalizing system. This case study offers informative parallels to the ``resistant embrace'' of structuralism in France that John Joseph (\hyperref[chap:joseph]{Chapter 6}, this volume) sketches and the ``\isi{unique form hypothesis}'' that Nick Riemer (\hyperref[chap:riemer]{Chapter 9}, this volume) imputes to present-day linguistics.

\section*{Acknowledgements}

Clemens Knobloch, 
Chloé Laplantine, 
H. Walter Schmitz 
and 
Manfred Ringma\-cher
all provided useful feedback on earlier versions of this chapter which led to its improvement.

\sloppy
\printbibliography[heading=subbibliography,notkeyword=this]

\end{document}
