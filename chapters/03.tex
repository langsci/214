\documentclass[output=paper]{langscibook} 
\ChapterDOI{10.5281/zenodo.2654353}

\author{Jean-Michel Fortis\affiliation{\textsc{cnrs}, Université Paris Diderot}}

\title{On Sapir's notion of form/pattern and its aesthetic background}
\label{chap:fortis}
 

\epigram{``I find that what I most care for is beauty of form, whether in substance or, perhaps even more keenly, in spirit. A perfect style, a well-balanced system of philosophy, a perfect bit of music, the beauty of mathematical relations — these are some of the things that, in the sphere of the immaterial, have most deeply stirred me.''\newline
Sapir, letter to Lowie, 29~September~1916 (cited in \citealt[79]{Silverstein1986})}

\abstract{On Sapir's view, units of cultural behaviour (such as linguistic units) can only be identified through the relations they maintain to other elements of the same kind. This set of interrelations is what Sapir calls a ``pattern'', or refers to simply as ``form''. The chapter begins by examining Sapir's notion of pattern in his analysis of phonological systems. It is shown that, to a certain extent, Sapir conflated the notion of pattern with that of \emph{Gestalt}, yet his own conception was idiosyncratic insofar as it placed much emphasis on the purely formal potency of patterns, understood as aesthetic configurations existing for form's sake and independent from functional motivations.

The second part of the chapter is devoted to Sapir's description of how patterns are formed and grasped. Complex interrelations are not laid bare in ordinary conscious thinking; they can only be accessed through an intuition that Sapir characterizes as a ``form-feeling''. Form-feeling, as Sapir himself tells us, takes its origin from art theory. It is argued that the source of this notion is to be found in German-speaking art theory, specifically the notion of \emph{Formgefühl}. In the course of the discussion, the hypothesis is set forth that Sapir's ``form drive'', which underlies the elaboration of patterns for form's sake, might also have its source in German thought, notably in Humboldt and Schiller.}

\begin{document}
\maketitle

\section{Introduction\protect\footnote{Parts of this chapter have already appeared as a post of the multi-author blog \emph{History and Philosophy of the Language Sciences} \citep{Fortis2014} and in an extended French version \citep{Fortis2015}. Many thanks to James McElvenny for his suggestions and corrections of this English version. }}
\label{sec:fotis:intro}

The vast range of scholarly interests which Sapir nurtured during his life far exceeds what would fall under our current conception of linguistic matters. In particular, psychology, especially Gestaltist, psychoanalysis, as especially represented by Carl Jung (1875--1961), music and aesthetics were to him concerns of prime importance. As we shall see, his interests are reflected in the general conceptions he entertained about culture and language, and more precisely about linguistic form, or, in terms also used in his writings, linguistic pattern or configuration.

In what follows, it will be shown that the aesthetic leitmotiv \rephrase{which runs}{running } through many of Sapir's writings is essential to understand what is idiosyncratic in his notion of form or pattern. The aesthetic viewpoint, as will be argued, is fundamental for understanding how Sapir conceived of an individual's relation to cultural and linguistic patterns; it also helps us see in what ways Sapir's ideas about the constitution of patterns and their diachrony deviated from anything we are familiar with. Further, the theoretical connections of Sapirian form with Gestaltist ideas and psychoanalysis might be best appreciated by, again, following the aesthetic thread.

What about the historical roots or inspirations of Sapir? Their very heterogeneity and the fact that they were not always disclosed by him appear to have produced a gap, or an indecision, in Sapirian studies. Were we to suggest probable influences beyond the well-attested ones, this gap could be at least partially filled and our comprehension of Sapir deepened. Any attempt in this direction is certainly worthwhile and we will do our best to offer proposals on this historical context throughout the discussion and particularly in the last part of this chapter.

\section{An example of pattern: the phonological system}
\label{sec:fortis:egpattern}

Sapir's conception of a phonological system is a good entry point for his notion of pattern. As early as \emph{Language} (\citeyear{Sapir1921}), phonemes are said to be ``points'' of an ``underlying phonetic pattern''. A potential simplifying misconception should first be dispelled: we may be tempted to read into Sapir's analysis of a phonological pattern the idea that phonemes are merely identified by their distinctive features, in effect, then, by the contrastive relations they hold to other phonemes of the language. Sapir's conception is more complex. In a phonological pattern, the relative distance of elements is also determined by the degree to which they share common contexts of occurrence and partake of the same functional/semantic role. See, for example, the fact that a Nootka speaker conflates /p'/ and /'m/ into a single phonological structure /C'/ \citep[55--57]{Sapir1933}. There can be no other reason, as Sapir explains, than the fact that the occurrences of /p'/ and /'m/ are sufficiently similar to warrant the assimilation of their phonological structure. This assimilation, in other words, manifests the fact that the occurrences of /p'/ exert an attraction on /'m/ which results in a levelling out of their phonological structure. The relative proximity of elements in the system is also determined by functional and semantic factors. For instance, in English /f/ contrasts with /v/, as /p/ contrasts with /b/, but /f/ is closer to /v/ than /p/ to /b/ for the reason that /f/ and /v/ belong to common paradigms, such as \emph{wife}/\emph{wives}. In turn, /f/ and /v/ form with /θ/-/δ/ and /s/-/z/ what we may call a subsystem or subpattern, in view of parallel voicing contrasts like \emph{sheath}/\emph{sheathe} and \emph{mouse}/\emph{mouses} \citep[48]{Sapir1933}. The very existence of idiosyncratic patterns, and the fact that the relational feelings of speakers have an effect on phonetic change, do not make it permissible ``to look for universally valid sound changes under like articulatory conditions'' \citep[48]{Sapir1933}. This is not to reject the search for regularities, nor is it in contradiction of the neogrammarian assumption of the inviolability of sound laws, rather it is an implicit restriction of their validity to a family of languages. In fact, Sapir occasionally points out the perpetuation of patterns or subpatterns within a language or genetically related languages.

On several points, it should be noted, Sapir's view is in line with Hermann Paul (1846--1921) in his \emph{Prinzipien} (1880--1920): linguistic elements form dynamic groups which absorb or repulse elements that are, respectively, similar or dissimilar in their form, function and semantics. The series \emph{wife}/\emph{wives} = \emph{sheath}/\emph{sheathe} = \emph{mouse}/\emph{mouses} would constitute, in Paul's parlance, a \emph{stoffliche Proportionengleichung}, a ``material equation'' \citep[86]{Paul1880}. In the German context, such views on representations acting as groups can be traced back to Johann Friedrich Herbart (1776--1841), whose psychology of the unconscious appears to have furnished theoretical tools to many thinkers, not only to Paul. By contrast with Paul, however, Sapir isolates phonemes from morphemes, an abstraction which Paul might have found implausible from a psychological point of view. In addition, Sapir's aesthetic conception of patterns strongly colours his interpretation of Herbartian groups, as we shall see later.

Again in conformity with Paul and much of the linguistic literature of the time, for Sapir the organization of groups is \emph{unconscious}. Latent factors can be brought to light in a variety of ways: through the conflation of phonemic structures, as in the Nootka example; through the filling in of phonemes which reflect co-occurrences latent in the speaker’s knowledge \citep[52--53]{Sapir1933}; or through what Herbartian psychology and Boas (1858--1942) had described as apperceptive phenomena, for example, in English speakers, the illusory addition of a weak consonant in syllables ending with a short vowel \citep[58--59]{Sapir1933}.\footnote{``A new sensation is apperceived by means of similar sensations that form part of our knowledge'' \citep[50]{Boas1889}. Apperception is one of those notions which belong to the stock-in-trade of Herbartian psychology. The concept of apperception was presumably passed on to Boas through Steinthal. Its application to the issue of ``alternating sounds'' in Native American languages is discussed by McElvenny in \hyperref[chap:mcelvenny]{Chapter 2} of this volume.} More generally, unconscious patterning gives rise to what we would characterize as ``categorial perception''; that is, the perception of forms that is consonant with their position in a pattern and abstracts away from physical features. Such perception, as \citet[46]{Sapir1933} points out, is a general trait of human cognition; indeed, the distinction between physical features and their position in a cultural or linguistic pattern, and the psychological primacy of the latter, is a point repeatedly emphasized in Sapir's texts. Lastly, and this brings us back to aesthetics, in Sapir's (\citeyear{Sapir1925}) ``Sound patterns in language'', the ability to access and use this sytem of positions, in other words \emph{speech}, is characterized as an ``art''. This rather surprising characterization, undeveloped and allusive as it is in the text, will hopefully be made more understandable when Sapir's notion of form (or pattern) and the aesthetic motif are more closely examined.

\section{Patterns as Gestalten}
\label{sec:fortis:patternsgestalten}

Most remarkable is the very salient fact that Sapir does not speak of a network of groups but of patterns of elements. This shift of emphasis only underlines the action of unconscious patterns on forms surfacing as conscious elements, for only elements are conscious, not groups. Speaking of ``pattern'' or, at times, of ``configuration'' also reveals the connection Sapir established between unconscious organization and configurations designated as \emph{Gestalten} in the psychology of the same name. 

We do not know for certain when Sapir became acquainted with \emph{Gestalt} psychology. Several testimonies (including a letter by Sapir himself) have revealed Sapir's admiration for Kurt Koffka (1886--1941) and more specifically his book \emph{The Growth of the Mind} (\citeyear{Koffka1924}), which he appears to have read in 1924.\footnote{Sapir's positive judgment of the book is conveyed with particular elation in a letter to Benedict (cited in \citealt[121]{Sapir2002}). David Sapir's testimony is cited in \citet[478]{Cowan1986}. These testimonies alone suffice to show that \citet{Murray1981}, who denies any influence of Gestaltist ideas on Sapir, has seriously misjudged the influence of \emph{Gestalttheorie} on Sapir's thinking.} We know the book lay at hand's reach in the family house, and that it was recommended reading for Sapir's seminars on the impact of culture on personality and on the psychology of culture at Yale (years 1933--34 and 1935--36; \citealt[677]{Sapir19991933}). It is also a matter of historical record that Sapir met Koffka personally in a symposium on the unconscious in 1927. 

There is certainly a kinship between the Gestaltist rejection of elementist psychological accounts and its holistic view of perception and behaviour and, on the other hand, Sapir's idea that all units of culturally determined behaviour are pyschologically active only as ``points in a pattern'', not as bundles of physical features. The ``psychological reality of phonemes'' (the title of Sapir's famous paper) has its match in the Gestaltist affirmation that ``sensations'' do not have phenomenal reality. The contexts of the two theories were vastly different, but this common point shoud be highlighted.\footnote{Sapir's view was also an argument in favour of a minimalist position in the debate on the phonetic notation most suited to Amerindian languages: should one complexify or simplify it? See \citet[285]{Darnell1990}.}

Perhaps most congenial to Sapir was Koffka's definition of a configuration (\emph{Ge\-stalt}) which, although it occurred in the context of a discussion of figure/ground organization, was of the widest generality. A configuration (or \emph{Gestalt}),\footnote{Mead had lent Koffka's book to Sapir in its English version, where ``configuration'' translates \emph{Gestalt}/\emph{Struktur} \citep[185]{Darnell1990}. Koffka explains that the translation as ``structure'' was not retained for fear that in an American context it might be interpreted in connection with structural psychology (i.e. the Wundtian-style analysis of mental states promoted by Titchener in the United States).} said \citet[146]{Koffka1924}, was definable as a ``co-existence of phenomena, in which each member possesses its peculiarity only by virtue of, and in connection with, all the others''. In a similar vein, Sapir explained that a linguistic sound 

\begin{quotation}
is not only characterized by a distinctive and slightly variable articulation and a corresponding acoustic image, but also — \emph{and this is crucial} — by a psychological aloofness from all the other members of the system. […] A sound that is not unconsciously felt as ``placed'' with reference to other sounds is no more a true element of speech than a lifting of the foot is a dance step unless it can be ``placed'' with reference to other movements that help to define the ``dance''. (\citealt[35]{Sapir1925}; Sapir's emphasis)
\end{quotation}

Sapir's allusion to an organized sequence of actions (dancing), in this passage and elsewhere \citep[104--105]{Sapir2002}, might point to the fact that patterned behaviour was appreciated as a major step towards an enlargement of the notion of configuration. This enlargement was especially noticeable in the way Koffka described Köhler's experiments with apes: learning how to solve a task is, says Koffka, the establishment of a new configuration in which an element of the field -- for example, a stick -- comes to find a role in an action sequence with a beginning and an end; that is, a configuration which has the property of closure (a term significantly reused by Sapir in \emph{The Psychology of Culture}, \citeyear{Sapir2002}: 104). Further, the apprehension of something as a tool is configurational insofar as a chimpanzee, when putting together a thicker and a thinner stick, is sensitive to their relative length, not to their absolute size; in another situation, the thinner stick may thus play the role of the thicker one.

The vicarious character of units of behaviour and their identification through the configuration of which they partake is precisely the point being made in the initial example of Sapir's (\citeyear{Sapir1925}) ``Sound patterns of language'': the expiration \emph{wh} that blows a candle gets entirely reconfigured when it becomes a linguistic gesture. 

An essential property of formal patterns is their transposability. By ``transposable'' is meant here the capacity for a system of relations to remain unaltered under a change in physical implementation. Language furnishes several instances of transposability. Thus, the formal patterning of language is said to underlie the possibility of ``linguistic transfers''; that is, the possibility of resorting to various (de)coding techniques, such as writing, lip-reading or gestural systems; or again, the system of initial consonants in English is a historical transfer from the Indo-European one \citep[200]{Sapir1921}. The latter kind of transposition, which involves a phonological system, will be further elaborated through an artificial example in ``Sound patterns of language''.

Now, transposability, in \emph{Gestalt} thought, furnished evidence for the existence of qualities not reducible to sums of sensations. Indeed, from the very beginning of \emph{Gestalt} psychology, in the seminal work of Christian von Ehrenfels (1859--1932), transposability was played upon as a favourite theme: ``proof of the existence of Gestalt qualities'', said \citet[90]{Ehrenfels1988}, ``is provided, at least in the sphere of visual and aural presentations, by the similarity-relations […] which obtain between melodies and figures having different tonal or positional foundations''. Note that the transposability of melodies as ``systems of relations'' was cited as a universal of human musical cultures by Carl Stumpf (1848--1936) in \emph{Die Anfänge der Musik} (\citeyear{Stumpf1911}), a book Sapir reviewed at an early stage of his career. Gestaltist ideas thus came to Sapir at least in part through the mediation of aesthetic theory, and much earlier than 1924. There remains, however, a point on which Sapir is at variance with the Gestaltists: whereas a \emph{Gestalt} quality is phenomenally more directly accessible than its elements, the structure of a pattern is, for Sapir, unconscious. However, its units are grasped in a way which, because it does not lay its structure bare, Sapir most often describes as a ``feeling''. We shall now turn our attention to this feeling for form.

\section{The form-feeling}
\label{sec:fortis:form-feeling}

In many places, Sapir refers to the grasping of patterns of all kinds, be they phonological, morphological and syntactic, or behavioural and social, as a ``feeling'' or, less frequently, as an ``intuition'' of the same order. This view, as far as I know, makes its first appearance in \emph{Language} (\citeyear{Sapir1921}), where it is applied to linguistic patterning. For example, we read that ``both the phonetic and conceptual structures show the instinctive feeling of language for form'' \citep[56]{Sapir1921} or that every language has a definite feeling for its inner phonetic system and ``also a definite feeling for patterning on the level of grammatical formation'' \citep[61]{Sapir1921}. The notion of a feeling for form/pattern recurs in different guises which we may assume to carry the same meaning: ``relational feeling'', ``form intuition'', ``feeling for form/relations/patterning/classification into forms'', ``to feel a pattern/form'' etc. These expressions are used in various contexts: quite generally, as above, to refer to the phonological/morphosyntactic apparatus of a language, as in discussing the unconscious direction imparted to thinking by the forms a language has laid down \citep[153]{Sapir1924}; more specifically, while speaking of vocalic alternations in English (\emph{goose-geese}, \emph{sing-sang-sung}; \citealt[60--61]{Sapir1921}), active constructions \citep[84--85, 111]{Sapir1921}, of which the speaker is said to feel the SVO structure, possessive pronouns, the animate/inanimate distinction \citep[156]{Sapir1921}, case-marking on the English interrogative pronoun \citep[159]{Sapir1921}, the semantic relation between \emph{boy} and \emph{man} \citep[61]{Sapir1929b}, the meaning of \emph{est-ce que} in French and of verbal stems in Athabaskan languages \citep[147]{Sapir1923}, causative forms \citep[154]{Sapir1924}, which are an unconscious, unreflective mode of the mental representation of the concept of causation;  French reflexive verbs \citep[116]{Sapir1931}, which, to French speakers, induce a ``formal feeling'', a sense of belonging together, although from an external perspective semantic homogeneity is hard to find. Of all these texts, the notion of ``form-feeling'' is probably most frequently referred to in ``Sound patterns'' (\citeyear{Sapir1925}).

A passage from ``The unconscious patterning of behavior in society'' (\citeyear{Sapir1927b}) provides a good illustration of the issues intertwined with the notion of form-feeling.

\begin{quotation}
To most of us who speak English the tangible expression of the plural idea in the noun seems to be a self-evident necessity. Careful observation of English usage, however, leads to the conviction that this self-evident necessity of expression is more of an illusion than a reality. If the plural were to be understood functionally alone, we should find it difficult to explain why we use plural forms with numerals and other words that in themselves imply plurality. ``Five man'' or ``several house'' would be just as adequate as ``five men'' or ``several houses.'' Clearly, what has happened is that English, like all of the other Indo-European languages, has developed a \emph{feeling} for the classification of all expressions which have a nominal form into singulars and plurals. So much is this the case that in the early period of the history of our linguistic family even the adjective, which is nominal in form, is unusable except in conjunction with the category of number. (\citealt[550]{Sapir1927b}; my emphasis)
\end{quotation}

The example brings home the point that a structural feature is, as it were, ``exercised'' in actual speech in a way that is not of the order of conscious knowledge. Such a feature gives form to experience and may perpetuate itself by the sheer force of the unconscious pattern which imposes itself on the speaker. Their thoughts being channelled in these formal grooves, speakers may resist the elimination of what, in the eyes of cool reason, would appear to be non-functional or a superfluous luxury.

Note too that the form-feeling has implications for the way diachrony should be conceived. In the passage just cited, and in other places, Sapir seems to be engaged in an implicit dialogue with Otto Jespersen (1860--1943), who had famously argued that languages evolve toward greater economy and analyticity (\citeyear{Jespersen1894}, \citeyear{Jespersen1965}; see e.g. \citeyear{Jespersen1965}: 207ff for the example of plurality). Against Jespersen, yet not in complete disagreement with him, Sapir apparently claims that languages may not evolve toward the complete elimination of superfluities and toward absolute or near absolute analyticity, for speakers' unconscious attachment to formal patterns carries with it an inertia which resists this evolution. We shall return to the issue of diachrony shortly.

A clue to the understanding of Sapir's ``form-feeling'' may be found in the following excerpt, which clearly points to the aesthetic source of the notion:

\begin{quotation}
Probably most linguists are convinced that the language-learning process, particularly the acquisition of a feeling for the formal set of the language, is very largely unconscious and involves mechanisms that are quite distinct in character from either sensation or reflection. There is doubtless something deeper about our feeling for form than even the majority of \emph{art theorists} have divined, and it is not unreasonable to suppose that, as psychological analysis becomes more refined, one of the greatest values of linguistic study will be in the unexpected light it may throw on the psychology of intuition, this ``intuition'' being perhaps nothing more nor less than the ``feeling'' for relations. (\citealt[156]{Sapir1924}; my emphasis)
\end{quotation}

There are two possible ways of interpreting the reference to aesthetics: our feeling for linguistic form can be conceptualized in analogy with its counterpart in aesthetic theory; or both feelings reflect a common ability, the intuitive grasp of complex patterns. From what we have said so far, from the way Sapir conflates phonological intuition with art \citep[34]{Sapir1925}, or seems to equate \emph{Gestalten} with aesthetic forms \citep[145--150]{Sapir2002}, we may gather that aesthetic intuition was for him a general ability exceeding the bounds of the perception of artistic forms as such (see too, in this respect, the epigraph to this chapter). Such an interpretation would allow us to draw a parallel between this formal linguistic play which is supposed to reflect an innate striving for formal elaboration and, on the other hand, the Boasian idea that artistic creation begins with the purely formal, unrepresentative exercising of technical skills \citep{Boas1927}. In the realm of aesthetic thought, Sapir would have as counterparts those theorists granting pride of place to ornamentation, decorative arts; that is, to formalist considerations. In the same way, linguistic change becomes comparable to stylistic change, at that time an all-important question of aesthetic theory. It is now time to see the relation of linguistic change to the aesthetic perspective.

\section{Diachrony}
\label{sec:fortis:diachrony}

We may wonder if Sapir's concepts of pattern and form-feeling have important consequences for his descriptive linguistic work. They certainly do in phonology. It is suggested here that his view of diachronic processes might furnish another illustration and demonstrate again the relevance of the aesthetic perspective.

In \emph{Language}, diachronic change is described as a ``drift'', a notion which \citet[155]{Sapir1921} defines as follows: ``The drift of a language is constituted by the unconscious selection on the part of its speakers of those individual variations that are cumulative in some special direction''.\footnote{On the meanings of ``drift'', and its reception after Sapir, see \citet{Malkiel1981}. Malkiel suggests that drift may have its source in the continental drift (\emph{Verschiebung} in German) of Wegener. The idea seems outlandish to me. Hermann Paul, like other authors, speaks of drift (also \emph{Verschiebung}) when dealing with those slight variations which cause constant linguistic change. The definition of ``drift'' just cited would be perfectly in line with Paul.} This view of change can be made more palpable through an illustration, Sapir's account of the progressive disappearance of \emph{whom} in favour of \emph{who}. According to Sapir, four causes have contributed to the decline of \emph{whom}. They are summarized in \tableref{tab:fortis:causes}.

\begin{table}
\begin{tabularx}{\textwidth}{lQQ}
\textbf{Cause} & \textbf{Phenomenon} & \textbf{Consequence} \\
\midrule
1 & The forms marking the non-subject (``objective forms'') are \emph{me}/\emph{him}/\emph{her}/\emph{us}/\emph{them}/\emph{whom}. In this group \emph{whom} is isolated.\newline The functional class of \emph{whom} comprises \emph{which}/\emph{what}/\emph{that} but these are not inflected.
& The isolation of \emph{whom} causes its weakening. \\
\midrule
2 & Interrogative words like \emph{where}/\emph{when}/\emph{how}, are invariable, except \emph{who}/\emph{whom} & The isolation of \emph{whom} causes its weakening. \\
\midrule
3 & Objective forms are strongly associated with the post-verbal position (cf. \emph{he told him, it's me}), while interrogative ones are strongly associated with pre-verbal positions. & \emph{whom} belongs to two groups whose members are associated with distinct positions. \emph{Who} is not associated with distinct positions and is thereby favoured over \emph{whom}. \\
\midrule
4 & \emph{whom} is followed by a slight hesitation in \emph{Whom did you see?} & \emph{whom} is often ``clumsy'', from a rhythmical point of view, which weakens it.
\end{tabularx}
\caption{Sapir's causes for the decline of \emph{whom}}
\label{tab:fortis:causes}
\end{table}

Points 1 to 3 in \tableref{tab:fortis:causes} are faithful to Hermann Paul's style of explanation: elements which formally or functionally deviate from a group (\emph{Isolierung}) are weakened, except if they are very frequent. However, Sapir's description has its own peculiarities. The frequency factor has disappeared and Sapir has his own way of accounting for the cause of isolation. For example, on the isolation of \emph{whom} in situation 1, he suggests that ``there is something unesthetic about the word. It suggests a form pattern which is not filled out by its fellows'' \citep[158]{Sapir1921}. He is a little more affirmative in case 2, when he adds: ``it is safe to infer that there is a rather strong \emph{feeling} in English that the interrogative pronoun or adverb, typically an emphatic element in the sentence, should be invariable'' (\citealt[159]{Sapir1921}; my emphasis). Apparently, a purely ``mechanical'' account of the formation and dissolution of groups of the kind advocated by Paul is not deemed sufficient. The form-feeling, with its aesthetic connotation, had to come into play.

As hinted at \rephrase{in the previous section}{above}, the aesthetic perspective on language made it possible to envisage a comparison of linguistic change and stylistic change. The similitude is explicitly endorsed in \emph{The Psychology of Culture}:

\begin{quotation}
Practically all aesthetic patterns run through such a gamut: a rise from humble beginnings, an authoritative pinnacle, a prestige hangover — then down! The progress of an aesthetic cycle, then, means that there is aesthetic development within an aesthetic idea. […] Even language forms have something like a cyclical development. Although the language's development is continuous, it is possible to define a certain set of linguistic forms — or point to a certain stage of development of a form — as classical. The classical stage would have a perfectly consistent and tightly-wrought use of forms. Now people participating in an aesthetic cycle are not conscious of it. \citep[132--133]{Sapir2002}\footnote{It is difficult to find any originality in this cyclical view of history. Winckelmann is famous for having defended it in aesthetics.} 
\end{quotation}

Sapir then goes to explain that English has embarked on an evolution toward analyticity but, unlike Chinese, has not yet completed the cycle.

As already noted, the formal efficacy of entrenched patterns explains Sapir's qualifications on Jespersen's idea of a progress toward analytic forms.\footnote{McElvenny (\citeyear{McElvenny2013}, \citeyear{McElvenny2017b}) shows how Jespersen's views relate to the debate on the form of international languages.} English, says Sapir, still mixes up concrete and relational concepts in some limited domains, hence is not fully analytic. For example, the animate/inanimate distinction correlates with distinctive markings, since \emph{I}/\emph{me} and the possessive \emph{'s} are associated with forms denoting animate entities. Through this convergence, formal configurations reinforce each other, with the consequence that ``however the language strives for a more and more analytic form, it is by no means manifesting a drift toward the expression of `pure' relational concepts in the Indo-Chinese manner'' \citep[168]{Sapir1921}. In other words, if we apply to this case the same reasoning as for \emph{whom}, linguistic change is at least partly determined by an aesthetic feeling responding to the (dis)harmony of groups. This view leads to the rejection of purely ``mechanical'' (Paul) and teleological (Jespersen) accounts.

\section{Form, function and formal play}
\label{sec:fortis:formalplay}

The potency of a pattern is not determined by the function it might fulfil; we have seen that formal patterns have their own efficacy. Reciprocally, function may counteract a well-established pattern. An example of such a counteraction in the non-linguistic realm is given in ``Anthropology and sociology'' \citep{Sapir1927a}. In many Indian tribes, Sapir observes, there is an entrenched social pattern according to which prestigious positions are a matter of inheritable privilege. This pattern may even extend to positions which should require special individual capacities, and thus may be transferred to domains in which it is clearly non-functional. However, some tribes resist this transfer, because ``the psychic peculiarity which leads certain men and women (`medicine-men' and `medicine-women') to become shamans is so individual that shamanism shows nearly everywhere a marked tendency to resist grooving in the social patterns of the tribe''. In the present case, functionality (the exigencies of the craft) supersedes a dominant social pattern (the prevalence of inheritable privilege).\footnote{Sapir's notion of cultural/social pattern is in line with Boasian relativism, and its opposition to cross-cultural descriptive schemes appealing to race, evolution or environmental factors. Within diffusionist Boasian anthropology, some emphasized that a proper understanding of the diffusion and assimilation of cultural traits involved moving to the pattern level: substantively identical cultural traits are functionally different if placed within different patterns \citep{Wissler1917}. A radical view holds that substantive traits are of little importance for characterizing some cultural patterns. Totemism, for example, is not to be defined by a substantive trait nor analysed as having originated from any particular trait (be it a guardian spirit, exogamy, taboo, the use of totemic names etc.). Rather, it is a classificatory social pattern, whose origin matters little; what matters is the totemic pattern spreading over a group \citep{Goldenweiser1912}. The analogy with the purely structural Sapirian view of a phonological pattern is obvious.} However, it is not clear that any such counteraction of function can be observed in the linguistic realm. Given what Sapir says about the greater insulation of language from conscious rationalization, it would be coherent to think that a counteraction of function is perhaps only found in non-linguistic domains: language ``forms a far more compact and inherently unified conceptual and formal complex than the totality of culture. This is due primarily to the fact that its function is far more limited in nature, to some extent also to the fact that the disturbing force of rationalization that constantly shapes and distorts culture anew is largely absent in language'' \citep[432--433]{Sapir1916}. 

This ``largely'' unilateral autonomization of form in the field of language would seem to imply that the aesthetic form-feeling plays a greater role in linguistic matters than in any other field. The action of this form-feeling would also be more coercive. In several texts, Sapir connects the potency of patterns with their being unconscious, saying for example that ``we act all the more securely for our unawareness of the patterns that control us'' \citep[549]{Sapir1927a}.\footnote{This conception, as noted by \citet{Joseph2002}, gives the linguist an important role in weakening the grip of linguistic patterns.} In this respect, language has a special status since, explains \citet[100]{Sapir1912}, ``linguistic features are necessarily less capable of rising into consciousness of speakers than traits of culture''. Though less radical, such an affirmation is in agreement with Boas' (\citeyear{Boas1911}: 67) claim that linguistic classifications, of all ethnological phenomena, are unique in being inaccessible to consciousness. For Sapir, the access point is obviously the form-feeling. 

The relative independence of form and function also manifests itself in a process we may call the ``semantic disinvestment'' of form. By this term is meant that the ``full'' content of linguistic forms may not be activated in all of their occurrences, insofar as forms may be simply conventionally applied to ends to which they are not suited. An example from \emph{Psychology of Culture} may illustrate this point (the square brackets indicate places where the reconstructed ``manuscript'' has been patched by significant additions from the editors): 

\begin{modquote}
Consider, for example, verbs that are not entirely active [in their meaning but are treated as active in the linguistic structure:] in English the subject ``I'' is logically implied to be the active will in ``I sleep'' as well as ``I run''. [A sentence like] ``I am hungry'' might, [in terms of its content, be logically] better expressed with ``hunger'' as the active doer, as in [the German] \emph{mich hungert} [or even the French] \emph{j’ai faim}. In some languages, however, such as Sioux, a rigid distinction is made between truly active and static verbs. […] [It seems, then, that] when we get a pattern of behavior, we follow that [pattern] in spite of [being led, sometimes, into] illogical ideas or a feeling of inadequacy. We become used to it. We are comfortable in a groove of behavior. [Indeed], it seems that no matter what [the] psychological origin may be, or complex of psychological origins, or a particular type of patterned conduct, the pattern itself will linger on by sheer inertia. […] Patterns of activity are continually getting away from their original psychological incitation. \citep[109--110]{Sapir2002}
\end{modquote}

In other words, the SV pattern is disinvested of its full significance when it gets applied to cases in which S is not an active doer and the verb is static (cf. also \citealt[14--15]{Sapir1921}). In English, the generalization of this pattern conforms to the general observation that ``all languages evince a curious instinct for the development of one or more particular grammatical processes at the expense of others, tending always to lose sight of any explicit functional value that the process may have had in the first instance, delighting, it would seem, in the sheer play of its means of expression'' \citep[60]{Sapir1921}. The description of this formal play is couched in terms that can hardly fail to evoke artistic activity. This step is taken most explicitly in ``The unconscious patterning of behavior in society''. In this text, the conception of language as an aesthetic product serves to capture two features of linguistic activity: the disconnection between form and function, yet the fact that the formal consistency of language seems to act as a surrogate of this functional demotivation:

\begin{modquote}
Purely functional explanations of language, if valid, would lead us to expect either a far greater uniformity in linguistic expression than we actually find, or should lead us to discover strict relations of a functional nature between a particular form of language and the culture of the people using it. Neither of these expectations is fulfilled by the facts. [… T]he forms of speech developed in the different parts of the world are at once free and necessary, in the sense in which all artistic productions are free and necessary. Linguistic forms as we find them bear only the loosest relation to the cultural needs of a given society, but they have the very tightest consistency as aesthetic products. \citep[550]{Sapir1927b}\footnote{This interplay between freedom and necessity invites a comparison with what Sapir says of the rules of etiquette: etiquette ``combines a strong moral necessity and tyranny and a felt element of choice'' \citep[236]{Sapir2002}.} 
\end{modquote}

An important aspect of the Sapirian version of the so-called Sapir-Whorf hypothesis may well reside in this aesthetic view, besides, of course, those facets it owes to other motivations, well described in \citet{Joseph2002}, and which relate in particular to the publication of Ogden and Richard's \emph{Meaning of Meaning} (\citeyear{Ogden1923}). In view of this aesthetic dialectic between the free and the necessary, \citet[462]{Allen1986} is quite justified in stating that, for Sapir, the linguistic coercion of thought and the compliance of behaviour with cultural patterns ``is not the grip of a master (culture) upon a slave (the individual) but is, instead, more closely analogous to the felt need of the member of an orchestra to play his instrument in accordance with a musical score''.

The fact that forms may be disinvested of their semantic/psychological content finds its counterpart in Sapir's typology of symbols. In the entry ``Symbolism'', which was written for the \emph{Encyclopedia of the Social Sciences}, \citet{Sapir1934} calls ``referential symbolism'' the kind of symbolism that has been divested of affective content, in contrast to those symbols that act as substitutes for emotionally charged behaviour, which are said to belong to the second main type, that of ``condensation symbolism''.\footnote{The manifestly Freudian ``condensation'', a rendering of \emph{Verdichtung}, only underlines the importance of affect in the way Sapir conceived of this symbolism, whose immediate emotional significance puts it at the origin of symbolization in mankind. There is a certain kinship between Sapir and some views defended by Ernest Jones (1879--1958) in his psychoanalytical essay on symbolism \citep{Jones1916}, in particular a duality of symbolisms correlated with the unconscious/conscious distinction.} During the evolution of mankind, one symbolism has developed from the other: 

\begin{modquote}
It is likely that most referential symbolisms go back to unconsciously evolved symbolisms saturated with emotional quality, which gradually took on a purely referential character as the linked emotion dropped out of the behavior in question. Thus shaking the fist at an imaginary enemy becomes a dissociated and finally a referential symbol for anger, when no enemy, real or imaginary, is actually intended. \citep[565]{Sapir1934}
\end{modquote}

From a psychoanalytical point of view, this was a very neutral and agnostic way of describing the evolution of symbolism, without, for instance, the concept of repression. Quite significant in this respect is the \emph{non-affective} factor adduced by Sapir to explain the development of referential symbolism, namely ``the increased complexity and homogeneity of symbolic material''; that is, the evolution to more richly patterned symbols. This can be brought in relation to Sapir's examples of pattern extensions, and their ``getting away from their original psychological incitation'' (cf. the quotation above). 

\section{The form drive}
\label{sec:fortis:formdrive}

Form for form's sake is the aesthetic motto for explaining the routinization of linguistic processes, against ``mechanical'' accounts which narrowly concentrate on low-level processes:

\begin{quotation}
It is usual to say that isolated linguistic responses are learned early in life and that, as these harden into fixed habits, formally analogous responses are made, when the need arises, in a purely mechanical manner, specific precedents pointing the way to new responses. We are sometimes told that these analogous responses are largely the result of reflection on the utility of the earlier ones, directly learned from the social environment. Such methods of approach see nothing in the problem of linguistic form beyond what is involved in the more and more accurate control of a certain set of muscles towards a desired end, say the hammering of a nail. I can only believe that explanations of this type are seriously incomplete and that they fail to do justice to a certain innate striving for formal elaboration and expression and to an unconscious patterning of sets of related elements of experience. \citep[156]{Sapir1924} 
\end{quotation}

The contrast between the hammering of a nail and speaking is reminiscent of that between blowing a candle and uttering the linguistic sound \emph{wh}; it is, says \citet[34]{Sapir1925}, what separates mere practical behaviour from art.

In the above passage, the mention of an ``innate striving for formal elaboration and expression'' echoes other declarations, such as the following one, in which the aesthetic leitmotiv reappears: ``the projection in social behavior of an innate sense of form is an intuitive process and is merely a special phase of that mental functioning that finds its clearest voice in mathematics and its most nearly pure aesthetic embodiment in plastic and musical design'' \citep[344]{Sapir1927a}. Sapir's appeal to a sort of instinctual ``form-craving'' of the human mind and to an innate sense of form (e.g. \citealt{Sapir1924}, \citealt{Sapir1927a}; see \citealt[445]{Handler1986}) is not without antecedents. His form-drive is reminiscent of the Schillerian \emph{Formtrieb}, which Friedrich von Schiller (1759--1805) characterized as a drive to a free expression of personality and toward insulating a permanent self from ever-changing worldly conditions (\citealt{Schiller1795}, letters 12 to 16). The wedding of this ``form-drive'' to the flow of sensations is accomplished through an aesthetic impulse, the ``play-drive'' (\emph{Spieltrieb}). Even if Schiller's \emph{Formtrieb} was not on Sapir's mind when he wrote \emph{Language}, Jung's (\citeyear{Jung1921}) \emph{Psychological Types}, a book and a theory Sapir was very fond of, may have reminded him of it. For Jung, however, the \emph{Spieltrieb} was not interpreted as an essentially aesthetic attitude, nor as a systematization of formal patterns, but rather the conciliation of abstract thinking and sensation, of ego-centred vs object-centred orientation, and the source of symbolic creativity.

Sapir is not the only linguist of the time to speak of a form-drive. As \citet{McElvenny2016} observes, Georg von der Gabelentz (1840--1893), in a passage of his \emph{Sprachwissenschaft} (\citeyear{Gabelentz20161891}), speaks of a drive towards the creation of forms (\emph{Formungs\-trieb}) which would acccount for the formal lavishness (\emph{Formengeprän\-ge}) of languages, whose profusion goes beyond functional needs. This \emph{Formungs\-trieb} accounts for people's delight in formal play, says Gabelentz, who describes this human urge with Schiller's word \emph{Spieltrieb}; that is, the play-drive which grounds the aesthetic attitude (\citealt[381]{Gabelentz20161891}; no explicit reference to Schiller is made). The play-drive implies that the ``little surplus of effort that I made on my work over and above bare utility was already a piece of love, and gave the dead material a breath of the personal for all time.'' Indeed, continues \citet[344]{Gabelentz20161891}, ``precisely the same thing happened with language'' (trans. \citealt[35]{McElvenny2016}).

In an addition to the second edition of Gabelentz' text, and in the context of the present discussion, Gabelentz' nephew, Albrecht Graf von der Schulenburg, refers back to Wilhelm von Humboldt (1767--1835) (see also \citealt{McElvenny2017a}). Especially praised are speakers of languages which systematize this formal play by resorting to obligatory inflections, that is, speakers of Indo-European languages; in them, says Schulenburg, one finds a specific sense of form (\emph{Formensinn}, a word also found in texts on art) and an outstanding ``aesthetic'' gift \citep[394]{Gabelentz20161891}. While the value judgement might not have been to Gabelentz' taste, I believe the reference to Humboldt puts us on the right track.

Although the term \emph{Formungstrieb}, so it seems, is not used by Humboldt (Jürgen Trabant, p. c.), what we do find in Humboldt is the idea of a formative power which is especially active in some phases of language evolution, a power that Humboldt calls \emph{Bildungstrieb}. The term can be found in two contexts: in texts about language, and in instances where the discussion revolves around biological questions (respectively \citealt{Humboldt1907}, vol. \textsc{vii}: 95, 168, cf. Eng. trans. in \citealt{Humboldt1988}, p. 88, ``constructive urge'', and p. 150, ``formative urge''; \citealt{Humboldt1903}, vol. I: 328). The latter contexts point to the biological source of the \emph{Bildungstrieb}, a concept borrowed from Humboldt's former teacher at Göttingen, Johann Friedrich Blumenbach (1752--1840). For \citet{Blumenbach1781}, the \emph{Bildungstrieb} is a force creating and perpetuating organic forms (Jürgen Trabant, p. c.). In the linguistic domain, the stage at which language forms, as it were, ``outgrow'' the mind seem to be evaluated negatively (cf. \emph{Über die Verschiedenheit}, Eng. trans. §20, in \citealt{Humboldt1988}), which of course separates him from Sapir.

In short, linguistic structures are produced by an instinct which governs the creation of aesthetic objects through its formal play. I believe that by setting Sapir's aesthetic form-drive in the very German genealogy sketched above I am not going far beyond the bounds of decent speculation. This brief digression on the sources of the form-drive is not all the German lead has to offer, as we shall now see.

\section{On the source of the form-feeling: Croce?}
\label{sec:fortis:croce}

We have shown that the form-feeling has its origin in aesthetic theory. Aesthetics is a continent unto itself, and the potential sources are many. Let us first go back to what Sapir himself said about his influence(s). The crucial passage is repeated here:

\begin{quotation}
Probably most linguists are convinced that the language-learning process, particularly the acquisition of a feeling for the formal set of the language, is very largely unconscious and involves mechanisms that are quite distinct in character from either sensation or reflection. There is doubtless something deeper about our feeling for form than even the majority of \emph{art theorists} have divined, and it is not unreasonable to suppose that, as psychological analysis becomes more refined, one of the greatest values of linguistic study will be in the unexpected light it may throw on the psychology of intuition, this ``intuition'' being perhaps nothing more nor less than the ``feeling'' for relations. (\citealt[156]{Sapir1924}; my emphasis)
\end{quotation}

In this passage, ``intuition'' is equated with ``feeling for form''. On the other hand, we have explicit statements by Sapir in \emph{Language} in which he acknowledges his debt to Benedetto Croce (1866--1952); further, in one instance, \citet[224]{Sapir1921} says he borrows the term ``intuition'' from Croce, who in his \emph{Aesthetics} uses ``intuition'' in contrast to ``logical knowledge''. Altogether, this may be conducive to an adventurous syllogism: Sapir owes his notion of intuitive knowledge to Croce, intuitive knowledge = form-feeling, \emph{ergo} Sapir's form-feeling is a version of Croce's intuition, or at least related to it. This conclusion is endorsed by \citet[347]{Modjeska1968}, who claims that in Croce Sapir ``found a confirmation, if not the source of his own thoughts on formal pattern''. \citet{Hymes1969} agrees with Modjeska, while \citet{Hall1969} begs to differ.

That Sapir borrowed ``intuition'' from Croce is acknowledged by Sapir himself, as we just saw. Whether the notion of form-feeling can be conflated with Croce's ``intuition'' is another matter. Croce's definitions of intuition, however hazy, show that intuition is for him a faculty essentially dedicated to the apprehension of individual objects. Further, Croce wields his notion of intuition against an intellectualist view of cognition and implicitly against Kant's concept of intuition; at any rate, his discussion shows he completely misses the Kantian doctrine of forms of intuition and categories, which certainly should detract from its interest for an informed reader.\footnote{Against Kant, Croce says for instance that we have ``intuitions without space and time'': ``Noi abbiamo intuizioni senza spazio e senza tempo: una tinta di cielo e una tinta di sentimento, un `ahi !' di dolore e uno slancio di volontà oggettivati nella coscienza, sono intuizioni che possediamo, e dove nulla è formato nello spazio e nel tempo'' \citep[6--7]{Croce1908}. Here, intuition is used in reference to Kant's \emph{Anschauung} (which adds to the confusion, if anything).}

In the \emph{Aesthetic}, there is no particular emphasis on the grasp of unconscious patterns. On the contrary, genius, as superlative intuition, is essentially conscious, and in the chapter on language (chap. 18), Croce makes clear that parts of speech, which might be taken here as building blocks of linguistic patterning, are dubious abstractions floating above linguistic intuition. Intuition is also, however, a faculty that is inherently expressive, insofar as its operation is fully realized (intuiting a geographical area is being able to draw it, says Croce); this aspect, at least, is consonant with the dynamic character of the Sapirian unconscious \citep[see][]{Allen1986}.

Given the philosophical context in which Croce introduces his notion of intuition, what was its relevance for Sapir? In \emph{Language}, occurrences of ``intuition'' that may be considered to come close to Croce's notion appear in the section devoted to  literary criticism; that is, when discussing the idiosyncrasy of writers. In this section, the irreducibly individual character of an artist's ``intuition'' is said to have its origin in personal experience, within ``thought relations'' which, says \citet[239]{Sapir1921}, ``have no specific linguistic vesture''. If this were not clear enough, shortly after this passage, a distinction is drawn between this personal intuition and the ``innate, specialized art of language'', an art that would seem to be exercised by the form-feeling.

When Sapir uses ``intuition'' in a sense that would appear to be more relevant to his own understanding of the term, he does not go back to Croce but to Jung. In this respect, the way he handles Jung's functional types of mental activity (thinking, feeling, sensation, intuition) is revealing. Of all these types, intuition is singled out. Intuition, he says, is not on a par with the rest of the functional types; it is rather a mode of apprehension which cuts across the other functional types. Intuition is really an awareness of relations provided by a quick rate of apprehension, and the intuitive mind might be described as ``an historical mind, aware of all the relations that are locked up in the given configuration'' \citep[167]{Sapir2002}. Thus, in the realm of abstract thinking, the quick glance of intuition is a privilege of the great mathematician, who sees the answer before it is proven \citep[167]{Sapir2002}. In the realm of sensationist apprehension, intuition is the process which lies behind the ability of a cook to project the result of combining flavours \citep[168]{Sapir2002}. Being thus generalized to all sorts of fields, Jungian intuition is redescribed by Sapir in a way that makes it come very close to the form-feeling.

The aesthetic aspect, however, is not essential in Jung's conception, except when the discussion leads him to find objections to it; for example, when he scrutinizes Schiller's \emph{Spieltrieb}. On the other hand, in the preface to \emph{Language} (\citeyear{Sapir1921}: iii), a short eulogy praises Croce for being ``one of the very few who have gained an understanding of the fundamental significance of language'', and apparently expanding on what this significance consists in, Sapir goes on to say that Croce ``has pointed out its [i.e. language's] close relation to the problem of art'' and that he is ``deeply indebted to him for this insight.''

A glance at what Croce has to say about language, both simplistic and vague, suggests that Sapir, beyond this fundamental insight, could find little of value for his own concerns. First, the notion of ``form-feeling'' does not figure in the theoretical apparatus of Croce. Second, Sapir had reservations about Croce. Thus, in notes he jotted down on Croce he criticizes him for conceding too much to an individual's expressive capacity and not enough to formal conventions \citep{Handler1986}. As a matter of fact, this is a recurring objection. It is for example levelled against Jung and Lévy-Bruhl: we should not transfer to individuals qualities which come from their complying with cultural patterns.

From this excursus on Croce, we may conclude, in agreement with \citet[441]{Handler1986}, that Sapir's analyses of linguistic patterning owe little to Croce, and we should take him at his word when he says that his debt to Croce is one fundamental insight, the connection of language to aesthetics. Further, even if Sapir borrowed ``intuition'' from Croce, his use of the term is his own and may at least as much reflect the influence of Jung.

\section{Form-feeling and Formgefühl: Vischer and Wölfflin}
\label{sec:fortis:vischerwoelfflin}

We are left without an answer to the question of Sapir's sources in aesthetics. I suggest that ``form-feeling'' is in fact a translation of the German \emph{Formgefühl}, a term commonly used by art theorists of the time. Note also that the plural in the quote above (``the majority of art theorists'') points to a notion that is not the prerogative of a single author, and this is indeed the case for the \emph{Formgefühl}. A few words need to be said about the historical background to this notion.

\emph{Formgefühl} has various meanings. In the \emph{Ästhetik}, the magnum opus of the ``ponderous Hegelian'' (Croce's words) Friedrich Theodor Vischer (1807--1887), the term is used abundantly without, however, being thematized as such. Its signification is essentially that of aesthetic sensibility, and it is most often used in connection with a people and a period. It may also characterize one of the opposed principles into which Vischer resolves styles, namely the painterly and the plastic (Vischer is one of the sources for the analyses of styles into opposing pairs; cf. for example the \emph{Principles} of Wölfflin).

Since the psychological aspect of the \emph{Formgefühl} is our first concern here, we should mention that early occurrences of the term in psychological literature can be found in the writings of the great mandarin of the field in Germany, Wilhelm Wundt (1832--1920). \citet{Wundt1874} appears to employ \emph{Formgefühl} in contradistinction to, on the one hand, sensations of (dis)harmony between elementary impressions and, on the other hand, ``intellectual'' contents associated with the perception of forms, including, for example, the functionality of body parts in representations of the human figure (cf. the 1902 edition, chap. 16, part 2). The \emph{Formgefühl} is thus associated with the perception of organization and order.\footnote{My thanks to David Romand for having called to my attention Wundt’s \emph{Grundzüge} and the Herbartians. According to Romand, the Wundtian concept was the one taken over by Lipps and Dessoir \citep{RomandIP}.} In this ``structural'' meaning, its genealogy can be traced back to some of Herbart's followers, namely, to Nahlowsky (1812--1885) and his notions of ``elementary feelings'' and ``group-feelings'' (\citealt{Nahlowsky1862}; \citealt{Romand2018}), and to Waitz (1821--1864) and his observations on the aesthetic effect of \emph{Form} and \emph{Gestalten} (\citealt{Waitz1849}; \citealt{Romand2015}, \citealt{RomandIP}).

The notion of \emph{Formgefühl} seems to gain a larger audience with the advent of an empathy-centred, psychological aesthetics. An important landmark in this tradition, which goes back to the Romantic era, is the work of Robert Vischer (1847--1933), son of Friedrich Vischer.\footnote{For an English introduction to Robert Vischer, see \citet{Barasch1989}.} In a short treatise (his dissertation) entitled \emph{On The Optic Feeling of Form} (\emph{Über das Optische Formgefühl}, \citeyear{Vischer1873}), Vischer explains that contemplating and forming images always implies an active involvement of the body or a projection of bodily feelings and affects onto the object (a projection he calls \emph{Nachfühlung}/\emph{Einfühlung}, ``concurring-feeling'', as it were, and ``empathy''). Thus, a rock facing a subject may appear to defy or challenge her, a road which widens awakens a triumphant feeling etc. This is not yet art, but its prelude: the artist's task consists in imbuing such projected feelings with a more general and spiritual meaning. In sum, the \emph{Formgefühl} is for Vischer a projection of a feeling into a form.

Heinrich Wölfflin (1864--1945), in his first study \citet{Wolfflin1886}, pursues Vischer's line of thought, and applies it more specifically to the description of factors which condition the affective effect produced by an architectural style. In \emph{Renaissance und Barock} (\citeyear{Wolfflin1888}), Wölfflin explains that the features which define a style reflect a way of projecting inner feelings and corporeal habits, characteristic of a period, into forms. The tapering of Gothic forms, for example, reflects a muscular tension and an effort of the will that one does not find in the serene and vigorous equanimity of Renaissance  constructions. Further, the \emph{Formgefühl} offers a psychological definition of style which cuts across arts and thus unites architecture with painting, sculpture and decorative arts (e.g. clothing). This relative homogeneity of style is manifested in recurrent formal patterns (e.g. the pointed elongated shape of Gothic art), of which the \emph{Formgefühl} is therefore both an intuition and a source.\footnote{See, for example, \citet[chap. 3]{Wolfflin1888}; the English translation has somewhat distorted the text, \emph{Formgefühl} being variously rendered as ``formal sensibility'', ``formal response'' and, worse, ``conception of form''.} Moreover, Wölfflin lays great importance on the idea that artistic forms cannot be determined by cultural-historical factors nor by functionality or technical necessity. And although the notion of \emph{Formgefühl} is still framed in an empathy-based theory (or Wölfflin's own version of empathy, the \emph{Lebensgefühl}), the \emph{Formgefühl} itself circumscribes a relatively autonomous formal plane.

On the whole, Wölfflin's formalist style of analysis, which reflects an emphasis on non-representative art (such as architecture), resonates with the great interest of the time in ornamental design and decorative art, exemplified in particular by Alois Riegl (1858--1905; see e.g. \citealt{Riegl1893}) and Gottfried Semper (1803--1879). The latter, for example, placed much emphasis on the role of decorative arts, small artefacts, costume, furniture and architecture (i.e. all objects close to the body; \citealt{Semper1884}).\footnote{On the relation of Wölfflin to Semper and Riegl, see \citet{Payne2012}.} Such an emphasis could hardly be lost on anthropologists who often had to deal with everyday objects. Indeed, Boas does not seem far from Semper when he states that ``so far as our knowledge of the works of art of primitive people extends the feeling for form is inextricably bound up with technical experience. Nature does not seem to present formal ideals, — that is fixed types that are imitated, — except when a natural object is used in daily life; when it is handled, perhaps modified, by technical processes'' \citep[11]{Boas1927}.

In this way, the formalist perspective in aesthetic theory may be considered as a counterpart to Sapir's view on the potential autonomization of linguistic form.

\section{Form-feeling and Formgefühl: Lipps and Dessoir}
\label{sec:fortis:lippsdessoir}

Perhaps most relevant for our concerns are the discussions of Theodor Lipps (1851--1914) and Max Dessoir (1867--1947), in view of their insistence on the structural features of form, and therefore their possibly greater proximity to Sapir's understanding of the form-feeling.

In his \emph{Aesthetics of Space} (\emph{Raumästhetik}), \citet{Lipps1897} draws a parallel between, on the one hand, this form of unconscious and rule-driven knowledge, intuited by feeling, which we exercise when engaged in ``mechanical activities'' (such as riding a bicycle) and, on the other hand, the feeling which rules our speech productions, the ``language-feeling'' or \emph{Sprachgefühl} (a term in common parlance at the time; cf. \citealt{Tchougounnikov}). Further, \citet[chap. 8]{Lipps1897} states that this ``language-feeling'' is akin to the ``form-feeling'' which is built from our bodily experience and our acquaintance with the world of physical objects, and which results in the grasp of general geometrical patterns. These various feelings, though rule-driven, do not rest on an exact memory of past events, since each new case which presents itself is different from the preceding ones; they constitute a \emph{sui generis} kind of knowledge, unconscious and ``amazingly sure'', says Lipps.

In an introduction to his conception of psychological aesthetics, \citet{Lipps1907} explains that the \emph{Formgefühl} is a feeling assigning a value to the way in which parts are articulated into a whole; that is, to the structure of a pattern. The rules which govern this part-whole organization fall under two main principles: those related to the identification of global organization (e.g. rhythm), and those related to the hierarchical structure of the whole. For instance, in the Greek temple, because of the regular disposition of columns, the principle of rhythmic organization prevails, while in the Gothic cathedral the hierarchical principle is dominant. The beautiful is defined as a vital affirmation of the Ego (\emph{Lebensbejahung}), an affirmation which results from a positive empathy, which Lipps attempts to define in not too nebulous terms. Finally, Lipps characterizes art as a formal language (\emph{Formensprache}), and this formal language he identifies with a play with forms endowed with a functional role (e.g. a capital stylized into a vegetal form).

Close to some positions advocated by Lipps, \citet{Dessoir1906} defines the \emph{Form\-gefühl} as that feeling which arises from the structural features of proportion, harmony and rhythm, as well as from the quantitative and intensive aspects of forms. The \emph{Formgefühl} itself is carefully distinguished from feelings associated with pure sensations and the content of aesthetic objects; it is therefore a feeling which revels in the organization of formal elements. Much of the discussion centres on rhythm and music and, in fact, the term \emph{Formgefühl} surfaces from time to time, in addition to Dessoir's text, in discussions about the ``new music'' (\emph{Neue Musik}), such as those of Schönberg and Webern (see e.g. \citealt{Webern1912}). Given Sapir's intense interest for music and the similarity he perceived between music and language \citep[156]{Darnell1990}, these discussions may have been a possible source too.

\section{Conclusion}
\label{sec:fortis:conc}

In a Sapirian spirit we may say that Sapir has assembled into a unique configuration ideas which he had found consonant with his own perspective. That linguistic structures are unconscious was almost a commonplace in the linguistics of the time. However, Sapir's notion of pattern has, to the best of my knowledge, no equivalent. On the one hand, patterns are formed out of groups which are formally and functionally/semantically defined, as in Paul's theory; on the other hand, the combinatorial potential of units, be they phonemes, morphemes or words, helps define unconscious groups, an aspect which brings him closer to Bloomfield. In contrast to Paul, the form-feeling is a window on unconscious structures; its intuitive grasp of linguistically relevant units attests to the psychological reality of forms which abstract away from physical features. The form-feeling warrants, perhaps makes possible, the linguist's labour.

Unconscious patterns were obviously connected in Sapir's mind with the notion of \emph{Gestalt}, and the way Koffka conceived of \emph{Gestalten} may have enticed him to generalize the notion of pattern-\emph{Gestalt} to any culturally significant activity; that is, beyond linguistic behaviour. As to the unconscious structuring of linguistic units, this was not apprehended by Sapir in the ``mechanical'' fashion of Paul, but as the result of the creative facet of the form-feeling, or form-drive. The form-drive and the form-feeling operate in accordance with entrenched patterns, which may have lost their functional motivation. The conventionality or routinization of patterns invites a parallel with what aesthetics knows as style, and we have seen that for Sapir the creation and perception of linguistic pattern is fundamentally of the same order as the artistic attitude. This insight he said he owed to Croce, but, as we have shown, it can be doubted that Croce's influence went far beyond this very general idea.

If the form-feeling is an allusion to the German \emph{Formgefühl}, as was suggested above, it seems legitimate to examine more closely this notion as it circulated in aesthetics, and ask what, among its various aspects in different authors, had seemed to answer to Sapir's concerns. In this respect, Lipps' theory seems to be especially relevant: like the Sapirian form-feeling, Lipps' aesthetic form-feeling is an unconscious form of knowledge which cannot be reduced to a kind of conceptual knowledge, yet it is rule-driven. Further, it is explicitly compared with that feeling for language which regulates speech production. Given his fame, Wölfflin may have come to Sapir's attention and may have suggested to him a parallel between language and style. Moreover, Wölfflin's formalist perspective and in the same respect that of Lipps and Dessoir was also potentially congenial to the Sapirian view of ``form for form’s sake''. In addition, we may speculate that the problem of stylistic change, of major importance for Wölfflin, could suggest a comparison with the question of linguistic change. Finally, the interplay, in art productions, between functionality, stylization and convention, between emotion-laden and detached formal play may have reinforced the Sapirian view of language as an aesthetic form.

\sloppy
\printbibliography[heading=subbibliography,notkeyword=this]

\end{document}
