\documentclass[output=paper]{langscibook}
\ChapterDOI{10.5281/zenodo.2654369}
\author{Nick Riemer \affiliation{The University of Sydney \& Laboratoire d’histoire des théories linguistiques, Université Paris-Diderot}}
\title{Linguistic form: A political epistemology}
\label{chap:riemer}

\abstract{This chapter explores ideological dimensions of contemporary mainstream linguistics, especially with reference to the ``unique form hypothesis'': the assumption that each language has a single form, which it is the role of linguistics to characterize. The chapter surveys grounds for scepticism about the hypothesis, reviews some recent ideological critiques of linguistics, and sketches the contours of a speculative ``political epistemology'' of the unique form hypothesis, suggesting how well-known critiques of the other social sciences might apply to structural linguistics research. It pays special attention to the unique form hypothesis' role as a vehicle for the discretionary intellectual authority of the linguistic expert -- in the pedagogical context, the lecturer who serves as the origin and authority of the ideas about language structure transmitted to the linguistics student. This authority is argued to replicate and so to normalize, in the domain of education, the kinds of relations of social domination on which contemporary political orders rest. As a result of this discussion, the theoretical biases of contemporary linguistics are replaced in the broader socio-ideological context to which they belong, and considered with respect to some classic political critiques of ``bourgeois'' social science.\label{q:riemer:domination}}

\begin{document}
\maketitle

\section{Introduction}
\label{sec:riemer:intro}

What might it mean to assert that a language has a form, and what political and ideological consequences, broadly conceived, might such an assertion currently entail? If, with Konrad \citet{Koerner1999}, we accept that linguists in the past have been ``particularly prone to cater, consciously or not, to ideas and interests outside their discipline and, as history shows, allowed at times their findings to be used for purposes they were not originally intended'', then the question arises as to whether, and if so how, this may also happen today.\footnote{An earlier version of some of these ideas was published as \citet{Riemer2016div}.} One of the functions of the humanities and social sciences in general, it has often been argued, is to supply an apologetics of the dominant political order and the ideologies that accompany it (\citealt{Nizan19711932}, \citealt{Chomsky19781967}, \citealt{Ingleby1972}, \citealt{Baudrillard19881972}, \citealt{Bourdieu1991}). What role in this apologetics might linguistics play? What insights might the ``critical'' tradition in the reflexive social sciences bring to our understanding of the nature of formalist linguistics as an intellectual, disciplinary and ideological project? In exploring these questions, this contribution starts not with the substantives \emph{language} and \emph{form}, but with the indefinite articles accompanying them, and the implications of singularity they introduce. Those articles express central intellectual and ideological characteristics of contemporary ``structural'' or ``formal'' linguistics -- those varieties of the discipline, that is, that posit a unique underlying ``form'' of language and set out to characterize it.\footnote{It is important for what follows to note that, as I use it here, ``formal linguistics'' has a far wider extension than usual. Whereas in its typical use ``formal'' denotes a feature of linguistic methodology -- the use of mathematizable techniques in grammatical analysis -- here it is simply used to refer to any theoretical endeavour that assumes that a language has a unique underlying form. Approaches that avoid this hypothesis are, needless to say, in a distinct minority in the discipline: the most obvious example, no doubt, is integrational linguistics, which rejects the assumption of ``a single vantage point from which language presents itself as forming a unified or homogeneous system'', along with ``the idea of a scientific search for the single best model or set of procedures for analyzing language and communication'' \citep[4, 15]{PableHutton2015}.} In its mapping of grammatical form, this kind of  linguistics deploys reductive, objectivizing and centripetal procedures in order to discern a single structural unity underlying the diversity of observable manifestations of language. The rules, generalizations, and categorizations deployed in describing grammar tend in a single direction: almost exclusively, intellectual effort is devoted to bringing complex facts under the scope of general rules, and to deriving the diverse manifestations of speech, sign and text from the operations of a unique and singular grammar -- the Platonic form that underlies the variety of human language, guarantor of the ``scientificity'' of linguistics.

``There is a single French language, a single grammar, a single Republic'', the French Education Minister, Jean-Michel Blanquer tweeted in November 2017 in the context of calls for French spelling reform in the interests of gender inclusiveness.\footnote{``Il y a une seule langue française, une seule grammaire, une seule République.'' See \url{https://twitter.com/jmblanquer/status/930813255211208707}.} But while traditional grammar has always sought to discern -- or, more frequently, establish -- forms (plural) in language, the ``unique form hypothesis'' -- the idea that each language has \emph{a single} form, which linguists scientifically reveal -- has not been a permanent feature of the discipline, and it is not a necessary one. Its dominance in our period therefore deserves to be critically analysed. In this chapter, I propose to explore the ideological dimensions of the unique form hypothesis in linguistics, particularly with respect to its political and ideological affordances in the era of contemporary ``platform'' capitalism \citep{Srnicek2017}. \citet[19]{Koerner2000} observes that ``linguistics, past and present, has never been `value free', but has often been subject to a variety of external influences and opinions, not all of them beneficial to either the discipline itself or the society that sustains it''. In the spirit of this remark, I will ask what might lend the unique form hypothesis plausibility in the context of the overarching ideological climate in which linguistics research and teaching currently unfold, understanding ideology in Slavoj Žižek's sense of ideas that are ``functional with regard to some relation of social domination (`power', `exploitation') in an inherently non-transparent way'' \citep[8]{Zizek1994}.

According to Christopher Hutton, ``linguists […] generally associate their discipline and the practice of linguistic analysis with a vague form of liberal progressiveness'' \citep[295]{Hutton2001}. A corresponding progressivism of some kind no doubt also characterizes most linguists' political orientation: right-wing political views are probably distinctly in the minority in the profession. Yet ideological critiques identifying political aspects of linguistics, whether progressive or anti-progressive, are only rarely acknowledged in the discipline itself. Many linguists, working under the assumption that they are doing ``normal science'', often hesitate to take ideological analysis seriously, and linguistics' scientific aspirations regularly serve to inhibit any critical reflection on either the epistemological status or political import of the discipline's theoretical results.

Such reflection is nevertheless both important and interesting. As Talbot Taylor notes, 

\begin{quotation}
if purportedly descriptive discourse on language is best reconceived as a (covertly authoritarian mode of) normative discourse, then the assertion of the political irrelevance and ideological neutrality of linguistic science can no longer be maintained. \citep[25]{Taylor1990}
\end{quotation}

\noindent Examining the ideological dimensions of linguistics in order to propose a ``political epistemology'' of the discipline -- an account of the political and ideological conditions that help secure acceptance of theoretical propositions -- can only enrich the way in which linguistics understands itself. Linguists may often prefer to maintain the illusion of a wholly independent and isolated discipline, but we are ourselves well and truly members of the body politic. Understanding how linguistics works also means understanding its connections to society, illuminating the ways in which external stakes can feed into linguistic research, and, correspondingly, how linguistic theories themselves can act on the world outside theory.\footnote{I have explored an aspect of this latter question in \citet{Riemer2019}.}

I begin by briefly surveying a number of the ways in which the reductive and formalizing energies embodied in the unique form hypothesis have been or could be called into question, thereby establishing that the unique form \emph{hypothesis} is, precisely, just that. I then review a number of ideological critiques of linguistics from the past several decades, before sketching the contours of a speculative ``political epistemology'' of the unique form hypothesis, which shows how well-known critiques of the other social sciences might apply to structural linguistics research. In particular, I consider the unique form hypothesis from the point of view of an aspect of the discipline often neglected by historians: its function in undergraduate education. This pedagogical function is essential for an accurate grasp of the ideological dimensions of linguistics as a discipline. I pay special attention to the unique form hypothesis' role as a vehicle for the discretionary authority of the linguistic expert -- in the pedagogical context, the lecturer who serves as the origin and guarantor of the ideas about language structure transmitted to the student. I consider an interpretation of this authority as prefiguring and normalizing, in the symbolic register, the kinds of relations of social domination on which contemporary political orders rest, and to which linguistics undergraduates, like those in other disciplines, must learn, as one of the key functions of a university education, to accommodate themselves. As a result of this discussion, the theoretical biases of contemporary linguistics are replaced in the broader socio-ideological context to which they belong, and considered with respect to some classic political critiques of ``bourgeois'' social science.

\section{Must language have a single form?}
\label{sec:riemer:singleform}

To raise the question is, of course, to bring linguistics' very disciplinary identity into question. In a striking example of the way in which the formal presuppositions of theories can be ignored by their own proponents, however, linguists sometimes react negatively to the suggestion that they are involved in characterizing \emph{the} form of languages, and protest against the dogmatism and epistemological closure that such a claim might be taken to entail. So it is important, at the outset, to register precisely that the very idea of \emph{the grammar}, \emph{the structure}, or \emph{the form} of a language, particularly when coupled with the empirical procedures of modern linguistic ``science'', carries exactly this implication of singularity. To characterize the phonological structure of a language is to claim that the language has -- until, of course, improvements in phonological theory prompt revisions -- \emph{this} phonology, and not some other. Equivalent remarks apply to the other grammatical subfields. Regardless of the open-mindedness or tolerance for different perspectives with which any particular researcher approaches the subject matter, it is intrinsic to the theoretical enterprise of formal linguistics in its contemporary guise that any paradigm within it advances a claim of uniqueness for its current best analysis of a particular grammatical phenomenon. This does not mean formal linguists are epistemic zealots: like any empirical researchers, they replace their current conception of what form is like when a better theory presents itself, all things being equal. But it is only if the unique form hypothesis is adopted that this process of theoretical replacement can be understood. If the presumption was that a multitude of different analyses of a grammatical phenomenon was possible, and that any number of different interpretations of the grammar could be equally ``correct'', or correct in different ways or for different purposes, formal linguistics' basic operational norms could not be sustained. Without the premise that languages have a unique form, monopolistic theoretical competition over the right way to characterize that form becomes meaningless.

The assumption that any language has a single form is, then, necessary to the very practice of the discipline. But the history, and even just the recent history, of reflection on language, within linguistics and outside it, offers ample grounds for calling that assumption into question.

Doing so does not mean denying that languages have forms at all or denying that they can be represented structurally; it just means denying that those forms and representations are necessarily unique. In this section, I will briefly review a handful of considerations which militate against the unique form hypothesis. These considerations do not always contradict the hypothesis directly; but their effect is to weaken its plausibility by calling into question many of the ideas about language or its structure which accompany it. The aim of my review of these considerations is therefore to illustrate that the unique form hypothesis is not self-evidently correct: that it has a case to answer. And if the hypothesis can at least be doubted, then we have good reason to enquire into the forces that quarantine it from that doubt so thoroughly in the modern discipline.

The most serious obstacle to the unique form hypothesis comes, no doubt, from the failure of linguistic theory to reach consensus on what such a form might be. This is more than the necessary ``failure'' intrinsic to ongoing empirical enquiry, in which the inadequacy of the previous best theory and its replacement by a successor is the very mark of scientific progress. This continual renovation of the best theory is not the case in linguistics, since even cursory inspection of the major subfields shows that linguists do not agree even on the premises on which the search for a unique form is to be conducted. Whatever its practitioners might think, linguistics is not characterized in a Kuhnian fashion by a contrast between periods of ``normal science'' in which the details of accepted paradigms are being refined, and wholesale paradigm shifts which thoroughly change the field's basic apparatus. Instead, ``formal'' linguistics is the site of numerous jockeying paradigms, none of which is the object of consensus, investigators not even sharing a single set of metatheoretical criteria on which the relative adequacy of different explanatory models could be judged (for example, only some researchers accept that a theory of syntax should be ``generative'' in the Chomskyan sense). In a situation like this, the idea that language does, in fact, \emph{have} a unique form available for investigators to discover, seems incongruous.

For the purposes of linguistics, a language has a form if it can be reduced to a series of rules (generalizations) which demarcate units which are fully part of the language (``grammatical'' ones) from ungrammatical or less grammatical ones. The true ``form'' of the language (``grammar'', ``competence'', ``\emph{langue}''), then, is the one for which the grammatical theory accounts, the others (``usage'', ``performance'', ``\emph{parole}'') being understood as derivative of this. One does not have to be a generativist to accept this elementary definition: it is implicit in essentially all attempts to explain ``surface'' well-formedness by reference to ``underlying'' grammatical rules. The role of underlying rules in grammar allows us to recognize a second obstacle to the unique form hypothesis. As acknowledged by \citet{Chomsky1986}, among others, the second Wittgenstein's (\citeyear{Wittgenstein20011953}: §§139, 40) sceptical critique of the notion of rule-following, especially in the version popularized by Saul \citet{Kripke1982}, poses a serious challenge to any attempt to ground grammar in rules. There is not the space here to go into the detail needed to develop the arguments that underlie the Wittgensteinian critique properly, or to address the numerous difficult questions it raises.\footnote{For an attempt to do the question greater justice, see section 3.3 of \citet{Riemer2005} and the references there.} We will instead briefly sketch one aspect of it.

Wittgenstein establishes that what constitutes following a rule or being in conformity with a rule is indeterminate: for any given rule, there is any number of different and indeed contradictory behaviours which might be described as following, or as in conformity with it. Applied to language, this means that a grammatical rule cannot correspond to, mandate or ``generate'' any single surface output; and, conversely, that the well-formedness of an utterance in a language cannot be exhaustively explained by any single rule or set of rules. The reason is the following: since, as Wittgenstein demonstrates, the \emph{way} in which a rule is to be followed is always ambiguous, and numerous \emph{different} and contradictory behaviours can all count as conforming to the rule, no rule or set of rules is sufficient on its own to specify a unique output. Any grammatical rule requires a set of further rules which governs the way in which it is to be applied -- yet this combination of first and second-order rules is itself powerless to specify any determinate set of grammatical sentences as its output, since the second-order rule is just as much in need of principles of interpretation setting how it is to be applied, as was the original rule itself. Second-order rules need third-order ones to set their application, third-order ones need fourth-order ones, and so on. An infinite interpretative regress is set in train that has been argued to undermine any attempt to conceive of rule systems such as grammars as anything other than heuristically convenient representations of aspects of language.

The conclusion I have drawn elsewhere about the effect of the ``rule-following argument'' in semantics generalizes to \emph{any} domain of formal structure: the rule-following argument renders \emph{all} competing rule-based explanations of a linguistic regularity equivalent \citep[52]{Riemer2005}. In proposing rules as part of the analysis of language structure, the linguist is relying on a tacit ``background'' of practices which allows her to apply those rules with confidence and generate the output in an apparently definite way, effectively ignoring the interpretative indeterminacy that attaches to any given rule, and even though the rule itself radically underdetermines the ``correct'' result.

The fact that \emph{any} rule relies on an \emph{infinite} number of interpretative meta-rules radically levels the status of the theoretical metalanguages in which grammatical rules are stated vis-à-vis the object languages which they supposedly explain: rather than standing \emph{over} our ordinary language practices in a relation of \emph{theoretical explanation} to them, the metalinguistic enunciation of grammatical rules emerges as simply a \emph{different kind} of linguistic practice, equally reliant on an unexplicated background as the object language practices for which it purports to account, and to which no explanatory priority can therefore be attached. There is, for the later Wittgenstein, simply no more fundamental level of simplicity in the explanatory order than our everyday language use: to seek out some more basic level of theoretical explanation underlying it is to substitute a deluded search for unattainable ``philosophical'' certainty for the (confusingly named) ``grammatical'' description of our everyday practices that is, he believes, the real, therapeutic task of reflection. If to have a form is to be characterized by rules exhaustively explaining well-formedness, Wittgenstein offers a critique of the very possibility of that explanatory project.

Another important line of objection to the unique form hypothesis takes its inspiration from the broad phenomenological tradition, and can be constructed through appeal to a range of thinkers such as Heidegger, Gadamer, Hubert Dreyfus and -- with a very different critical sociological twist -- Bourdieu.\footnote{Christopher Lawn’s (\citeyear{Lawn2004}) comparison of Gadamer and Wittgenstein can usefully be consulted here as an indication of the continuities between this and the previous line of critique.} The premise of this line of thought can be captured in the proposition that ``philosophy [and so linguistics] has from the start systematically ignored or distorted the everyday context of human activity'' and that the everyday world cannot be represented by a ``theory [that] formulates the relationships among objective, context-free elements (simples, primitives, features, attributes, factors, data points, cues, etc.) in terms of abstract principles (covering laws, rules, programs, etc.)'' \citep[25, 28]{DreyfusDreyfus1988}. Applied to linguistics, the essence of this critique can be summed up in the proposition that the formal structure which linguistic science sees as the basis of linguistic competence does not reflect any underlying linguistic ``essence'', but should be understood instead as a product of artificial situations of ``breakdown'' in which our ordinary relation to our linguistic practices has been suspended. The formal rules and categories posited to underlie speech do not, on this account, reflect anything deep about the nature of language: they are, instead, theorists' elaborations of the heuristics to which speakers appeal \emph{post hoc} in order to consciously and artificially rationalize aspects of their unreflective linguistic behaviour. For Gadamer, for instance, outside situations of breakdown, speakers are not just unaware of language as form: they are even unaware of language as \emph{language}:

\begin{quotation}
No individual has a real consciousness of his speaking when he speaks. Only in exceptional situations does one become conscious of the language in which he is speaking. It happens, for instance, when someone starts to say something but hesitates because what he is about to say seems strange or funny. He wonders, ``Can one really say that?'' Here for a moment the language we speak becomes conscious because it does not do what is peculiar to it. \citep[64]{Gadamer19761966}
\end{quotation}

\noindent Form is what is left when meaning has been emptied out. Many phenomenologically inspired thinkers like Gadamer, accordingly, maintain that speaker-hearers are simply not aware of language as form, in the sense of a dimension of speech separated from meaning. In \emph{The phenomenology of perception}, Merleau-Ponty denies that speech and thought (meaning) are ``thematically given'' to the speaker independently of each other: in fact, he says, ``they are intervolved, the sense being held within the word, and the word being the external existence of the sense'' \citep[182]{Merleau-Ponty19621945}. The speaker is not aware of \emph{two} things when speaking, the form (the word) and its meanings (her thoughts); the very division between the two only emerges when the speaker steps out of their unmonitored, prereflective linguistic habitus and adopts an artificially external attitude to it -- Bourdieu's (\citeyear{Bourdieu20031997}: 12) ``scholastic disposition''.

In this situation of what Gadamer calls the intrinsic ``self-forgetfulness'' of language, the idea that structural form underlies linguistic action represents a serious misconstrual: language is, first and foremost, the activity of speaking, and structure -- ``form'' -- is an artificial domain of constructed regularity carved out from it \emph{a posteriori} (for some interesting illustration, see \citealt{Preston1996}). Language is not, therefore, fundamentally semiotic: it should not be seen as a code uniting (more or less) fixed forms with (more or less) fixed meanings:

\begin{quotation}
Signs […] are a means to an end. They are put to use as one desires and then laid aside just as are all other means to the ends of human activity. […~A]ctual speaking is more than the choice of means to achieve some purpose in communication.  The language one masters is such that one lives within it, that is ``knows'' what one wishes to communicate in no way other than in linguistic form. ``Choosing'' one's words is an appearance or effect created in communication when speaking is inhibited. ``Free'' speaking flows forward in forgetfulness of oneself and in self-surrender to the subject-matter made present in the medium of language. \citep[87]{Gadamer19761972}
\end{quotation}

\noindent If language is not a system of signs, and if speech is not to be theoretically conceptualized as the mere \emph{implementation} of an antecedent grammatical structure, then the interest of formal approaches to characterizing this structure is immediately diminished. In terms largely compatible with Gadamer's, \citet[37, italics original]{Bourdieu1991} criticizes ``the \emph{intellectualist philosophy} which treats language as an object of contemplation rather than as an instrument of action and power'', a treatment he sees as perfectly instantiated in the word-plus-definition conception of the vocabulary embodied in dictionaries. The postulation of structure on which formal linguistics depends can be criticized, that is, for a fundamental, intellectualist misconstrual of the nature of our relationship to language, and hence of the nature of language itself. In \emph{Pascalian meditations}, Bourdieu pursues a similar line of thought:

\begin{quotation}
Projecting his theoretical thinking into the heads of acting agents, the researcher presents the world as he thinks it (that is, as an object of contemplation, a representation, a spectacle) as if it were the world as it presents itself to those who do not have the leisure (or the desire) to withdraw from it in order to think it. He sets at the origin of their practices, that is to say, in their ``consciousnesses'', his own spontaneous or elaborated representations, or, worse, the models he has had to construct (sometimes against his own naive experience) to account for their practices. \citep[51]{Bourdieu20031997}
\end{quotation}

\noindent Bourdieu emphasises the intrinsic distortion that this kind of theoretical modelling introduces:

\begin{quotation}
simply because we pause in thought over our practice, because we turn back to it to consider it, describe it, analyse it, we become in a sense absent from it; we tend to substitute for the active agent the reflecting ``subject'', for practical knowledge the theoretical knowledge which selects significant features, pertinent indices (as in autobiographical narratives) and which, more profoundly, performs an essential alteration of experience. \citep[51--52]{Bourdieu20031997}
\end{quotation}

\noindent He concludes that ``it is very unlikely that anyone who is immersed in the scholastic `language game' will be able to come and point out that the very fact of thought and discourse about practice separates us from practice'' \citep[52]{Bourdieu20031997}. If, \emph{simply in virtue of their status as representation and explanation}, theoretical models must be understood as ``distortions'' of the reality they model, there is even less justification for seeing any one particular theoretical model as uniquely accurate. On Bourdieu's account, theoretical knowledge of language, with the forms it posits, is something intrinsically different from the practical knowledge that speaking deploys; there is therefore even less reason to anoint a \emph{single} theoretical representation as the definitive unique body of forms underlying linguistic practice.

The final source of doubt about the unique form hypothesis that I will briefly mention comes from the work of researchers in the ``translanguaging'' movement:

\begin{quotation}
The point is simple: a named national language is the same kind of thing as a named national cuisine. Like a named national cuisine, a named language is defined by the social, political or ethnic affiliation of its speakers. Although the idea of the social construction of named languages is old in the language fields, it is often not understood. The point that needs repeating is that a named language cannot be defined linguistically, cannot be defined, that is, in grammatical (lexical or structural) terms. And because a named language cannot be defined linguistically, it is not, strictly speaking, a linguistic object; it is not something that a person speaks. \citep[286]{OtheguyReid2015}
\end{quotation}

\noindent As is the case in generativism (see, e.g., \citealt{Chomsky2000horizons}), translanguaging scholars only recognize the existence of idiolects, ``the system that underlies what a person actually speaks, … [consisting] of ordered and categorized lexical and grammatical features'' \citep[289]{OtheguyReid2015}. Insofar as ``linguistic form'' is understood as the form of a \emph{language}, where the latter is defined pretheoretically and exemplified by such things as ``French'', ``Turkish'' or ``Arabic'', translanguaging scholars explicitly reject the proposition that such form exists.

\section{Ideological critiques of linguistics: a sampler}
\label{sec:riemer:ideologicalcritiques}

There are, then, many reasons for which the unique form hypothesis might be doubted. But it will be clear that the grounds for scepticism that we have just surveyed are either highly marginal within linguistics, or come from disciplinary traditions outside it. The analytical task that faces us, therefore, is to understand why this is the case. Why is questioning of the unique form hypothesis so alien to mainstream linguistics itself?

In looking to what I am calling a ``political epistemology'' of the discipline, my presupposition is that the answer is, in part at least, ideological. Linguistics occupies a highly independent -- sometimes, indeed, isolated -- position within the contemporary humanities and social sciences. That intellectual autonomy, however, does not entail social, political, or ideological innocence or neutrality. Nor does it mean that linguistics has no effect on the world beyond its own intellectual frontiers, still less that it is not influenced by the overall context in which research is conducted (see \citealt[182]{Joseph2002} for some pertinent observations). So linguists must not, in Talbot Taylor's (\citeyear{Taylor1990}: 20) words, ``continue to mistake theories of the nature of languages and linguistic competence as culturally neutral and value-free, conceiving of ourselves as unbiased conveyors of scientific objectivity''. This is particularly the case given that \emph{acceptable} ideological objectives of grammatical and linguistic analysis are not infrequently avowed perfectly openly. Linguistic description, especially, has often been framed as a progressive intellectual project designed to loosen the arbitrary grammatical authority of social elites.\footnote{This is by no means a uniquely modern framing. According to Talbot \citet[11]{Taylor1990}, the 18\textsuperscript{th} century grammarian Horne ``Tooke argued for a descriptive approach to language in part because he felt it would help to free language from the control of political authorities and would thereby offer access to the use of that powerful instrument by the politically oppressed.''}

On the other side of the ledger, reactionary ideological consequences of modern models of grammar have also often been denounced, though mainly from outside linguistics itself. To generalize massively, the main line of critique can be summarized in the proposition that linguistics' semiotic and cognitive premises entail an instrumentalist, asocial vision of language and humanity, well suited to the liberal ideology of capitalist exploitation. As far as I know, however, the history -- a fascinating one -- of ideological critiques of linguistics remains to be written. In order to situate the ideas that follow, in this section I will briefly mention some more recent critiques, before considering their relation to the unique form hypothesis in the next. Because these critiques are often not well known, I will not hesitate to quote from them generously.

Arguably the most important ideological effects of linguistics are those \rephrase{}{which } it shares with the social sciences and humanities more generally. A common line of critique targets the historical role of disciplines in this category, linguistics included, in promoting norms of bourgeois liberalism, understood as the dominant ideology of competitive capitalism. Applied to linguistics, the core of this critique would focus on the discipline's near-universal construal of language as a sign system, along with the model of autonomous subjecthood that accompanies it. The significance of the semiotic framing of language derives from the status of signs as things which people autonomously and rationally \emph{use} to further their ends:

\begin{quotation}
When you speak, you are using a form of telemetry, not so different from the remote control of your television […] Just as we use the infrared device to alter some electronic setting within a television so that it tunes to a different channel that suits our mood, we use our language to alter the settings inside someone else’s brain in a way that will serve our interests. (\citealt[275--276]{Pagel2012}, quoted by \citealt[75]{EnfieldSidnell2017})
\end{quotation}

\noindent The semiotic view of language entails that speakers' and hearers' relation to language is essentially instrumental, but not only in the way acknowledged in the quotation: quite aside from any effect produced by signs on hearers (``alter[ing] the settings inside someone else’s brain in a way that will serve our interests''), the speaker uses signs in order to convey the coded meanings which correspond to the conceptual or denotational content they wish to express, following the rational determinations of an underlying code. This semiotic-instrumental conception of language as code locates the source of speech uniquely in the individual, and wholly obfuscates social determinants of linguistic acts.\footnote{The individualist bias of theories in pragmatics has been a particular object of criticism by scholars working on non-Western and postcolonial communities: see \citet{AnchimbeJanney2017} for a summary.} Structural linguists, \citet[44]{Bourdieu1991} says, ``merely incorporate into their theory a pre-constructed object, ignoring its social laws of construction and masking its social genesis''. This asocial vision elevates the individual's means-end rationality as the all-important parameter governing speech. It thereby promotes the fantasy of a rational, sovereign, and unfettered subject with a uniform code at her disposal, free of the constraints introduced by class, gender or ethnicity, either in the speech situations in which she might participate, or in her access to the code itself. This is the very ideology of autonomous rational agenthood that accompanied, for instance in Locke, the development of bourgeois liberalism and the market economy, and that is essential to their justification \citep{Losurdo2014}. The ideological rationale for the ``free'' market rests on the fiction of the subject as \emph{homo economicus}, a maximally informed, rational and independent agent of commodity transactions in an individualized, competitive market -- a fiction obligingly affirmed not only by the semiotic conception of language and its various philosophical elaborations, but by the premises of much other work in the humanities and social sciences.

The advent of the forms of heavily authoritarian capitalism characteristic of the administered economies of the twentieth century elicited a famous ideological critique from members of the Frankfurt School. This critique of ``instrumental reason'' -- the term is Horkheimer’s (\citeyear{Horkheimer19921947}) -- has clearly been of significant influence on the more specific critiques of linguistics we will consider shortly. The modern West, Horkheimer and Adorno claimed, perverts reason, reifies domination as law and organization, and leads to a ``nullification'' of the individual in the face of dominant economic powers \citep[xvii]{HorkheimerAdorno20021944}. ``Bourgeois society,'' they say

\begin{quotation}
is ruled by equivalence. It makes dissimilar things comparable by reducing them to abstract quantities. For the Enlightenment, anything which cannot be resolved into numbers, and ultimately into one, is illusion; modern positivism consigns it to poetry. Unity remains the watchword from Parmenides to Russell. All gods and qualities must be destroyed. \citep[4--5]{HorkheimerAdorno20021944}
\end{quotation}

\noindent The effect is that individuals, in all their particularity, contradictions and \emph{anomie}, ``are tolerated only as far as their wholehearted identity with the universal is beyond question'' \citep[124]{HorkheimerAdorno20021944}, where ``the universal'' stands for the permanence of social compulsion, the form in which the inexorable power of the modern socioeconomic order confronts individuals, ``who must mold themselves to the technical apparatus [of the economy] body and soul'' \citep[23]{HorkheimerAdorno20021944}. The principle of universality or identity, in this vision, ``strives to suppress all contradiction'', a process which, as Terry \citet[127]{Eagleton1991} puts it, ``has been brought to perfection in the reified, bureaucratized, administered world of advanced capitalism''.

This process of universalization and suppression of difference is reflected in linguistics' construal of the activity of speech as selection from a shared, formalizable semiotic code: rather than intersubjective expressions of ourselves, meanings are reified (commodified) components of a formal calculus which we freely exchange to accomplish certain goals, and from which we are therefore essentially alienated. By installing the same formal code in the head of every speaker, grammatical theory accomplishes a wholesale cognitive uniformization, offering a striking illustration of Horkheimer and Adorno's (\citeyear{HorkheimerAdorno20021944}: 3) claim that for the modern sensibility ``anything which does not conform to the standard of calculability and utility must be viewed with suspicion''. Rationality is domination's \emph{nom de guerre}: ``the impartiality of scientific language,'' \citet[17]{HorkheimerAdorno20021944} say, ``[…] merely provide[s] the existing order with a neutral sign for itself''.  Christopher Hutton’s (\citeyear{Hutton1999}) demonstration of the links between German ``mother tongue'' linguistics and National Socialism provides sobering empirical illustration of Horkheimer and Adorno's ideas, in a mode inflected by vitalistic and mystical sensibilities.

The structuralist emphasis on the unique form underlying speech forces the contingency of linguistic convention into the background and thereby displaces attention from its changeable character. This displacement is, indeed, intrinsic to the very project of writing \emph{a} grammar of \emph{a} language, where both are thought of as inherently singular. The totalizing and singularizing picture of language that emerges contributes to what has sometimes been identified as the wider ideological purpose of the social sciences in general: to distract attention from the alterability of human social arrangements, thereby affirming the inevitability of the status quo, reflecting ``a world of objects frozen in their monotonously self-same being, […] thus binding us to what is, to the purely `given'\thinspace'' (\citealt[126]{Eagleton1991}, on Adorno). Linguistics' structural universalizing, on this vision, has the effect of dematerializing the conception of humanity by abstracting it from local circumstances. By stressing what is supposedly necessary to an undifferentiated human nature, it forces the contingency of the social order into the background and displaces attention from their volatile and hence politically modifiable character. In a similar vein, Blommaert and Verschueren (\citeyear{BlommaertVerschueren1991}, \citeyear{BlommaertVerschueren1992}) have characterized the assumptions lying behind much work in linguistics as ``the dogma of homogeneism'': ``a view of society in which differences are seen as dangerous and centrifugal, and in which the `best' society is suggested to be one without intergroup differences'' \citep[362]{BlommaertVerschueren1992}. In enforcing a singular vision of linguistic structure, theoretically suppressing linguistic variation, and tacitly canonizing the standard (often, national) language, the unique form hypothesis contributes centrally to the homogeneist dogma.

Critiques along these lines have a long pedigree. Almost forty years ago, Deleuze and Guattari highlighted what they took to be the political implications of the modern linguistic project. ``Since,'' they asked, ``everybody knows that language is a heterogeneous, variable reality, what is the meaning of the linguists' insistence on carving out a homogeneous system in order to make a scientific study possible?'' Their answer deserves to be quoted in full:

\begin{quotation}
It is a question of extracting a set of constants from the variables, or of determining constant relations between variables (this is already evident in the phonologists' concept of commutativity). But the scientific model taking language as an object of study is one with the political model by which language is homogenized, centralized, standardized, becoming a language of power, a major or dominant language. Linguistics can claim all it wants to be science, nothing but pure science -- it wouldn't be the first time that the order of pure science was used to secure the requirements of another order. What is grammaticality, and the sign S, the categorical symbol that dominates statements? It is a power marker before it is a syntactical marker, and Chomsky’s trees establish constant relations between power variables. Forming grammatically correct sentences is for the normal individual the prerequisite for any submission to social laws. No one is supposed to be ignorant of grammaticality; those who are belong in special institutions. The unity of language is fundamentally political. \citep[100--101]{DeleuzeGuattari19871980}
\end{quotation}

``Linguistics,'' \citet[21]{Deleuze1977} comments elsewhere, ``has triumphed at the same time that information has been developing as power, and imposed its own image of language and thought, suitable for the transmission of slogans and the organisation of redundancies''.

Recognition of an underside of theoretical analysis -- its ideological links with socio-political domination -- is a constant in modern analysis of instrumental reason. Deleuze and Guattari's remarks recall Althusser's (\citeyear{Althusser2015}) discussion of the ideological import of idealist philosophy. For Althusser, the very project of ``philosophical languages'' à la Descartes or Leibniz -- a tradition well and truly alive in contemporary linguistics -- unwittingly serves an inherently authoritarian and conformist political stance. The ``vertiginous exercises'' of philosophical analysis, Althusser tells us, ``are not neutral'', but intrinsically beholden to the power of the status quo:

\begin{quotation}
Even if they have no object, they have well known objectives, or, at least, stakes. Since they speak of order, they speak of authority and thus of power, and since there is no power other than the established one, that of the dominant class, its power is the one they serve, even if they don’t know it, and especially if they believe they are combatting it. \citep[107]{Althusser2015}\footnote{Original: ``S'ils n'ont pas d'objet, ils ont des objectifs, ou, à tout le moins, des enjeux bien connus. Comme ils parlent d'ordre, c'est qu'ils parlent d'autorité, donc de pouvoir, et comme il n'est de pouvoir qu'établi, celui de la classe dominante, c'est le sien qu'ils servent, même s'ils ne le savent pas, et surtout s'ils pensent le combattre.'' In all cases where no translation is cited in the bibliography, translations are my own.}
\end{quotation}

Sandrine Sorlin follows a similar line in criticizing the totalizing and reductive vision of linguistic theory:

\begin{quotation}
The common denominator of philosophical and universal language, standard languages, and […] the Saussurean concept of ``langue'' could be a single attempt at reduction and autonomization. […] Like the universal languages which linguistically take account of the world in a single ``glance'', grammatical and linguistic activity is motivated by the same ``aim of linguistic unity'' consisting in making language ``single and visible''. \citep[103]{Sorlin2012}\footnote{``Ce qui pourrait être le dénominateur commun des diverses entreprises linguistiques étudiées, à savoir les langues philosophiques et universelles, les langues standard, et, ici, le concept saussurien de `langue', c'est une même tentative de réduction et d'autonomisation. […] À l’image des langues universelles qui rendent linguistiquement compte du monde d’un seul `coup d’œil', l’activité grammaticale et linguistique est animée par la même `visée d’unité langagière' consistant à rendre la langue `une et visible'.''}
\end{quotation}

\noindent -- for her, a highly ideological result:

\begin{quotation}
While ``pure'' linguistics believes itself to be neutral from the political and social point of view, without being conscious of it, it is at base eminently ideological. Its implicit acceptance of pre-established political categories is masked by its methodological rigour. \citep[113]{Sorlin2012}\footnote{``[…] alors même que la linguistique `pure et dure' se croit neutre du point de vue politique et social, sans en être consciente, elle est au fond éminemment idéologique. Son acceptation implicite des catégories politiques préétablies est masquée par sa rigueur méthodologique.''}
\end{quotation}

Even more recently, Philippe \citet[73]{Blanchet2016} has drawn attention to linguistics' disciplinary role in maintaining ``glottophobia'' -- ``the directly human, social, political and ethical dimensions of linguistic discrimination'' -- by promoting ``a dissociation […] between language and society, between linguistic practices and speakers, between linguistic forms and individual and collective forms of existence''. The debt of this analysis to the Frankfurt School is clear:

\begin{quotation}
This dissociation has been effected by a long western intellectual -- including philosophical and scientific -- tradition, which has conceptualized ``language'' [\emph{la langue}] as a cognitive tool: as a \emph{tool}, it is therefore supposedly exterior to the human and able to be evaluated, changed, validated or invalidated from a strictly technical point of view; as a set of cognitive operations, it is supposedly exterior to the social and able to be evaluated, developed, implemented or corrected from a strict neurological and mathematical point of view. \citep[73--73]{Blanchet2016}\footnote{``Au-delà de l'adhésion cynique à un projet de société inique, ce qui rend possible le masquage de la glottophobie, c'est-à-dire des dimensions directement humaines, sociales, politiques, éthiques, des discriminations linguistiques, c'est une dissociation opérée entre langue et société, entre pratiques linguistiques et locuteurs, entre formes linguistiques et formes d'existence individuelle et collective. Cette dissociation a été réalisée par une longue tradition intellectuelle occidentale, y compris philosophique et scientifique, qui a conceptualisé `la langue' comme un outil cognitif: comme \emph{outil}, elle serait extérieure à l'humain et pourrait être évaluée, modifiée, validée ou invalidée d'un point de vue strictement technique; comme ensemble d'opérations cognitives, elle serait extérieure au social et pourrait être évaluée, élaborée, implémentée ou corrigée d'un strict point de vue neurologique et mathématique.''}
\end{quotation}

Finally, it is necessary to mention critiques of the well-known links between linguistics and colonialism. Christopher Hutton has emphasized that ``the history of modern linguistics […] is coextensive with that of high colonialism and inextricably tied to it.'' ``The practices of descriptive linguistics,'' \citet[291]{Hutton2001} writes, ``require forms of privileged social access, and the attempt to set up a typology in which the relationships between the world's languages are laid out is an expression of a universal `panoptic vision'\thinspace''. This theme will be taken up in the following sections.

\section{Ideology as process or as magic}
\label{sec:riemer:ideologymagicprocess}

The critiques of orthodox linguistics we have now surveyed should be taken seriously. Yet, when they are not simply ignored, most of them are likely to be characterized as arbitrary or unbalanced. Penelope \citet[391]{Brown2017}, for instance, summarily dismisses Bourdieu-inspired objections to the Brown and Levinson politeness framework as ``postmodern posturing'' which, she thinks, conveys the ultimatum ``study a phenomenon my way or not at all''. By no means all linguists would assent to Brown and Levinson's politeness theory. By contrast, the unwillingness to entertain foundational challenge evident in Brown’s reaction is, unfortunately, far more characteristic of the discipline.

It is nevertheless true, in so far as it is possible to judge, that most linguists would explicitly oppose the universalizing and dominating politics which the critiques we have surveyed associate with the discipline. That is only to be expected: ideology would not exist if consciously held intentions and beliefs were transparently reflected in their holders' intellectual and discursive practices.

In any case, the totalizing, hegemonic dimensions of linguistic theory that critics have identified are certainly not the only ones which students will retain from their undergraduate training. The intellectual climate of linguistics is, as I have already noted, surely mostly progressive, opposed to discrimination and, above all, antiracist. As one American textbook expresses it, ``looking more closely at languages, and in particular at languages that might seem exotic to us, can make us more tolerant'' \citep{Gasser2012}. Opposition to ``prescriptivism'', which is hammered into students from the first moments of their linguistic study, is the most concrete manifestation of this kind of ``tolerance''. As for another core aspect of linguistics pedagogy, structural analysis of unfamiliar languages, there is no doubt that this can offer powerful lessons in human diversity. Linguistics also fosters values like curiosity, logical rigour, and appreciation of difference, along with other mental capacities which can be harnessed for anti-reactionary and critical ends. It is surely not among linguistics graduates that one should seek virulent racists.

These considerations are certainly relevant, but they do not disprove the existence of the ideological effects discussed in the previous section. Instead, they suggest that linguistics is not ideologically uniform, and that those effects are not the only ones which need to be taken into account. In its intellectual and educational practices, linguistics is, like any complex intellectual institution, heterogeneous: on the one hand, its practitioners are mostly characterized by an open, liberal, vaguely left-leaning political ethos which the discipline's content cannot but reflect in some ways; on the other, linguists' theoretical assumptions are inherited from longstanding, often more conservative, intellectual traditions which should not be expected to be in phase with this encompassing political culture, and which allow the discipline to be valued precisely \emph{as} an autonomous field with its own traditional modes of internal validation.

However, for the purposes of the ideological critique of a discipline, it is not enough to reason solely from the \emph{content} taught to students. Doing so would leave us open to Baudrillard's important objection against what he calls a ``magical'' conception of ideology. In standard ideological critique, \citet[79, italics original]{Baudrillard19881972} says, ``ideology […] always appears as the overblown discourse of some great theme, content, or value […] whose \emph{allegorical} power somehow insinuates itself into consciousness (this has never been explained) in order to integrate them. These become, in turn, the \emph{contents of thought} that come into play in real situations''. But critique of this kind of ideological effect, he says,

\begin{quotation}
feeds off a magical conception of its object. It does not unravel ideology as form, but as content, as given, transcendent value -- a sort of mana that attaches itself to several global representations that magically impregnate those floating and mystified subjectivities called ``consciousnesses.'' \citep[79]{Baudrillard19881972}\footnote{See \citet{Larrain1994} for a discussion of Baudrillard on ideology.} 
\end{quotation}

\noindent As we have seen, ``instrumental reason'', ``rationalism'', ``individualism'', ``homogeneism'', ``ethnocentrism'' or ``colonialism'', are among the ``great themes, contents or values'' that have been argued to be conveyed by linguistics or the social sciences in general. Baudrillard's critique would consist in asking exactly \emph{how} these themes come to \emph{actually affect} linguistics students' beliefs and practices, as well as their consciousness: the failure to address this point no doubt accounts for the tenuous, far-fetched, or arbitrary impression that the critiques discussed in the last section may have left on some readers. 

In order to develop a non-magical, non-allegorical account of the effect of ideological content, Baudrillard insists that attention must be paid to the \emph{forms} and \emph{processes} of that content's transmission. Ideology, \citet[80]{Baudrillard19881972} claims, \emph{is} nothing less than ``the process of reducing and abstracting symbolic material into a \emph{form}'' (italics added). This means that the analytical challenge is to account for ideological effects in a way that explains how abstract doctrines influence practice and consciousness not by simply positing a black box, but by taking into account the \emph{processes} involved in the generation of ideological effects at their point of production. If we want to critique linguistics effectively, we have to do better than solely discerning abstract analogies or ``allegories'' between linguistic and political ideas. Neither bourgeois individualism, nor colonialism, nor any of the other ideological values purportedly conveyed by linguistic theorizing can be claimed to be automatically induced in students simply because linguistic theory can be described as ``individualist'' or ``colonialist'' in certain respects. If they could, it would be possible to detect harmful ideological effects under any disciplinary bed, justifying the impression of arbitrariness that ideological critiques risk giving.

To properly establish an argument about the ideological tenor of a discipline, analysis of content must be linked to analysis of the discursive and other material forms in which that content is transmitted \citep[cf.][]{Debry19961994}. Attention to linguistics as a set of \emph{educational forms}, \emph{processes or practices} therefore calls for analysis, since it is through exposure to those forms and participation in those practices in the course of disciplinary socialization that new audiences of students are brought to take on the attitudes and practices of ideological interest. The different effects that linguistics might have on students constitute the most significant concrete influence that the discipline has, but they are almost never discussed seriously. When it comes to analysis of the real effects of education in linguistics, the discipline typically does not come closer than conventional rhapsodic claims of linguistics' ability to equip students for the needs of the information economy:

\begin{modquote}
\sloppy
Students who major in linguistics acquire valuable intellectual skills, such as analytical reasoning, critical thinking, argumentation, and clarity of expression. This means making insightful observations, formulating clear, testable hypotheses, generating predictions, making arguments and drawing conclusions, and communicating findings to a wider community. Linguistics majors are therefore well equipped for a variety of graduate-level and professional programs and careers. (Linguistic Society of America, ``Why major in Linguistics?''\\
\url{https://www.linguisticsociety.org/content/why-major-linguistics}, 10 April 2018)\todo{maybe move all URLs to footnotes due to tough hyphenation issues?}
\end{modquote}

How, then, might a more serious, non-``magical'' account of the ideological effects of linguistics pedagogy advance beyond the kind of marketing discourse evident in this claim?

To answer, we can start with the epistemic or justificatory status which students are encouraged to attribute to linguistic knowledge. As I have emphasized, this knowledge does not command a similar level of disciplinary consensus to the results taught to students in the ``hard'' sciences -- far from it.  Nevertheless, linguistic theory, particularly those parts of it constituting the core of the discipline -- those, in other words, in which the unique form hypothesis is asserted most categorically -- are regularly presented to students as ``scientific'', and therefore as enjoying an epistemic authority \emph{qualitatively} similar to that of the natural sciences -- not as great, certainly, but nevertheless of the same basic kind. Some linguists would no doubt hesitate to make that claim openly, substituting for ``scientific'' expressions like ``empirical'' or ``systematic'', but the idea is always there in the background, as can be easily confirmed by an inspection of Linguistics websites, including that of my own department, with their explicit references to ``science'', or to unmistakably ``scientific'' methodologies (``discovering the common properties'' of languages or of ``the human language capacity''; italics below are added):

\begin{quotation}
Linguistics is the \emph{scientific study of language}, aimed at finding out what language is like, and why. Each of the world's 6000 languages is a rich and textured system, with its own sounds, its own grammar, and its own identity and style. From the Amazon to Africa, from Southeast Asia to Aboriginal Australia, we use language to think with, to persuade others, to gather information, to organize our activities, to gossip, and ultimately to structure our societies.
(\url{http://sydney.edu.au/arts/linguistics/}, 25 July 2018)
\end{quotation}

\begin{quotation}
\emph{Sciences of language} degree. Linguistics sets itself the task of discovering the common properties of languages by studying their formal properties, their history, their diversity, their acquisition, and their pathologies.
(\url{http://www.linguist.univ-paris-diderot.fr/\_media/plaquette_licence_sciences_du_langage_nov_2016.pdf}, 25 July 2018)\footnote{``Licence \emph{Sciences du langage} \textsc{sdl}. La linguistique s'efforce de dégager les propriétés communes des langues en étudiant leurs propriétés formelles, leur histoire, leur diversité, leur apprentissage, leurs pathologies.''}
\end{quotation}

\begin{quotation}
Yes, \emph{linguistics is a science}! […] \emph{Linguists develop} and \emph{test scientific hypotheses}. Many linguists appeal to statistical analysis, mathematics, and logical formalism to account for the patterns they observe.\\
(\url{http://www.linguisticsociety.org/content/why-major-linguistics}, 24 July 2018)
\end{quotation}

\begin{quotation}
It is impossible to overstate the fundamental importance of language to individuals and society. Linguistics—the \emph{scientific study of language structure}—explores this complex relationship by asking questions about acquisition, production, comprehension and evolution.\\
(\url{https://arts-sciences.buffalo.edu/linguistics.html}, 24 July 2018)
\end{quotation}

\begin{quotation}
Linguistics is the \emph{science of language}. It is not about learning a new language; rather, we study everything about language itself, ranging from how speech is produced to the relationship between language and the human mind / brain, and the role language plays in society.\\
(\url{http://www.humanities.uct.ac.za/hum/departments/linguistics}, 24 July 2018)
\end{quotation}

\begin{quotation}
Linguistics is the \emph{scientific study of human language}, from the sounds and gestures of speech up to the organization of words, sentences, and meaning. Linguistics is also concerned with the relationship between language and cognition, society, and history.\\
(\url{https://www.ling.upenn.edu/}, 24 July 2018)
\end{quotation}

\begin{quotation}
General Linguistics cross-linguistically explores the structures of the sound systems, morphology, phrase construction, meaning and use of linguistic expressions and attempts to derive these from general laws of communication and the human capacity for language
(\url{https://www.linguistik.hu-berlin.de/de/institut/professuren/allgemeine-sprachwissenschaft/allgemeine-sprachwissenschaft-s-prof/}, 25 July 2018)\footnote{``Die Allgemeine Sprachwissenschaft untersucht einzelsprachübergreifend die Strukturen der Lautsysteme, der Wortbildung, des Satzbaus, der Bedeutung und der Verwendung sprachlicher Ausdrücke und versucht diese aus allgemeinen Gesetzmäßigkeiten der Kommunikation und der menschlichen Sprachfähigkeit abzuleiten.''}
\end{quotation}

\noindent The ideology of scientificity reflected here is not just a matter of academic marketing: it remains embedded in linguistics education throughout undergraduate studies in the subject and, in generativism, is strongly asserted in the form of opposition to ``methodological dualism'' (\citealt{Chomsky1995}; see \citealt[367]{Johnson2007} for a defence of the claim that there is ``a remarkably tight point-by-point agreement between the relevant aspects of linguistic methods and the underlying logic of the other sciences''). Whatever the paradigm in question, approaching languages ``scientifically'' means discovering a unique form underlying the diversity of speech.

In the context of their ``scientific'' study of language, students are explicitly or implicitly encouraged to accept the following broad presuppositions:

\emph{Totalizing objectivity}. The language practices of human communities should be approached from the point of view of their formal and structural coherence, with the aim of reducing them to the single (ideational) reality of linguistic structure (grammar). The structural reality thereby discovered is factual and objective on every level of linguistic analysis: every language has a unique, precise and discoverable level of semantic content, a unique representation of morphosyntax and phonology; a unique information structure, etc. It is the job of linguistic theory to reveal all these levels, with the overall aim of bringing to light \emph{the} grammar of \emph{the} language.

\emph{Reducibility}. Actually observed utterances are therefore the imperfect realizations of a level of underlying, more regular structure. The flux of ``performance'', replete with non-normative structures, is mined to extract fixed categories, on the hypothesis that variation is not essential to language, but the cloak in which an invariant structure is concealed. Actual utterances, with their numerous ``ungrammatical'' phrases, ``sentence fragments'' and ``production errors'', are thus degraded in comparison to the underlying representations which they imperfectly realize, and which can be captured in a unique and stable metalanguage in a way which reconciles cultural and cognitive diversity. The recognition of variation is not consistent throughout the discipline: as is often admitted by variationists, the study of linguistic variation is principally concerned with dominant languages.\footnote{Cf. the ``Widening horizons: cross-cultural approaches to linguistic variation'' workshop at \textsc{nwav} 45 in 2016, whose abstract starts with the words: ``Despite great advances in variationist sociolinguistics in the last decades, a major limitation is the fact that the great majority of studies are done on relatively few languages; existing work in our discipline also leaves non-Western societies massively under-represented. Hence the accepted wisdom and prevailing theories and models in sociolinguistics actually rest on a culturally narrow base'' (conference booklet, pp. 32--33, \url{http://web.uvic.ca/~ddenis/NWAV\%2045\%20Booklet.pdf}).} In one's own language, variation can therefore be studied, but in someone else's, uniformity is assumed.

\emph{Formalizibility}. Languages lend themselves to a formal or quasi-formal description through rule-systems.

\emph{Transparency}. This formalization is, most often, transparent (intuitive, shallow), in the sense that the rules believed to underlie the object-language can be expressed in the theorist's native language without any need for this to be enriched with an extended apparatus of technical concepts. For instance, ``thematic roles'' (agent, patient, recipient, etc.), a core component of descriptive and theoretical grammar, are defined through ordinary language expressions (``move'', ``action'', ``place'', ``possession'', etc.), and definitions of Vendlerian aspectual categories make reference to commonsense notions like ``limited'', ``instantaneous'', and so on. Wierzbicka and Goddard's well known Natural Semantic Metalanguage (\textsc{nsm}) framework is a striking example: in this theory, all possible word meanings are reduced to intuitive definitions in ``natural'' language, supposedly without the least technical accretions \citep[see][]{Wierzbicka1996}. To analyse semantics in \textsc{nsm}, there is therefore no need to develop a sophisticated technical apparatus: ordinary language suffices. Not all semantic theories, of course, are as reluctant as \textsc{nsm} to adopt a technical metalanguage. Neverthlesss, transparency is characteristic of a large part of linguistics, especially as it is presented to students.

\emph{The authority of linguistics}. Thanks to the properties of objectivity, reducibility and formalizability, linguistics is a science, and linguists hold an intellectual authority which qualifies them to pronounce on human linguistic nature in their own right, without mastering the technical competencies of the biological or brain sciences.

None of the principles I have listed would be accepted without qualification by all linguists. Nonetheless, they constitute a reasonably accurate summary of the hypotheses that most students studying ``mainstream'' linguistics, especially in the English-speaking world, are encouraged to embrace during the early years of their linguistics study. These years are, of course, the operative period for the purposes of analysing the discipline's most important ideological effects, because most students never advance to a stage where the premises of linguistic research are seriously challenged or complexified: in order to study the ideological effects of linguistics, it is undergraduates, not doctoral students, who should be observed.

From the beginning of their linguistic studies, students learn that language can legitimately be approached in the highly systematizing and totalizing way that is necessary if a unique form underlying the plurality of a speech community's linguistic practices is to be revealed. In thinking about language within their own society or outside it, students are trained in an essentially reductive and classificatory approach to human diversity. This framework defines a unique, idealized, normative model of language and meaning (the ``language faculty'', ``linguistic universals'', ``grammatical structure'', ``semantic/conceptual structure''), with reference to which linguistic variability is conceptualized. Almost always, intellectual effort in linguistics is devoted to referring complex and multifaceted facts to a framework of general rules, in order to reduce the motley variety of human languages to the operations of a unique and singular structure. The reductive, universalizing and classificatory mental habits formed during this training constitute, I believe, the principal mechanisms by which the ideological effects identified in the previous section are created.

In fostering their capacities of abstraction and idealization in the context of the unique form hypothesis, students are trained and examined in formal techniques of reduction and analysis much more than in hermeneutic ones of interpretation or complexification. In line with this orientation, students are often taught that:

\begin{itemize}
    \item Predicates belong to a small handful of semantic categories (those of \citealt{Vendler1957});
    \item A language's vocabulary can be exhaustively categorized into a finite set of lexical categories;
    \item Discourse has a basic unit — the phrase, utterance, or turn;
    \item Some phrases are grammatical, others ungrammatical;
    \item Propositions (truth-conditional statements) are at the base of meaning;
    \item Conversation is governed by a small number of conversational maxims or similar principles;
    \item Speech acts can be taxonomized into a finite number of specific categories;
    \item Words' diverse uses can be reduced to a meaning or definition, or a finite set of these, reducible in turn to a set of conceptual primitives.
\end{itemize}

Behind the variety and complexity of human speech acts, a kind of underlying force or power can therefore be identified: abstract linguistic reason, the essential properties of the linguistic ``system'', deriving from psychological, biological or quite simply grammatical constants. 

On the whole, the concepts I have just listed are not approached as \emph{partial interpretative perspectives} on linguistic facts, useful for certain specific purposes. Instead, they are reified, and claimed to constitute the permanent essence of linguistic structure. Linguistic diversity ends up being understood as what is left after the maximum number of cross-linguistic generalizations have been extracted. Linguistic aspects of human life are presented as the rational products of underlying rule-systems. For this to be plausible, significant idealization is necessary: what is studied are ``grammar'', ``vocabularies'', ``language families'' -- imaginary, idealized constructs remote from, and not easily de-idealized to, situated acts of speech.

It is precisely because they have been idealized that languages admit the generalizations about them that students are encouraged to make. It goes without saying that both generalization and idealization are necessary and unavoidable in intellectual activity and there could be no question of studying language without them. But they can be presented to students in different ways, and the universalizing and reductive manner in which they are currently understood in linguistics is only one of them \citep[cf.][]{StokhofLambalgen2011}.

What ideological effect might this kind of training have? There are two aspects that are worth exploring: the implications of the fact that the universalizing and reductive theorization of unique form is conducted in a Western, usually English, metalanguage; and the meaning of the universalizing and reductive energies themselves that students are trained to channel. We will deal with these in turn in the next two sections. First, however, it is worth emphasizing the modernity of this intellectual configuration. Experts on language have not always claimed that grammatical knowledge encompassed human language in an exhaustive, scientific way. Eighteenth-century English grammarians, for instance, quite frequently acknowledged that at least some aspects of the grammatical structure of English simply could not be summed up in neat rules. In such cases it is not the ``head'' -- the seat of rationally statable, rule-based grammatical knowledge -- which is the judge of what the correct construction is, but the ``ear'', which ``will overrule judgement and theory'', as the English grammarian Anselm Bayly put it in his \citeyear{Bayly1772} \emph{Plain and complete grammar of the English Language} \citep[61]{Bayly1772}.\footnote{See \citet[26, 44]{Bayly1772} for some illustrative passages.} As well as the rational principles governing language structure, then, Bayly recognized the influence of a whole domain of different ones, connected with aesthetic or perhaps stylistic, rather than strictly rational, principles, and reinforced by usage. These mark out a territory into which grammar is represented as unable to venture, and which escapes from the possibility of description by objective rules. Bayly's vision is characteristic of the period: language is hybrid in nature, largely constituted by rational, orderly principles which can be described and submitted to conscious regulation, while at the same time containing aspects which evade the grip of rule-based formalization, and which are a matter of ``taste and judgement'', as William Cobbett expressed the point \citep[56]{Cobbett19831818}.\footnote{The ear, then, occasionally trumps the head. But this principle of the sovereignty of the ear was applied only selectively, to those cases where the grammarian could not devise any rules to neatly describe the particular aspect of grammar in question. Where such rules could be invented, no amount of appeal to the ``natural demand'' of the ear would be countenanced. For Lowth, for instance, even though ordinary language use is sanctioned by the ear, this in itself gives it no grammatical warrant; as he explains \citep[9]{Lowth1762}, English is very often spoken inaccurately, no matter how good it sounds to the ear. The contradiction of allowing that some aspects of language could be left to the discernment of the ear, but that in other apparently similar cases the ear had to be ignored, seems not to have been noticed.} Far from a formal theory being able to account for the entirety of discourse, grammarians acknowledged that there were some regions into which their expertise could not penetrate.

Bayly and Cobbett's present-day successors do not have the luxury of being able simply to declare some aspects of grammatical organization off-limits. The centralizing and universalizing intellectual dynamic of formal linguistics has the goal of reducing \emph{all} of a community's language practices to a single structure, a grammar -- and often then to claim that a single set of theoretical categories is capable of accounting for \emph{all} languages (universal grammar, ``the basic blueprint that all languages follow'', as it is put by a well-known linguistics textbook, \citealt[18]{Fromkinetal2010}). This theoretical process frames the structure of language and languages, including the ``structure'' of meaning, as a unique and determinate object open to empirical methods of discovery, aspiring to the imagined epistemology of the natural sciences (see \citealt{Zwicky1973} for a striking example).\footnote{I am grateful to Geoff Pullum for this reference.}

\section{Western ethnocentrism}
\label{sec:riemer:westernethnocentrim}

The first of the two ideological effects of linguistics pedagogy that we will discuss lies in the implications of the metalanguage in which this kind of analysis is conducted. It is no doubt in semantics -- a domain presupposed by a great deal of linguistic description and theory -- that the relevant effects can be most clearly observed. Semantics depends on the proposition that the linguist's native language is an adequate medium for the representation of meaning cross-linguistically. If, like most linguistic semanticists, I hold a mentalist theory of meaning, then I am justified in using a minimally enriched version of my own native language to reveal what others have in mind when they speak, regardless of what language they happen to be using. Semantic theory, as expressed in English, reveals both the content of others' semantic representations, and the conceptual structures on which this content rests.

The very tool of cross-linguistic semantic research -- a Western metalanguage, usually English -- therefore participates in what \citet[109]{AnchimbeJanney2017} have called ``the ad hoc transformation of the West's emic research perspectives into the prescribed etic standards for the rest of the world''. This entails some uncomfortable consequences: even if ``exotic'' languages are configured differently from the researcher's metalanguage, they can nevertheless, at base, be ``contained'' in the latter. The point is not limited to semantics: in all domains of grammar, the mainly Western languages which serve as metalanguages for comparative research do not assume the status of languages like any other, into which ``exotic'' languages can be translated in necessarily approximate, rudimentary, contextually variable ways: they are, on the contrary, master-codes in which fixed, context-independent, explanatory representations of exotic meaning can be definitively supplied. The universe of meaning in non-Western languages turns out to be completely ``legible'' or ``decipherable'' in Western metalanguages.

In a discussion of informant-training in his once well-known \citeyear{Samarin1967} handbook on linguistic fieldwork, William \citet[41]{Samarin1967} states that ``the ultimate goal is to get the informant to think about language as the investigator does [and to answer questions in] the way he should respond''. Such a frank admission that the goal of fieldwork is to substitute Western metalinguistic categories for indigenous ones, even in the consciousness of the informant, would not, of course, be easily avowable today. Nevertheless, contemporary glossing practices have exactly the same effect, as though Samarin's instructions to field-workers -- to aim for the native informant to wholly assimilate the linguist's metalinguistic categories -- were still in full effect. The most visible semantic theories, such as cognitive semantics or Wierzbicka's Natural Semantic Metalanguage \citep[see][]{Wierzbicka1996} give the impression that English, in which research in these frameworks is mostly presented, is not just spoken in every airport and hotel in the world, but in every head as well. The uncomfortable conclusion is that this arrangement is, in short, a striking example of Christine Delphy's (\citeyear{Delphy2008}: 31) definition of racism: the idea that ``the characteristics of the dominant are not seen as specific characteristics but as the […] normal way of being''  — normal, in the sense that it is the vocabulary of dominant languages which provides the universally valid metalanguage in which the significations of any language can be represented.\footnote{``Les caractéristiques des dominants ne sont pas vues comme des caractéristiques spécifiques mais comme la façon d'être […] normale.''}

Charles Taylor observes that

\begin{quotation}
We are always in danger of seeing our ways of acting and thinking as the only conceivable ones. This is exactly what ethnocentrism consists in. Understanding other societies ought to wrench us out of this; it ought to alter our self-understanding. \citep[129]{Taylor1985}
\end{quotation}

\noindent However, current semantic theories are not meant to entail any alteration to their users' self-understanding.  From the moment that one presents expressions in English as markers of invariant semantic ``primitives'', the possibility is excluded that their meanings could be changed by their analytical function. Semantic analysis is not dialectical: an object-language expression is analysed through a known, usually native-language expression, whose meaning is presumed to be fixed and settled, and which can therefore serve as a point of reference for the representation of exotic meanings.

It is one thing to state -- incontestably -- that different languages can be translated and understood for the purposes of a very wide range of practices and interactions. It is quite another to imagine that that reflects a cognitive identity in meaning, and that a single language -- most often, English -- provides an all-encompassing metalanguage capable of representing all other languages' meanings. John Lucy's (\citeyear{Lucy1997}: 333) critique of the Berlin and Kay colour typology, that it ``dictated in advance the possible meanings the terms could have since no other meanings were embodied in the [Munsell colour] samples'', can be generalized to all of semantics: the use of English as a metalanguage also dictates in advance the possible meanings of object-language terms.

Descriptive linguists who engage in fieldwork know very well, and frequently mention, how far their metalinguistic tools are provisional and inadequate to theoretical expectations, unable to account definitively for languages' structural and semantic reality. Linguistic theory, by contrast, conveys an entirely different idea. For the epistemology of theoretical linguistics, it is more or less inconceivable that our own linguistic categories might be inappropriate for the representation of foreign meanings. Of course, it is freely admitted that certain parts of the vocabulary -- words for colours, for emotional states, etc. -- wholly or partly resist metalinguistic definition, but those very parts are problematic for the semantic analysis of our own languages too.  For everything which \emph{can} be represented metasemantically, English -- the dominant language of metalinguistic analysis -- works. For metalinguistic purposes, there are no areas in which English turns out to be less adequate than others. Difference is abolished, with the English-speaking student in semantics being trained in an analytical technique that rests on the presupposition that \emph{their} language and \emph{their} meanings are, in a sense, the only ones that exist, since they can serve as a universal metalanguage for the representation of foreign or exotic meanings. There is a significant symbolic violence in this position: not only does the world \emph{speak} English, it thinks in it too.

\section{Theoretical domination?}
\label{sec:riemer:theoreticaldomination}

Anglophone ethnocentrism is not the only ideological value reinforced by undergraduate linguistics education. The second ideological consequence of linguistics pedagogy derives from the reductive intellectual dynamic of the unique form hypothesis itself and the claims to scientificity that accompany it. The field of theoretical competition over language in undergraduate linguistics education, I suggest, inducts students into practices which will be reengaged when, after graduation, they enter the labour market and come to participate fully in the competition of differing social interests that that entails. This induction operates on two levels. Fields of linguistics (phonology, morphology, syntax, historical linguistics, and much of semantics and pragmatics) in which instruction is mainly based around problem sets and concrete analysis model the norms of orderly, rule-governed and dispassionate decision-making essential to the ideology of contemporary technocratic administration. This much is more or less explicitly admitted in the marketing many linguistics departments regularly undertake. The spirit of disciplined, hierarchical reasoning characteristic of formal linguistics recalls Horkheimer's (\citeyear{Horkheimer19921947}: 22) critique of the reduction of language in modernity ``to just another tool in the gigantic apparatus of production in modern society''. It is, consistently, also strongly reminiscent of Max Weber's principles of bureaucracy \citep[329--341]{Weber1947}, which I present here in a selective and summarized form \citep[177--178]{Blackburn1967}:

\begin{enumerate}
    \item All official actions are bound by rules with the official subject to strict and systematic control from above.
    \item Each functionary has a limited and defined sphere of competence.
    \item The organization of offices follows a principle of hierarchy with each lower one subordinate to each higher one.
\end{enumerate}

\noindent In drumming a procedural, rule-based approach to complexity into students, linguistics education trains them in the habits of streamlined rational organisation well suited to the demands of administrative work in many domains.

The second way in which the reductive training of scientific linguistics operates ideologically is through a tension created by the unique form hypothesis itself: the clash between the search for a single, definitive representation of language structure, and the fact that multiple analyses of any theoretical problem can always be envisaged (this is, of course, just one instance of the more general underdetermination of theory by evidence in empirical enquiry). Theoretical linguistic analyses are perspectives on or interpretations of languages. Any grammatical analysis depends on a multitude of little decisions about how a chaos of variable performance data is to be idealized and normalized in order to be turned into the imaginary constructs of ``language'' and ``grammar''. As acknowledged by \citet[147]{Hockett1958} in the passage quoted by Kaplan (this volume, p.~\pageref{kaplan:hockettquote}), these depend on creative decisions informed by a myriad of considerations on which opinions can legitimately differ. In this situation, it demands significant intellectual determination to elevate contingent and hermeneutic answers to these questions into unique, ``scientific'' and definitive analyses. In fields like semantics and pragmatics, where the role of the investigator's subjective, discretionary judgement is determinant in arriving at a definitive theoretical analysis, analytical indeterminacy is overwhelming, though typically not fully acknowledged, with claims of the empirical authority and uniqueness of the researcher's preferred analysis remaining largely unqualified by any recognition of theoretical pluralism. When the availability of more than one theoretical solution is acknowledged, it is typically resolved by appeal to values of ``parsimoniousness'', ``elegance'', or ``explanatory'' capacity. The claim of any one analysis to empirical or scientific authority therefore depends on a hermeneutic -- subjective, discretionary, aesthetic -- judgement \emph{par excellence}, the judgement that solution \emph{x} is ``simpler'', more ``elegant'', ``economic'', or ``explanatory'' than solution \emph{y}.\footnote{From this perspective, it is striking to note that \citet[159]{Ludlow2011} denies that ``there is a genuine notion of simplicity apart from the notion of `simple for us to use'\thinspace'': simplicity, he says, ``is in the eye of the theorist'' \citep[161]{Ludlow2011}, and varies from one research community to the next, and over time.} \citet[105]{Althusser2015} refers to the formalist or taxonomic tradition in idealist philosophy as the ``mania for domination through categorisation''. Exactly such a mania characterizes linguistics in its pursuit of the unique form hypothesis. Students learn that the manifest diversity of possible solutions to analytical problems cannot be maintained: despite appearances, only one of the many possible analyses of a phonological, syntactic or semantic problem can be endorsed as accurate, and dominate theoretically.

Hermeneutic considerations therefore underlie claims of empirical accuracy in core linguistics subfields. The same hermeneutic foundation is evident on the higher level of framework selection. Questions of choice between theoretical frameworks (generativism versus ``West Coast'' functionalism in syntax, Relevance Theory versus more classically Gricean approaches in pragmatics, Wierz\-bic\-kian versus cognitive, or truth-functional versus definitional semantics) cannot be resolved by objective considerations, and the role of essentially discretionary and interpretative judgement in preferring one approach to another is inescapable. Yet the proponents of different frameworks rarely find the need to justify their theoretical choices in depth, and certainly do not engage in detailed theory comparison, but still enjoy the full force of claims of empirical uniqueness. The student of a Chomskyan will benefit from demonstrations of the ``scientific'' or ``empirical'' accuracy of generativism and of the mistakenness of alternative paradigms like Cognitive Linguistics. Cognitive linguists, in turn, claim a scientific authority for their own, different analyses. And so on: despite the courtesy and collegial respect evident in the majority of linguistics departments that I have observed, each ``academic lobby'' \citep[155]{Rastier1993} presents its own approach to language as the correct, and most often, as the only really legitimate one, despite the self-evident fact that it is only ever one among a number of theoretical alternatives, with choice between them being established on essentially discretionary grounds.\footnote{See McElvenny's discussion (this volume, p. \pageref{q:mcelvenny:domineering}) of Boas' ``domineering role in the world of Americanist anthropology, freely blocking the work of researchers who did not meet his frequently quite arbitrary standards''.} Bourdieu's (\citeyear{Bourdieu20031997}: 44) reference, in the context of philosophy, to ``the contradiction […] which arises from the existence of a plurality of philosophical visions, each claiming exclusive access to a truth which they claim to be single'' carries over to linguistics perfectly.

Undergraduate textbooks, accordingly, most often shelter their preferred theory and methodology from the threat of alternative perspectives through claims of the disciplinary longevity, influence, or institutional entrenchment of the favoured approach: ``a longstanding and influential view about language'', states \citet[6]{Kearns2008} at the start of her introduction to truth-functional, formal semantics, ``is that the meaningfulness of language amounts to its `aboutness'\thinspace''. Adger's (\citeyear{Adger2003}: 14) stipulation of theoretical assumptions at the start of \emph{Core Syntax} is similarly discretionary: ``[t]he approach to syntax that we will take in this book, and which is taken more generally in generative grammar, assumes that certain aspects of human psychology are similar to phenomena of the natural world, and that linguistic structure is one of those aspects''; challenges to this assumption are not even mentioned. Frameworks are taught not so much because they and their assumptions are ``right'', though this is certainly implied, but because they are more ``influential'' or ``general'' than their competitors. No greater accountability from teachers for their choices of theoretical perspective is expected. \citet[9]{Lawson2001} notes that ``[t]he rhetoric of dismissing a theory as `uninteresting' […] seems to be one of the stock-in-trade notions of introductory, as well as advanced, texts in linguistics''. Twentieth-century textbooks of linguistics, he notes, ``present the historically most-highly contested elements of their theories simply as fact'', and ``the most tenuous and problematic premises of a linguistic theory have tended to be presented to the reader of introductory linguistics texts as a natural assumption, true by definition or out of common sense'' \citep[12]{Lawson2001}.

Whether within frameworks or between them, it is the institutional authority held by an academic in the classroom setting that allows the arbitrariness of these theoretical choices to be obscured, the existence of analytical indeterminacy and competing theoretical frameworks to be rejected in an essentially voluntaristic way, and the threat these pose to the academic's own theoretical preferences and hence authority to be obviated. Students quickly learn that linguistic experts can claim authoritative ``scientific'' or ``empirical'' uniqueness for their preferred theoretical framework, even in the absence of disciplinary consensus.\footnote{The discretionary authority detained by the academic is, perhaps, nowhere in greater evidence than in the grammaticality assignments on which syntactic theorizing rests. There can be no rules to determine whether a sentence is grammatical: the native speaker's intuition is the only judge, and one of the most commented-on features of syntax classes is the regularity of disagreements. These inevitable disputes are a prime arena for the imposition of the linguist's own preferences: for the purposes of a syntax class, a sentence is grammatical if the lecturer says it is. The discretionary authority exerted by the linguist in stipulations about grammaticality is a microcosm of the authority they detain more generally.}

This exercise of discretionary theoretical power, we might speculate, constitutes the most important ideological consequence of the unique form hypothesis as an educational practice in linguistics. The spectacle of theoretical justification to which students are exposed in their linguistics training habituates them to a certain acceptance of arbitrary symbolic authority -- their lecturer's -- which will be rapidly reactivated outside the university in the figure of their employer, landlord or political ``representative''. This authority is at its most obvious when students sit examinations or submit work to be marked: here, the extent to which academic success is a function of their lecturer's discretionary judgement is clear. In submitting to their lecturer's theoretical authority over the thoroughly material stakes of their academic results, students reinforce dispositions that will be reengaged in the far more coercive world of labour-market exploitation which they will soon fully (try to) join.

For academics, too, the stakes of theoretical competition are not just immaterial or intellectual, confined to a world immune from any extra-disciplinary considerations. Theories are also the instruments of careers, and enable the acquisition and exercise of institutional power and professional advancement. Theoretical pluralism and the evaluative equivalence it suggests between different frameworks sits uneasily in a rigidly hierarchical institutional context like that of the university. In such a world, theoretical competition is natural. Only if the unique form hypothesis is in place can intellectual competition for the best theory of language be aligned with material competition for professional rewards.

I have suggested, then, that linguistics education ends up prefiguring the conflict of interests in society. It does not do this, however, by conveying any explicit theoretical content: asserting that would be precisely the ``magical'' or ``allegorical'' view of ideology criticized by Baudrillard. Rather, the ideological import of linguistics education should be located in its pedagogical \emph{processes} and \emph{forms} of transmission. In studying linguistics students learn to submit to -- and to assume -- a certain way of exercising arbitrary symbolic power in the domain of theory, by gradually accepting the scientistic pretensions of a basically discretionary, subjective institutional practice. Students studying linguistics are encouraged to develop generalizations and theories about linguistic aspects of the human world in a highly reductive and abstract way, subject to fairly lax empirical controls. The verification procedures they are trained to employ rarely go beyond the idealized and hence hypothetical representations under study, and are strongly conditioned by their lecturer's interpretative preferences. By validating their own theoretical preferences in the context of the unique form hypothesis and by effectively sheltering them from serious contestation, academics model for students the way that claims of scientificity, reason and empirical responsibility can be deployed to legitimate individual sovereign interests.

The arbitrariness of justification in the theoretical order, embodied in the regimes of authority of the university, comes therefore to correspond to the arbitrariness of the material and political order outside it. In giving students, at an important stage of their intellectual development, and at the very moment when they are on the point of entering the full-time labour-market, the authorization to claim scientific status, in the context of the unique form hypothesis, for what remain essentially discretionary and unoperationalized interpretations, linguistics education, whatever its other effects, normalizes the unjustifiable exercise of power.

The fact that the ideological properties of linguistics are rooted in the wider institutional context of higher education means that commonalities between linguistics and other ``human sciences'' should exist. This is, indeed, the case: habituation to the arbitrariness of intellectual power is arguably a hallmark of education in the humanities in general (see \citealt{Riemer2016} for some preliminary discussion). It is commonplace to insist on the capacity of the humanities to foster students' critical capacities, but the complement of this process is a risk that is often ignored -- the possibility that humanities disciplines, linguistics included, end up habituating students to different kinds of arbitrary symbolic domination, forerunners of the very real forms of domination to which they will soon have to reconcile themselves as job seekers amid the madness of capitalist labour markets, or that they will themselves exert as members of the comparatively privileged Western middle classes \citep[see][]{Pinsker2015}.

\section{Education and linguistic ‘science’ in a post-truth world}
\label{sec:riemer:education}

If linguistics was a natural or ``hard'' science -- if, that is, theoretical activity was governed by protocols generally accepted throughout the discipline, thereby producing objective and agreed-on results -- we would be wholly justified in accepting every theoretical linguistic result, regardless of its apparent ideological tenor. This is not, however, our situation. As we have noted, there is no single theory accepted discipline-wide: linguists do not even agree on how to define the object they study. Unlike the sciences of nature, linguistics, as a human ``science'', concerns the behaviour of autonomous creatures endowed with their own ways of existing and understanding the world. Given this, it is not self-evident that theoretical understanding is obtained through an objectifying and reductive analytical procedure, assimilating grammar and meaning to a determinate object able to be studied using the empirical techniques of the natural sciences, rather than through a pluralistic process of interpretation, drawing the study of language closer to that of other socio-cultural performances. Anthropology, literary history and sociology are all empirical disciplines which propose explanations, not just descriptions, of the objects they study. But they do not have the ambition of producing reductive and singular analyses of their objects. As far as linguistics is concerned, it is no more obvious that it should advance unique analyses of grammatical and semantic ``structure'' than it is that literary historians should converge on a unique interpretation of a canonical text.

Linguistic analyses of grammar and meaning intrinsically entail conclusions about the conceptual competencies of speakers and the cultural resources of communities. Sidelining the entire interpretative dimension of linguistics, ignoring the multiplicity of analyses that is always possible, claiming to discover a unique conceptual form underlying speech -- this is, as we have seen, what a large part of linguistics education involves. At a time when racist and other identitarian forms of discrimination are strongly on the rise, when many political actors seek to caricature the psychology of entire civilizations and social categories in ways whose reactionary intentions are only too clear, and in the ``post-truth'' era when the results of scientific research are routinely threatened by pseudo-sciences in the pockets of influential political lobbies, linguists have a responsibility not to insist on the necessity or scientific credentials of their fundamentally wholly hermeneutic analyses, if we do not wish to reinforce the abuses of science and expertise characteristic of our age.

Just as it is important to validate ``minor'' languages, a challenge which linguists often take up explicitly, minor \emph{linguistics} should be validated too. To do so is natural, given a basically hermeneutic understanding of what the discipline is. In this chapter, I have described some of the ideological factors that entrench the unique form hypothesis in linguistics, especially in its ramifications in undergraduate education, and which obscure the ample reasons to call it into question. The purely speculative, strictly non-``scientific'' nature of this ``political epistemology'' of the unique form hypothesis might strike readers from the mainstream of linguistics as problematic. Such a reaction would be mistaken. A discipline's development does not involve just the collection, analysis and theorization of data, but should also consist in collective reflection on the various aims and effects of those practices. This reflection must not allow itself to be diverted into a purely ``academic'' and abstract investigation of the sociology of linguistic theory, valuable though that would be in its own right. Linguists are not sociologists, and we do not have to be in order to undertake metatheoretical reflection on the possible social meaning of our practices. In a world disfigured by the ecological, economic and political violence of the neoliberal capitalist order, the value of theoretical understanding and education derives from the contribution they make to harnessing reason for the progress of society. It is therefore incumbent on those of us responsible for the creation and transmission of knowledge to interrogate our own practices in order to assess how far they facilitate or obstruct this goal. As participants in the education of the next generation of workers, unemployed, exploiters and voters, it is difficult to reflect too deeply on our discipline's possible social effects.

This vision entails no dogmatism, and certainly does not threaten, as one might be tempted to think, to coercively subordinate linguistics to any particular political program. On the contrary, it allows us to conceive of the discipline as a site of a pluralistic and reflexive exchange, and justifies a blossoming of different theoretical frameworks and approaches. As a disciplinary practice, that is, as a matter of fact, what linguistics often already is. That the discipline's conventional epistemology can only analyse this as theoretical \emph{competition} is a fact that surely sits uneasily with the solidarity that should be at the origin of intellectual progress, whether in theory or education.

\sloppy
\printbibliography[heading=subbibliography,notkeyword=this] 
\end{document}
