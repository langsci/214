\documentclass[output=paper]{langscibook}
\author{Els Elffers}
\title{Linguistics as a ``special science''. A comparison of Sapir and Fodor}
\label{chap:elffers}

\abstract{Independently of each other, the linguist-anthropologist Edward Sapir (1884–1939) and the philosopher of mind Jerry Fodor (1935–2017) developed a similar typology of scientific disciplines. ``Basic'' (Fodor) or ``conceptual'' (Sapir) sciences (e.g. physics) are distinguished from ``special'' (Fodor) or ``historical'' (Sapir) sciences (e.g. linguistics). Ontologically, the latter sciences are reducible to the former, but they keep their autonomy as intellectual enterprises, because their ``natural kinds'' are unlike those of the basic sciences. Fodor labelled this view ``token physicalism''. Although Sapir's and Fodor's ideas were presented in very different periods of intellectual history (in \citeyear{Sapir1917} and \citeyear{Fodor1974}) and in very different intellectual contexts (roughly: \emph{Geisteswissenschaften} and logical positivism), the similarity between them is striking. When compared in detail, some substantial differences can also be observed, which are mainly related to contextual differences. When applied to linguistics, Sapir's and Fodor's views offer a perspective of autonomy, albeit it in different ways: for Fodor, but not for Sapir, linguistics is a subfield of psychology.}
 
\begin{document}
\maketitle

\section{Introduction} 
\label{sec:elffers:intro}
In \citeyear{Fodor1974}, Jerry Fodor (1935-2017) introduced ``token physicalism'', a non-reductive variety of physicalism, which applies to ``special sciences''.\footnote{Token physicalism belongs to a larger class of non-reductive types of physicalism. Supervenience physicalism and emergentism are other members. John Stuart Mill (1806–1873) is generally regarded as an early representative of non-reductive physicalism.} According to Fodor, special sciences, such as economics, psychology and linguistics cannot be entirely reduced to physics, which is a ``basic science''. Such a reduction would imply that special sciences actually disappear as autonomous sciences. 

According to Fodor, special sciences retain their autonomy, because reduction is possible only with respect to the events they describe (``tokens''), not with respect to properties or natural kinds (``types''). For example, economic events such as monetary exchanges are, ultimately, physical events, but, from a physical point of view, very heterogeneous ones. There is no single physical natural kind corresponding to the economic natural kind ``monetary exchange'', because such exchanges may involve ``strings of wampum, […] dollar bills, […or] signing one's name to a check'' \citep[103]{Fodor1974}. Physics, according to Fodor the only ``basic science'', develops taxonomies of physical phenomena in terms of physical properties. Special sciences develop their own taxonomies of, ultimately, physical phenomena as well, but in other terms, not belonging to the vocabulary of physics.

In this chapter, I will compare Fodor's token physicalism with ideas of Edward Sapir (1889–1939), presented in an article published in \citeyear{Sapir1917}. I will argue that Sapir's ideas are highly similar to Fodor's. Despite differences, Sapir's ``conceptual sciences'' and ``historical sciences'' resemble Fodor's basic and special sciences to such a degree that, in this respect, Sapir can be regarded as Fodor's predecessor.\label{q:elffers:probsols}

Historians, including historians of linguistics, apply the concept ``predecessorship'' in different and partially unfounded ways. In section \ref{sec:elffers:pitfalls}, I will briefly discuss this problem and present my own view of predecessorship, including its implications for the concept ``predecessorship of token physicalism''.

In the sections that follow, I will argue that this concept applies to Sapir. Section \ref{sec:elffers:superorganic} will discuss Sapir's distinction of conceptual and historical sciences in detail. In section \ref{sec:elffers:tokenphysicalism}, Fodor's token physicalism is further analysed. Together, these sections present a picture of similar theories, developed in different periods, intellectual contexts, and with different motivations. Sections \ref{sec:elffers:similaritiesdiffs}–\ref{sec:elffers:characterizing} present a systematic comparison of both theories. In sections \ref{sec:elffers:lawsexceptions}–\ref{sec:elffers:clues}, both theories will be discussed in a broader context, both chronologically and intellectually. 

The views of both Sapir and Fodor were presented without any special focus on linguistics. In linguistic circles, their views are not well known. In section \ref{sec:elffers:linguisticsasspecialscience}, I will explore the linguistic implications of Sapir's and Fodor's varieties of token physicalism.

\section{Pitfalls of predecessorship}
\label{sec:elffers:pitfalls}

``Predecessorship'' belongs, together with some other concepts (e.g. ``influence'' or ``source''), to the more dangerous instruments of the historian's toolbox. They are applied in multifarious and sometimes confusing ways. Present-day dangers of ``predecessorship'' can be partially attributed to the belated influences of older approaches to intellectual historiography:

\begin{enumerate}
    \item The exegetical ``history of ideas'' approach, with its focus on isolated and quasi-immutable ``ideas'' or ``themes'', and their march through history.
    \item The historicist approach (in one of many meanings of this term)\footnote{The meaning of ``historicism'' applied here is related to the meaning of other ``-ism'' terms such as ``psychologism'' or ``scientism''; these terms claim to reveal ``where'' the essence of things has to be looked for. This meaning of ``historicism'' has to be distinguished from, e.g., Popper's use of the term \citep[cf.][43]{Elffers1991}.} of interpreting chronological sequences of events in causal, teleological or developmental terms.
\end{enumerate}

Unwarranted claims of predecessorship are corollaries of (1) and (2).\footnote{Cf. \citet[chaps. 2 \& 3]{Elffers1991} for a more thorough and comprehensive discussion of these influences in present-day intellectual historiography, and for more details of the alternative approach, briefly indicated on p. \pageref{q:elffers:probsols} of this chapter as reconstruction of earlier scientific ideas ``as problem solutions in the context of the contemporary intellectual state-of-the-art''.} I mention an example of both, to be found in historiography of linguistics:

\begin{enumerate}

\item[Ad 1.] In \citet{Antal1984}, the entire history of linguistics is interpreted in terms of an alternation of  two themes: ``psychologism'' and ``objectivism''. Hermann Paul (1846–1921) is thus presented as a ``psychologistic'' predecessor of Noam Chomsky (b. 1928). The term ``psychologism'' applied to approaches as far apart as those of Paul and Chomsky, is, however, almost meaningless, and so is the predecessorship conclusion based upon it.

\item[Ad 2.] In Chomsky’s (\citeyear{Chomsky20091966}) well-known \emph{Cartesian Linguistics}, the seventeenth-century Port-Royal Grammar is presented as a – still imperfect – predecessor of twentieth-century generative grammar. This claim has been amply criticized as being based upon an incorrect and biased interpretation of seven\-teenth-century grammar, and as a specimen of presentism, Whig history and ancestor hunt. All these defects are rooted in the historicist idea of chronology as a series of developmental steps towards the present. 

\end{enumerate}

Although pitfalls (1) and (2) are well known today, the danger of unwarranted claims of predecessorship still exist. It is natural for historians to compare phenomena over time. Discovering similarities easily creates ``the temptation to discern and extract pervasive themes or patterns running through and manifested in the succession of events and activities'' \citep[7-8]{Robins1997}. Moreover, historicism (conceived in the above manner) is still influential in the way it permeates our common historical vocabulary, which ``presents history as a `stream' which proceeds irresistibly […]. Metaphors talking of `progress' […] constitute examples: `avant-garde' art, advanced technique […], locutions like `keeping pace with' […] or `being in advance of' one's time as well as clock-metaphors, such as `turning back' or `stopping' the clock […]'' (\citealt[131]{Dussen1986}, transl. E.~E.)\footnote{``…waarbij de geschiedenis wordt voorgesteld als een `stroom', die onweerstaanbaar […] voortgaat. Metaforen waarin over een `vooruitgang' wordt gesproken […] zijn hier voorbeelden van: `avant-garde' kunst, geavanceerde techniek […] het spreken van een `meegaan' met de tijd […] of zijn tijd `vooruit' zijn, evenals klok-metaforen, zoals de klok `terugdraaien' of `stilzetten' […]''}

If the above pitfalls are avoided, the establishment of predecessorship relations in intellectual history may become a more complicated task, but it continues to be interesting and rewarding; indeed, even more so, because we are now disregarding superficial historical similarities as well as irrelevant later developments. Instead, we are thoroughly analysing and comparing the actual contents of scientific ideas, which are carefully reconstructed as problem-solutions within the context of the contemporary intellectual state of the art.

Following this approach, I assume that a predecessor of token physicalism was similar to Fodor with respect to the questions that Fodor answered by postulating token physicalism, and to the answers themselves.

Questions – predecessors of token physicalism are involved in:

\begin{enumerate}
    \item[a.] ontological questions concerning basic categories of entities,
    \item[b.] epistemological questions about the basic categories of separate disciplines.
\end{enumerate}

Answers: Predecessors of token physicalism present theories which take into account questions (a) and (b) and assume that for one or more ``basic'' disciplines, the categories of (a)-answers and (b)-answers are identical. For other, ``special'', disciplines, the categories of (a)-answers and (b)-answers are non-identical.\footnote{Against this background, I regard Seuren's (\citeyear[827-832]{Seuren2016}) claim that a scholar much earlier than Sapir, Hyppolyte Taine (1828–1893), anticipated Fodor's token physicalism as unconvincing. Seuren presents quotations to support his view, but none of them suggests a distinction comparable to the distinction between (a) and (b).}

Predecessorship thus conceived is typically unconstrained by terminology. Terminological identity may conceal fundamental differences in content, and vice versa. Consequently, predecessors of token physicalism may apply quite different terms from those used in the above preliminary assumption. The only requirement is that the content of their statements can be interpreted in terms of this assumption. This also applies to Fodor himself: \citet{Fodor1974} does not use the term ``ontological'' at all, and he sometimes gives the term ``epistemological'' a rather specific meaning.\footnote{``Epistemological'' as used in \citet[113]{Fodor1974} refers to the ``context of discovery''. This deviates from the usual reference, which is primarily to the ``context of justification''.\label{fn:elffers:6}} But the content of his statements meets the requirement of being interpretable in the above terms, as I hope to show below.

\section{Sapir: against a ``superorganic''}
\label{sec:elffers:superorganic}

Sapir's \citeyear{Sapir1917} article is titled ``Do we need a `Superorganic'?'' It is a reaction to the anthropologist Alfred Kroeber's (1876–1960) article ``The Superorganic'' \citep{Kroeber1917}. Both articles were published in subsequent issues of \emph{American Anthropologist}, an anthropological journal that is still quite prominent in the field.\footnote{Sapir's article appeared in the section ``Discussion and correspondence''. Another comment on Kroeber's article by A. A. \citet{Goldenweiser1917} was included in the same section.}

Both Kroeber and Sapir were students of Franz Boas (1858–1942), the ``founding father'' of American anthropology. Kroeber, who became an influential American anthropologist, argues in his \citeyear{Kroeber1917} article against the reduction of anthropology to biology. He states that human cultural behaviour, unlike animal behaviour, cannot be explained through an appeal to inheritance plus Darwinian adaptation, nor to personal psychology. The forces of culture, a superorganic and autonomously developing entity, are the main determinants. For anthropology, this superorganic is the actual object of research.\footnote{Herbert Spencer (1820–1903) coined the term ``super-organic'' to focus on social organization, in the first chapter of his \citeyear{Spencer1898} \emph{Principles of Sociology}, entitled ``Super-organic Evolution''.}

When Sapir wrote his critical article, he was working as director of the Anthropological Division of the Geological Survey of Canada in Ottawa. This was a very productive period in his career. Anthropological linguistics, which included the investigation and description of American Indian languages never studied by academics before was his main area of research. He exchanged correspondence with Kroeber over a period of many years.

In ``Do we need a `Superorganic'?'', Sapir begins by welcoming Kroeber's ``salutary antidote'' to the trend of applying methods used by the exact sciences to the study of culture. But he also feels that Kroeber ``has allowed himself to go further than he is warranted in going'' on ``two points of considerable theoretical importance'' \citep[441]{Sapir1917}. Although only the second point directly concerns our subject, I will also briefly discuss the first one, because there is, according to Sapir, a connection between them.

The first point concerns Kroeber's denial of any influence by individuals on the course of cultural history. Sapir admits that the influence of individuals is mostly highly exaggerated by historians. He fully recognizes that individual thought and action are very much moulded by cultural traditions, and that the cultural influence of most individuals is nil. If it is not nil, broader cultural conditions are necessary to trigger this influence. But this does not obviate the influence of at least some individuals – such as Napoleon, Jesus, Shakespeare or Beethoven – on cultural history, according to Sapir. A total social determinism goes too far.

The second point concerns the nature of social phenomena. Kroeber claims that they are built out of organic phenomena but are not reducible to organic phenomena, just as organic phenomena are built out of inorganic phenomena but are not reducible to them. A superorganic social ``force'' is assumed, which is manifested in social history.

Sapir regards the above analogy as false. The types of irreducibility are entirely dissimilar. Sapir's ontology is trialistic. He assumes three basic types of entities: inorganic, organic and psychic. Social phenomena are not a fourth type, as Kroeber feels they are, but ``merely a certain philosophically arbitrary but humanly immensely significant \emph{selection} out of the total mass of phenomena ideally resolvable into inorganic, organic and psychic processes'' (\citealt[444]{Sapir1917}, italics Sapir). Social phenomena are, therefore, not at all conceptually irresolvable but experientially irresolvable. Conceptual irresolvability is what separates inorganic, organic and psychic phenomena; these are, in Sapir's terms ``true conceptual incommensurables'' \citep[445]{Sapir1917}. Experiential irresolvability is entirely different: it refers to classes of directly experienced phenomena, demarcated not in terms of ontology, but in terms of values that determine their selection. These classes are studied in historical sciences. Conceptually demarcated classes are studied in conceptual sciences.

Sapir illustrates his distinction between types of science using the example of geology:

\begin{quotation}
    Few sciences are so clearly defined as regards scope as geology. It would ordinarily be classed as a natural science. Aside from paleontology, which we may eliminate, it does entirely without the concepts of the social, psychic or organic. It is, then, a well-defined science of purely inorganic subject matter. As such, it is conceptually resolvable, if we carry our reductions far enough, into the more fundamental sciences of physics and chemistry. But no amount of conceptual synthesis of the phenomena we call chemical or physical would, in the absence of previous experience, enable us to construct a science of geology. The science depends for its \emph{raison d'être} on a series of unique experiences, directly sensed or inferred, clustering about an entity, the earth, which from the conceptual standpoint of physics is as absurdly accidental or irrelevant as a tribe of Indians or John Smith's breakfast. The basis of the science is, then, grounded in the unique relevance of particular events. To be precise, geology looks in two directions. In so far as it occupies itself with abstract masses and forces, it is a conceptual science, for which specific instances as such are irrelevant. In so far as it deals with particular features of the earth's surface, say a particular mountain chain, and aims to reconstruct the probable history of such features, it is not a conceptual science at all. In methodology, strange as this may seem at first blush, it is actually nearer, in this respect, to the historical sciences. It is, in fact, a species of history, only the history moves entirely in the inorganic sphere. In practice, it is, of course, a mixed type of science, now primarily conceptual, now primarily descriptive of a selective chunk of reality.\label{q:elffers:sapirquote}
    \citep[445]{Sapir1917}\footnote{In this quotation, physics and chemistry are both mentioned as fundamental sciences of the inorganic. In 1917, reducibility of chemistry to physics was not at all as generally accepted as it is today \citep[cf.][13, 17-18]{Hettema2012}.} 
\end{quotation}

As examples of ``chunks of reality'' studied by historical sciences, Sapir also mentions, next to the earth, ``France, the French language, the French Republic, the romantic movement in literature, Victor Hugo, the Iroquois Indians, some specific Iroquois clan, all Iroquois clans, all American Indian clans, all clans of primitive peoples.'' \citet[446]{Sapir1917} stresses that none of these terms has any relevance in a purely conceptual world, whether organic, inorganic or psychic.

These examples are not selected arbitrarily. Sapir wants to show (i) that historical sciences apply to ``history'' in a much wider sense than the word ordinarily indicates, (ii) that historical sciences not only study directly experienced entities, but also more abstract entities.

Sapir elaborates on (ii) in order to explain two further differences between types of science: ``such concepts as a clan, a language, a priesthood'' might suggest a similarity with ``the ideal concepts of natural science'', which also ``lack individual connotation'' and appear in generalized laws. Logically, both sets of concepts are involved in similar operations such as observation, classification, inference, generalization etc. ``Philosophically'', however, the concepts are distinct, because, in actual fact, the social concepts are not ``ideal'' at all; they are ``convenient summaries of a strictly limited range of phenomena, each element of which has real value'':

\begin{quotation}
    Relatively to the concept ``clan'' a particular clan of a specific Indian tribe has undeniably value as a historical entity. Relatively to the concept ``crystal'' a particular ruby in the jeweler's shop has no relevance except by way of illustration. It has no intrinsic scientific value. Were all crystals existent at this moment suddenly disintegrated, the science of crystallography would still be valid, provided the physical and chemical forces that make possible the growth of another crop of crystals remain in the world. Were all clans now existent annihilated, it is highly debatable, to say the least, whether the science of sociology, in so far as it occupied itself with clans, would have prognostic value.
    \citep[446-447]{Sapir1917}
\end{quotation}

A corollary of this difference is the different status of laws in both types of science. A sociological law is a generalization, an abbreviation for a finite number of phenomena. Exceptions occur, and the laws become ``more and more blurred in outline with the multiplication of instances'', whereas this multiplication makes natural laws ``more and more rigid'' \citep[447]{Sapir1917}. Natural laws cover an indefinitely large number of phenomena and have to be exceptionless: an exception necessitates a new formulation of the law.

Sapir concludes his article by connecting his two criticisms of Kroeber: if the nature of historical phenomena had been sufficiently clear to him, he would have felt no need to invoke a ``superorganic'' force as a unique explanans in history, and to deny individual force.

\section{Fodor: against reductive physicalism}
\label{sec:elffers:tokenphysicalism}

Fodor's article is titled ``Special sciences (or: the disunity of science as a working hypothesis)''. It was published in \citeyear{Fodor1974} in \emph{Synthèse}, a well-known philosophical journal that is still published. It takes as its starting point the ``typical thesis of positivistic philosophy of science […] that all true theories in the special sciences should reduce to physical theories in the long run'' \citep[97]{Fodor1974}. This thesis, and its foundation in a materialist ontology, were the cornerstones of the Unity of Science movement, to which Fodor's title alludes. This movement was narrowly related to logical positivism during the first decades of the twentieth century. Since those days, questions about the unity of science and about reductivism have never disappeared from the philosophical agenda.

When Fodor wrote ``Special sciences'', he was a professor in the departments of philosophy and psychology at the Massachusetts Institute of Technology. Philosophy of mind and language was his central subject of research. He had already published widely on many themes related to this area. In \citeyear{Fodor1975}, his seminal book \emph{The language of thought} would appear. In ``Special sciences'', psychology is by far the science that receives the most attention.

Fodor addresses a problem that results from the positivistic assumption that the subject matter of a special (i.e. non-physical) science, such as psychology, is part of the subject matter of physics. A generally accepted inference from this assumption is that psychological theories must reduce to physical theories. This causes methodological problems for psychology; the discipline should actually disappear as a separate science. \citet[98]{Fodor1974} wants to ``avoid the trouble by challenging the inference''.

Assuming that sciences are about events, Fodor claims, in agreement with the physicalists, that ``all events that the sciences talk about are physical events […]'' \citep[100]{Fodor1974}. He calls this doctrine ``token physicalism''. But he rejects the stronger reductionist doctrine of ``type physicalism'', which claims that, in addition, every property mentioned in the laws of any science is a physical property. Token physicalism claims that, for example, every psychological event is identical to a neurological event, but not every psychological property is identical to a neurological property.

The reason why type physicalism is too strong a thesis is that interesting generalizations in special sciences are often about events whose physical descriptions have nothing in common. Moreover, the question ``whether the physical descriptions have anything in common is, in an obvious sense, entirely irrelevant to the truth of the generalizations, or to their interestingness, or to their degree of confirmation, or, indeed, to any of their epistemologically important properties […]'' \citep[103]{Fodor1974}. As an example of such a generalization, Fodor refers to Graham's Law, an economic law about monetary exchanges. In the above introduction, this example was already mentioned to illustrate the wildly different physical events which correspond to the concept of ``monetary exchange'' (transactions with bills, cheques etc.). These events do not correspond to a natural kind in physics. Similarly, although psychological events correspond to neurological events, ``there are no firm data for any but the grossest correspondence between types of psychological states and types of neurological states, and it is entirely possible that the nervous system of higher organisms characteristically achieves a given psychological end by a wide variety of neurological states'' \citep[105]{Fodor1974}.\footnote{Fodor refers to the physiological psychologist Karl Lashley as a defender of this claim. He also acknowledges that there is much ``psychology and brain'' research throughout the world, which is based upon the assumption that psychological types correspond to neurological types \citep[105]{Fodor1974}.\label{fn:elffers:physpsych}}

Fodor further supports his token physicalistic view by arguing that his view explains (i) that laws of special sciences have exceptions, (ii) why there are special sciences at all.

\begin{enumerate}
    \item[Ad i.] Given the assumption that, in a special science law, physical counterparts of the antecedent as well as the consequent consist of heterogeneous disjunctions, the counterpart ``law'' cannot be a genuine physical law.\footnote{This is a very brief and simplified presentation of a complex argument, presented in \citet[109]{Fodor1974}.} Exceptions occur when the physical counterpart of an instantiation of the antecedent of a special science law has no lawlike connection with one of the disjunctive physical counterparts of the consequent. According to Fodor, this is a common situation in a special science such as psychology: there are always exceptions to psychological generalizations which are  ``uninteresting from the point of view of psychological theory'' \citep[111]{Fodor1974}.

    \item[Ad ii.] According to reductionists, special sciences exist for practical, ``epistemological'' (cf. note \ref{fn:elffers:6}) reasons. If neurons were not so small and brains were on the outside of the head, we would do neurology instead of psychology. Fodor does not agree: even if brains were on the outside, we would not know what to look for, lacking ``the appropriate theoretical apparatus for the psychological taxonomy of neurological events''. Moreover, he assumes that such a corresponding taxonomy does not necessarily exist, that ``quite different neurological structures can subserve identical psychological functions […] In that case the existence of psychology depends not on the fact that neurons are so sadly small, but rather on the fact that neurology does not posit the natural kinds that psychology requires'' \citep[113]{Fodor1974}.
\end{enumerate}

Special sciences exist autonomously, because other taxonomies are required alongside the taxonomy which suits the purpose of formulating exceptionless basic physical laws. The other taxonomies are necessary for the formulation of important generalizations in areas of knowledge such as psychology or economics.

\section{Similarities and differences}
\label{sec:elffers:similaritiesdiffs}
The last two sections show two scholars struggling for a plausible philosophical reconstruction of science in general and its division into separate disciplines in particular. Independently from each other and separated by nearly six decades, they devised a nearly identical theory.\footnote{Of course, Fodor \emph{could} have read Sapir's article, but I regard this as improbable. As far as I know, Fodor never refers to Sapir. Moreover, Sapir's intellectual activities and viewpoints were unrelated to Fodor's area of interest, or even repugnant to him \citep[cf.][]{Pullum2017}.} According to this theory, boundaries between disciplines are not merely determined by the kind of stuff they investigate. Although some (``conceptual'' or ``basic'') sciences can be demarcated along these lines, other (``historical'' or ``special'') sciences are demarcated in a different way. Their object of investigation consists of heterogeneous stuff, but is homogeneous by its relevance to the purposes of the area of knowledge to which they belong.

For Sapir, the theory was a welcome alternative to Kroeber's ontological way of rescuing the autonomy of sociology and anthropology through the assumption of a superorganic force. For Fodor, the theory was a welcome alternative to reductive physicalism, with its problematic methodological requirements, especially for psychology.

Due to these different backgrounds, the theories have a different ``appearance''. In Fodor's discourse, subtle logical properties of scientific theories are taken into account, as was (and is) usual in positivistic-oriented philosophy of science. In Sapir's and Kroeber's discourse, this approach is entirely absent, also in conformity with what was (and is) usual in philosophy of non-exact sciences.

In the following sections, Sapir's and Fodor's theories will be compared in more detail. Their common basic idea is elaborated in partially different ways by both scholars. Part of these differences can be shown to be related to the intellectual context in which the theories were developed.

In the rest of this article, I will use Fodor's term ``token physicalism'' to refer to the common view of Sapir and Fodor.\footnote{The literal meaning of the term has to be bracketed in Sapir's case, because of his trialistic ontology.} In the same vein, I will adopt Fodor's terms ``basic science'' and ``special science'' for the similar types of sciences distinguished by both scholars.

My comparison is almost entirely based upon the articles just discussed. Neither Sapir nor Fodor elaborated their theory further in later publications. Fodor, however, returned to the subject in his article ``Special sciences: still autonomous after all these years'', published in \citeyear{Fodor1997}. This article consists of a defence of his view against the criticism of \citet{Kim1992}. In the course of this defence, some aspects of token physicalism are presented in more detail than before. An addition, which is relevant to our comparison with Sapir, is that special sciences are now explicitly described in functionalistic terms. Their physically heterogeneous natural kinds are functionally homogeneous, in the same way as physically heterogeneous types of artefacts (can openers, mousetraps) are functionally homogeneous \citet[160]{Fodor1997}. This characterization was lacking in the \citeyear{Fodor1974} article, although ``psychological functions'' are mentioned. The term ``functional'' must be interpreted in a very broad sense, because it is equally applied to biology, psychology and geology. The last mentioned example of a special science is a new one, and identical to Sapir's example. Like Sapir, \citet[160]{Fodor1997} claims that mountains are made ``of all sorts of stuff'', but that ``generalizations about mountains-as-such […] serve geology in good stead''.

Taking into account the \citeyear{Fodor1997} additions to Fodor's theory, the views of Sapir and Fodor, as presented in section \ref{sec:elffers:superorganic} and \ref{sec:elffers:tokenphysicalism} can be schematically juxtaposed as in \tabelref{tab:elffers:sciences}.

\begin{table}
\label{tab:elffers:sciences}
\begin{tabular}{p{1.8cm} p{2.2cm} p{2.2cm} p{2.2cm} p{2.2cm}}
  \multirow{2}{*}{ } & \multicolumn{2}{c}{Basic sciences} & \multicolumn{2}{c}{Special sciences} \\
  & \multicolumn{1}{c}{\emph{Sapir}} & \multicolumn{1}{c}{\emph{Fodor}} & \multicolumn{1}{c}{\emph{Sapir}} & \multicolumn{1}{c}{\emph{Fodor}} \\ \hline
  & Physics, & Physics & Sociology, & Psychology, \\
  & Chemistry, & & Anthropology, & Linguistics, \\
  \emph{Sciences} & Geology, & & Linguistics, & Biology, \\
  & Biology, & & Geology, & Geology  \\
  & Psychology & & {(}Cultural{)} History & \\ \hline
 
 \emph{Demarcation} & Ontological & Ontological & Experiential & Functional \\ \hline
 
 \emph{Exceptions of laws?} & No & No & Yes & Yes \\
\end{tabular}
\end{table}

\tabelref{tab:elffers:sciences} shows that Sapir's and Fodor's varieties of token physicalism are different at two points: (i) their selection of basic and special sciences, (ii) their characterization of special sciences. As to (i), we may ask how far the differences can be related to contemporary ontological assumptions. As to (ii), we may ask how far apart the standpoints actually are, given the similarity of both scholars' general view of the special sciences. Likewise, we may ask how far their agreement about the issue of exceptions to laws actually goes, given the different motivations of these ideas, observed earlier. I will discuss these three issues in separate sections.

\section{Which basic and special sciences?}
\label{sec:elffers:basicspecial}

Fodor recognizes one basic science, physics, which is in conformity with the positivistic discourse he connects with. In the same vein, he also mentions chemistry as a science that has been successfully reduced to physics.

His most important example of a special science is psychology. The anti-reductionist defence of the autonomy of this science is his central aim, and directly relevant to his work as a cognitive psychologist. In his seminal book \emph{The language of thought} (\citeyear{Fodor1975}), the text of ``Special sciences'' is included in the introductory chapter, which presents the foundations of the psychological and linguistic approach described and applied in the rest of the book.\footnote{There are some minor differences between the article and the book section. The book section contains more notes and is extended by some final paragraphs.}

Linguistics is not explicitly discussed in the \citeyear{Fodor1974} article.\footnote{There is, however a note reference to Chomsky's (\citeyear{Chomsky1965}) statements about natural language predicates, to support Fodor's claim that natural kind predicates of the special sciences cross-classify the physical natural kinds.} However, Fodor has always incorporated linguistics in psychology, following Chomsky's views and elaborating this connectedness in more detail than Chomsky did (cf., e.g., \citealt[149]{Fodor1985}, quoted in footnote \ref{fn:elffers:nomologically}, and \citealt[278]{LoewerRey1991}). \emph{The language of thought} bears clear witness to this approach. So there can be no doubt that, for Fodor, linguistics is a special science. Other special sciences, such as economics and geology, are dealt with as instructive examples.

Sapir distinguishes three irreducible ontological categories: inorganic, organic and psychic. Inorganic sciences are physics, chemistry, and, partially, geology; psychology is the basic science of the psychic. Sapir does not mention examples of organic sciences, but we may assume that biology is the main, or even only, example of this category.

Sapir does not present arguments in favour of his trialistic ontology. He simply claims that ``the organic can be demonstrated to consist objectively of the inorganic plus an increment of obscure origin and nature''. There is ``a chasm between the organic and the inorganic which only the rigid mechanists pretend to be able to bridge. There seems to be a unbridgeable chasm […] between the organic and the psychic, despite the undeniable correlations between the two. Dr. Kroeber denies this \emph{en passant} […]'' \citep[444]{Sapir1917}.

These quotations show that Sapir is aware of the existence of divergent ontological ideas, but he does not feel obliged to supply arguments for his own view. This is not surprising when we take contemporary ontological thought into account. Vitalism, the idea that organic nature is created from chemical elements plus the action of a ``vital force'' had been waning over several decades, but was not at all extinct \citep[cf.][]{Beckner1967}. Psychology was, despite some reductionistic attempts, still largely regarded as studying purely mental entities. This applies, for example, to Gestalt psychology, an approach Sapir found appealing \citep[cf.][xvi]{Sapir1994}.

An example of a special science is, for Sapir, in the first place, social science, including anthropology, the common discipline of Kroeber and himself. Other examples are history – cf. Sapir's term ``historical sciences'' – and, partially, geology. Given the above examples of ``chunks of reality'' studied by historical sciences, we can add linguistics (cf. ``the French language'') and literary history (cf. ``the romantic movement in literature'').

In summary, Sapir's and Fodor's examples as well as ideas about the position of separate disciplines in their dichotomy are partially different. This is mainly due to their different basic ontologies and their implications, especially for psychology. A remarkable conclusion about linguistics is that its status of special science has a different meaning for Sapir and Fodor. For Sapir, a language is an ontologically heterogeneous entity. So linguistics is not reducible to psychology, nor to any other basic science. Fodor includes linguistics in psychology, but for him, psychology is itself a special science, due to ontological irreducibility. In section \ref{sec:elffers:linguisticsasspecialscience} I will return to this issue.

\section{Characterizing special sciences}
\label{sec:elffers:characterizing}

The categories/types/natural kinds of special sciences are ontologically heterogeneous, but
``experientially'' (Sapir) or ``functionally'' (Fodor) homogeneous. At first sight, these characterizations are dissimilar. Experiences are direct and unique, functions are conceptualized regularities. Therefore, when both scholars conclude that a certain discipline belongs to the special sciences, their reasons for the classification appear to be different. On the other hand, their common focus on areas consisting of human institutions (clans, economics) or ``interesting'' phenomena (mountains) suggests that they may share the same basic insight, but reconstruct it in different terms.

The shared example of geology may serve to clarify this point. For Sapir, geology is a special science, because it ``depends, for its \emph{raison d’être}, on a series of unique experiences, directly sensed or inferred, clustering about an entity (the earth, a mountain chain)'' \citep[445]{Sapir1917}. For Fodor, it is essential that mountains, however ontologically heterogeneous, enter into generalizations that ``serve geology in good stead. [… U]nimaginably complicated to-ings and fro-ings of bits and pieces at the extreme \emph{micro-level} manage somehow to converge on stable \emph{macro-level} properties'' \citep[160]{Fodor1997}. On the next page, these macro-level properties are equated with functional properties, as in psychology and biology.

My hypothesis is that these different characterizations are connected to the different discourses in which both scholars are operating. Sapir conceives of ``historical sciences'' as comparable to \emph{Geisteswissenschaften}, referring to \citet{Rickert1913}. This class of sciences is often characterized as ``idiographic'', and is contrasted with the ``nomothetic'' \emph{Naturwissenschaften}. Hence Sapir's emphasis on particular, directly experienced events and on ``the unique or individual, not the universal'' \citep[446]{Sapir1917}. At the same time, the above citation also refers to ``inferred'' experiences and later on, ``such concepts as a clan, a language, a priesthood'' are denied individual connotation and supposed to be involved in the same operations as natural science concepts: ``observation, classification, inference, generalization, and so on'' \citep[446]{Sapir1917}, exactly the operations Fodor frequently refers to with respect to all sciences.

Fodor's suggestion that, in special sciences, the generalizations are all of the functional type has, in turn, to be taken with a grain of salt. When applied to geology, the term ``functional'' is almost meaningless. Sapir's appeal to ``a certain philosophically arbitrary but humanly immensely significant selection out of the total mass of phenomena'', quoted above, seems to be a more adequate, but for Fodor undoubtedly too subjective, characterization of what special sciences are about, although he does not eschew the term ``interesting''.\footnote{In Fodor's ``Special sciences'', there are some references to the alleged ``interestingness'' or ``importance'' of the natural kinds of a special science. Compare the following passage about monetary exchange: ``The point is that monetary exchanges have interesting things in common. But what is interesting about monetary exchanges is surely not their commonalities under physical description'' \citep[103-104]{Fodor1974}.} So Sapir and Fodor appear to appeal to the same insight, worded differently.

There is another difference between Sapir's and Fodor's ideas about special sciences. In Sapir's examples, ontologically heterogeneous features are simultaneously realized, for example in the earth, or a mountain chain. In Fodor's special sciences, they are realized in different events (the ``tokens'') at different moments, for example in various monetary transactions.\footnote{Consequently, Fodor's presentation of the physical counterparts of a special science predicate as a disjunction does not apply to the physical counterparts in Sapir's examples. In these cases, they constitute a conjunction.} This difference is not entirely watertight, however. Sapir refers to events too (cf. the quotation on p. \pageref{q:elffers:sapirquote}). His incorporation of history in the special sciences and examples such as ``the French Republic, the romantic movement in literature, Victor Hugo'' also suggest that the heterogeneous counterparts of special science entities may be events. Fodor's extension of the class of special sciences to geology and his comparison with artefacts, in turn, implies that he also recognizes the possibility of simultaneous presence of heterogeneous features.

Certainly Sapir and Fodor did not have \emph{exactly} the same idea of special sciences in mind. But their ideas were more similar than their formulations suggest at first sight.

\section{Laws and exceptions}
\label{sec:elffers:lawsexceptions}

Sapir and Fodor are both convinced that special science laws have exceptions. For both scholars, scientific practice is an important argument. Sapir describes this practice and contrasts it with natural science practice: ``If, out of one  hundred clans, ninety-nine obeyed a certain sociological `law', we would justly flatter ourselves with having made a particularly neat and sweeping generalization; our `law' would have validity, even if we never succeeded in `explaining the one exception''' \citep[447]{Sapir1917}. According to Fodor, the idea that laws of special science are exceptionless has to be rejected because it ``flies in the face of fact. There is just no chance at all that all the true, counter-factual supporting generalizations of, say, psychology, will turn out to hold in strictly each and every condition where the antecedents are satisfied'' \citep[111]{Fodor1974}.

Both Sapir and Fodor thus take the requirement of \emph{historical adequacy} (conformity to clear cases of scientific practice) for philosophy of science seriously and derive a strong argument for exceptions to special science laws from actual scientific practice. When it comes to \emph{philosophical adequacy}, however, their arguments differ widely. Sapir appeals to his above-mentioned claim that special sciences are about particular events. ``Laws'' are actually abbreviations for a finite number of phenomena. Sapir admits that this is a complicated issue and adds here a footnote about Rickert for further reading.

Fodor's argument is entirely based upon the disjunctive character of the antecedent and the consequent of the physical counterpart of special science laws. The resulting physical ``law'' is not a genuine law (cf. section \ref{sec:elffers:superorganic} above) and this explains why special science laws have exceptions.

With respect to philosophical adequacy, Sapir's as well as Fodor's explanation appeals to the pseudo-lawlike character of special science ``laws''. However, the ways in which pseudo-lawlikeness is argued for are different.

Summarizing the last three sections, we may conclude that some aspects of token physicalism are elaborated in different ways by both scholars. These differences can be shown to be related to the temporal and intellectual context in which the theories were developed.

\section{Getting involved}
\label{sec:elffers:gettinginvolved}

In the following three sections, Sapir's and Fodor's token physicalism will be embedded in a wider context. The rise and development of their theories can be further clarified in this way. There is, firstly, the preliminary question of how they got involved in the problem of relations between disciplines and, secondly, whether their similar solutions were based on any clues in their intellectual environments. Finally, we may ask what, in general, became of Sapir's and Fodor's token physicalism. Neither Sapir nor Fodor was a specialist in general philosophy of science. During his student years, Sapir did not follow a philosophy programme, but his education in Germanic philology certainly yielded some knowledge of German philosophy, the breeding ground for the distinction between \emph{Naturwissenchaften} and \emph{Geisteswissenschaften}. Fodor was educated in philosophy. He was a pupil of Hilary Putnam and acquired a thorough knowledge of philosophy of science, but philosophy of mind became his specialization. Like many scientists, especially in the humanities and the social sciences, both scholars became involved in the broader issue of relations between disciplines through problems in their scientific work or through reflection on this work.

In Sapir's case, his master thesis on Herder's \emph{Ursprung der Sprache} \citep{Sapir1907} bears witness to an early interest in the foundations of linguistics, but he did not become involved in foundational issues again until \citeyear{Sapir1917}. Kroeber's article seems to have been the direct incentive for Sapir's development of token-physicalistic ideas. He must have been dissatisfied with Kroeber's ontological answer to the question of what social sciences are about. Sapir's title ``Do we need a `superorganic?'\thinspace'' reveals an Ockhamian approach: we must, if possible, avoid an unnecessary appeal to unknown and questionable entities such as Kroeber's superorganic force. Token physicalism supplied a promising alternative.

In Fodor's case, there is not, as far as I know, such a direct ``external'' occasion for his development of token physicalism. My hypothesis is that there was an ``internal'' occasion. As well as Putnam, Noam Chomsky, his MIT colleague, became very influential to Fodor's intellectual development. Fodor adopted Chomsky's mentalistic approach of claiming psychological reality for linguistic categories, rules etc. When Fodor wrote ``Special sciences'', he was probably simultaneously writing \emph{The language of thought}, a book which went further than Chomsky in postulating mental, and even innate, entities, structures and operations in the cognitive systems of thinking and communicating humans. Token physicalism could furnish a foundation for this approach by emphasizing the autonomy of psychology. The fact that the text of ``Special sciences'' constitutes the second section of the introduction to \emph{The language of thought} is an indication that Fodor saw it that way.\footnote{The introduction is titled ``Two kinds of reductionism''. Its two sections are named after views Fodor argues against: ``Logical behaviorism'' (about Wittgenstein's and Ryle's views of psychology) and ``Physiological reductionism''.} The final paragraphs of the introduction, absent in ``Special sciences'', confirm this suggestion. Compare the concluding sentences: ``It has […] been the burden of these introductory remarks that the arguments for […] the physical reduction of psychological theories are not, after all, very persuasive. The results of taking psychological theories literally and seeing what they suggest that mental processes are like might, in fact, prove interesting. I propose, in what follows to do just that'' \citep[26]{Fodor1975}.

\section{Clues to token physicalism}
\label{sec:elffers:clues}

Both Sapir and Fodor present their varieties of token physicalism as new ideas. Indeed, there were no earlier theories with this content. But there certainly were ideas of others which functioned as substantive building blocks or as sources of inspiration for their views.

Both scholars only refer briefly to fellow scholars in their texts. Apart from Kroeber, Sapir only refers to Rickert, in the footnote reference mentioned above (the only footnote in the article). Sapir characterizes Rickert's \emph{Die Grenzen der naturwissenschaftlichen Begriffsbildung} as ``difficult but masterly'' and continues: ``I have been greatly indebted to it.'' This is understandable: Rickert's way of distinguishing \emph{Geisteswissenschaften} and \emph{Naturwissenschaften}, not in terms their subject matter, as other philosophers would have it, but in terms of their ways of concept formation, appears to have inspired Sapir directly \citep[cf.][]{Anchor1967}.  Therefore, I do not share Silverstein's doubts about this indebtedness to Rickert: ``While Sapir, in his paper, expresses his debt to Rickert […], it is clearly Boas' discussion of 1887, the very phraseology and terms of which he repeats, that underlies his discourse'' \citep[70, fn.5]{Silverstein1986}.\label{q:elffers:silversteinref}

In any case, neither Rickert nor Boas developed anything comparable to token physicalism. Both scholars adopted the distinction between \emph{Geisteswissenschaften} and \emph{Naturwissenschaften}. But both assumed a much deeper chasm between the two types of science than Sapir did, by restricting \emph{Geisteswissenschaften} to a ``value-laden'' (Rickert) or ``affective'' (Boas) focus on \emph{individual} entities and regarding all generalizing thought as proper to natural sciences only.\footnote{See \citet{Anchor1967}  and \citet{Silverstein1986} for Rickert's and Boas' views, respectively.} We observed above that Sapir was also inclined to take into account the individuality of the phenomena described by special sciences. But he also recognized their clustering into abstract, generalized entities, which are subjected to operations such as ``classification, inference, generalization, and so on'' in these sciences. This view strongly deviates from Rickert's and Boas' views and is similar to Fodor's view. Both Sapir and Fodor claim that special sciences share their general methodology with basic sciences.

Fodor does not mention any indebtedness. Nevertheless, token physicalism is often regarded as similar to Putnam's idea of ``multiple realizability'', presented in several publications in the nineteen sixties (cf. \citealt{Putnam1960}, \citealt[xiii]{LoewerRey1991}). Multiple realizability is the thesis that the same mental property can be implemented by different physical properties. Actually, without mentioning the term, \citet[105-106]{Fodor1974} refers to this idea. He explicitly mentions Putnam's reference to computers as possible providers of physical counterparts of psychical events. Connections are also observed with Davidson's ``anomalous monism'', which, like Fodor's theory, restricts the links of the physical and the psychical to the level of events (cf. \citealt{Davidson1970}, \citealt[xxxi]{LoewerRey1991}). Fodor does not refer to Davidson's theory, but a reference to \citet{Davidson1970} in \emph{The language of thought} (p.200) proves that he knew about it. So Fodor's idea of how psychology reduces to physics was clearly prepared by other philosophers he knew about. Fodor, however, extended Putnam's and Davidson's solutions to the mind-body problem to a thesis about sciences in general, their typology and their characteristics as intellectual enterprises.

\section{What became of token physicalism?}
\label{sec:elffers:whatbecame}

Neither Sapir's nor Fodor's version of token physicalism was elaborated further by their authors after the publications discussed above. Two further questions will be explored now:

\begin{enumerate}
    \item[a.] Did token physicalism, as presented in these publications, play a role in their later work?
    \item[b.] Did token physicalism play a role in the work of later scholars?
\end{enumerate}

Answering these questions exhaustively is far beyond my limited state of knowledge, but this does not prevent me from making some tentative suggestions.

As to the first question, token physicalism, not surprisingly, ``sets the stage'' for Sapir's and Fodor's further research in their respective ``special sciences''. Sapir presents and practises linguistics and anthropology as autonomous sciences; Fodor's ``psychosemantics'' is also practised autonomously, without any appeal to specific brain states.\footnote{\emph{Psychosemantics} is the title of a \citeyear{Fodor1987} book by Fodor. I apply the term here to the totality of Fodor's work on cognitive psychology and its relations to semantics.} But in their writings, token physicalism is not at all prominent; it is a background framework rather than a major discussion theme.

For example, Sapir does not refer at all to his typology of sciences in his \citeyear{Sapir1929} article ``Linguistics as a science'', although the main theme of this article is the relation between linguistics and other sciences.\footnote{In \citet{Sapir1929}, linguistics is emphatically presented as a science aiming at generalization, explanation, laws etc., which is at odds with Silverstein’s (\citeyear{Silverstein1986}) idea that the views presented in \citet{Sapir1917}, as he interprets them in Boasian terms (cf. p.˜10), permeate Sapir's entire oeuvre.}  The conclusions drawn – e.g. that linguistics is not ``a mere adjunct of either biology or psychology'' \citep[214]{Sapir1929} – are in line with those drawn in \citeyear{Sapir1917}, but they are attained without any appeal to the distinction between conceptual sciences and historical sciences. The same is true of the passage about the definition of language in the first chapter of Sapir's seminal book \emph{Language} (\citeyear{Sapir1921}). There, Sapir claims that language cannot be defined ``as an entity in psycho-physical terms alone'' and that language can be discussed ``precisely as we discuss the nature of any other phase of human culture – say art or religion – as an institutional or cultural entity, leaving the organic and psychological mechanism back of it as something to be taken for granted'' \citep[10-11]{Sapir1921}.\label{q:elffers:huminst}

A clear echo of Sapir's discussion with Kroeber can be found in Irvine's reconstruction of Sapir's lectures on the psychology of culture, presented in the 1930s \citep{Sapir1994}. In a lecture on ``difficulties of the social sciences'', Sapir mentions the problem that ``the culturalist […] cannot be absolutely sure of the limits or bounds of what he is dealing with'', unlike physicists, who ``know what particular corner of the universe they are dealing with''. Another difficulty is the essential uniqueness of cultural phenomena. Referring to Rickert, Sapir contrasts the physicist, who deals with a conceptual universe, covering all possible phenomena, in an abstract way allowing for one hundred percent accuracy, with the social scientist, who studies all actual and unique phenomena, without this same level of accuracy \citep[56-57]{Sapir1994}. In another lecture, Sapir explicitly refers to his discussion with Kroeber. A sentence literally repeated from \citet[444]{Sapir1917} concludes the passage: ``Social science is not psychology, not because it studies the resultants of superpsychic or superorganic, but because its terms are differently demarcated'' \citep[245]{Sapir1994}. But again, none of these claims is argued for in terms of an explicit and general typology of sciences, as presented in \citet{Sapir1917}.

Fodor now and then refers to token physicalism after \citeyear{Fodor1975}. Like Sapir, he wrote an article which surprisingly omits the subject (``Some notes on what linguistics is about'', \citeyear{Fodor1985}). In an article about the mind-body problem in \emph{Scientific American} \citep{Fodor1981}, one paragraph is devoted to a brief explanation and defence of token physicalism as part of the solution to this problem. In his books \emph{Psychosemantics} \citep[5-6]{Fodor1987} and \emph{The elm and the expert} \citep[39]{Fodor1994}, the ``special science'' status of psychology is mentioned but, as in Sapir's case, without a reference to the broader context of token physicalism and the issue of typology of sciences.

The inconspicuous role of token physicalism in Fodor's work cannot be better illustrated than by the obituaries that appeared after his death on 29 November 2017. Of the eight obituaries I read, all paying ample attention to the content of Fodor's scientific work, only one, \citet{Rey2017}, mentions token physicalism.

Our second question about the role of token physicalism in the work of later scholars receives a negative answer in Sapir's case. As far as I know, Sapir's token physicalism \emph{avant la lettre} was not discussed by other scholars. His distinction between conceptual and historical sciences was neither adopted nor criticized by his linguistic or anthropological colleagues. Autonomy versus reducibility was an important and controversial issue for all humanities and social sciences, before and after 1917, but Sapir's solution does not seem to play any role in this multi-faceted discussion.\footnote{This tentative conclusion is based on a search in Google and Google Scholar for ``conceptual science''. No items were found containing this expression in the Sapirean sense.}

Fodor's token physicalism, on the other hand, became a rather popular issue in philosophical discussions, and remains so up to the present day.\footnote{The most recent article devoted to token physicalism I found is \citet{DiFrisco2017}.} Many philosophers of science have analysed and commented on Fodor's views. A considerable portion of their reactions are critical and try to vindicate some variety of reductive physicalism. As an example of the broad impact of Fodor's token physicalism, I would mention \emph{The Electric Agora} (``a modern symposium for the digital age''), which devoted a Special to ``Jerry Fodor's `Special sciences''' in 2015. After a brief introduction about ``one of the most influential essays in the philosophy of science since the Second World War'' \citep[1-2]{Kaufman2015}, thirty comments follow.

In areas outside philosophy, I found very few reactions to Fodor's token-physical\-istic ideas.\footnote{I found only six non-philosophical items via a Google Scholar search for ``token physicalism''.}  It is sometimes suggested that all practising cognitive scientists now adopt Fodor's line of thought and proceed without any appeal to neurology. For example, \citet{Jones2004}  claims that ``this [token physicalism] has been the consensus view among cognitive scientists since at least the mid-seventies'', due to Fodor's \citeyear{Fodor1974} article. This is, however, an overstatement. In the same article, when talking about belief states, Jones claims that their reduction to physical neurological states ``has been at the centre of numerous research projects in the behavioural and brain sciences for decades'' \citep[423]{Jones2004}. This recent observation shows the lasting validity of an earlier claim by Fodor himself that many psychologists are type physicalists who believe that every psychological kind predicate is lawfully related to a neurological kind predicate and that ``there are departments of psycho-biology or psychology and brain science in universities throughout the world whose very existence is an institutionalized gamble that such lawful coextensions can be found'' (\citealt[105]{Fodor1974}, cf. footnote \ref{fn:elffers:physpsych}).

In summary, Sapir's token physicalism seems not to have left traces in the work of later scholars.\footnote{There might be traces in work I did not consult: later publications by Kroeber, or his correspondence with Sapir, which lasted for several decades. On p. \pageref{q:elffers:silversteinref}, \citet{Silverstein1986} was mentioned for drawing attention to Sapir's token physicalism from the perspective of the history of linguistics.} Fodor's token physicalism was only partially influential as a programme for research in cognitive science. But it did become the subject of a lively philosophical debate that is still ongoing.

\section{Linguistics as a ``special science''}
\label{sec:elffers:linguisticsasspecialscience}

Recently, a newly appointed professor of Dutch Linguistics at Leiden University claimed, in his inaugural lecture, that linguistics is in crisis because it is thought it may become superfluous fairly soon. Language, as a cognitive phenomenon, can now be investigated through brain research, so why should there be a separate discipline of linguistics alongside neurology? The answer is that the role of linguistics has not yet become entirely irrelevant because the help of linguists is still necessary for the correct interpretation of the neurocognitive data \citep{Barbiers2017}.

This line of argument, which presupposes correspondences between linguistic and neurological natural kinds, is a clear example of reductive, type-physicalistic thought. Such a radically reductive view of linguistics is not new, but recent developments in neurolinguistics have made it much more prominent and much more applicable (and actually applied) in research practice. But it is not, and never was, the only view. On the contrary, there are many linguistic approaches that do not make any appeal to neurology, either because of a more autonomous psychologistic conception of cognitive-linguistic research, or because of a more radically autonomous, non-psychologistic view of linguistics (cf. \citealt{Botha1992}; \citealt{Elffers2014}).

Thus far, token physicalism does not play a role in discussions about these approaches. This might be due to the context in which it was introduced – anthropology in Sapir's case, philosophy of science in Fodor's case. Neither \citet{Sapir1917} nor \citet{Fodor1974} explicitly refer to linguistics but, as was argued in section \ref{sec:elffers:basicspecial}, both scholars certainly incorporated linguistics in the category of special sciences, even if this incorporation has a different meaning for Sapir's and Fodor's varieties of token physicalism. For Sapir, psychology is a basic science. Linguistics, as a special science, has a non-psychological status. Language belongs, with art and culture, to the category of ``human institutions'' (cf. the quotation on p. \pageref{q:elffers:huminst}). For Fodor, only physics is a basic science; psychology is a special science. Given Fodor's psychologistic view of linguistics, linguistics is also a special science.\footnote{\citet[149]{Fodor1985} claims that ``it is nomologically necessary that the grammar of a language is internally represented by speaker/hearers of that language''. In itself, Fodor's token physicalism allows for an ``institutional'' interpretation of language as well (cf. his discussion of economics). \citet[422-423]{Jones2004} regards multiple realizability as a typical feature of institutional facts in general.\label{fn:elffers:nomologically}}

Can token physicalism, if plausible at all, play a relevant role in the discussions of linguistic approaches examined above? A positive answer seems possible. The conception of linguistics as a ``special science'' can play a supportive role in the argumentation of both psychologists and non-psychologists. For psychologists, token physicalism can help to justify the fact that they do not appeal to neurology. Thus far this justification is often lacking or unconvincing. For example, many cognitive linguists \citep[cf., e.g., ][]{Langacker1999} make strong statements about mental architecture and processes, without discussing questions of neurological reality. \citet[5-6]{Chomsky1987} claims that such discussions are unnecessary, because chemists, too, ``have not stopped to discuss `abstractly construed' molecule elements, the periodic system and so on''. The analogy fails, because chemists \emph{could} apply the vocabulary of atomic physics instead, whereas linguists are far from knowing what corresponds neurologically to their psychological-linguistic categories. Token physicalism provides a better justification for not discussing neurological equivalents of linguistic natural kinds.

For non-psychologists, the autonomy of linguistics often implies a rather problematic ontological status of language. Like Kroeber, they look for an ontological answer to questions of non-reducibility. For example, according to \citet{Cooper1975}, language belongs to a separate ``linguistic reality'', \citet{Itkonen1978}  assumes a non-empirical ``social reality'' which incorporates language, \citet{Katz1981} localizes language in an abstract ``Platonist'' realm. In all cases, there is, apart from Ockhamian considerations, the problem of explaining the interaction of these separate realms with the psychological realm of actual use and knowledge of language. Without suggesting that token physicalism offers ready-made solutions, I feel that it has certain advantages: language use consists of (psycho-)physical events (tokens) and the linguist's constructs they instantiate (types, natural kinds) are epistemologically but not ontologically autonomous.\footnote{For non-psychologists, Sapir's variety of token physicalism is, of course, a better example than Fodor's. Ironically, Itkonen is the only one of these more recent scholars who made a thorough study of Sapir's work, including his anthropological publications, but he seems to have missed \citet{Sapir1917}, and his interpretation of Sapir's view as identical to his own non-empirical view of linguistics is a mistake \citep[cf.][62-65]{Itkonen1978}.}

Of course, this extension of the topic in the final paragraphs of this chapter is too fragmentary. But it may give an impression of how the idea of linguistics as a ``special science'' can play a role in discussions of linguistic approaches.

\sloppy
\printbibliography[heading=subbibliography,notkeyword=this] 
\end{document}